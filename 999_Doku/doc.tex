% !TeX spellcheck = de_CH
%%%%%%%%%%%%%%%%%%%%%%%%%%%%%%%%%%%%%%%%%%%%%%%%%%%%%%%%%%%%%%%%%
%  _____   ____  _____                                          %
% |_   _| /  __||  __ \    Institute of Computitional Physics   %
%   | |  |  /   | |__) |   Zuercher Hochschule Winterthur       %
%   | |  | (    |  ___/    (University of Applied Sciences)     %
%  _| |_ |  \__ | |        8401 Winterthur, Switzerland         %
% |_____| \____||_|                                             %
%%%%%%%%%%%%%%%%%%%%%%%%%%%%%%%%%%%%%%%%%%%%%%%%%%%%%%%%%%%%%%%%%
%
% Project     : BA Welti Keller
% Title       : 
% File        : doc.tex Rev. 00
% Date        : 15.09.2014
% Author      : Tobias Welti
%
%%%%%%%%%%%%%%%%%%%%%%%%%%%%%%%%%%%%%%%%%%%%%%%%%%%%%%%%%%%%%%%%%

\include{header}

\begin{document}
\title{Bachelorarbeit (IT)}
\author{Tobias Keller, Tobias Welti}

% !TeX spellcheck = de_CH
%%%%%%%%%%%%%%%%%%%%%%%%%%%%%%%%%%%%%%%%%%%%%%%%%%%%%%%%%%%%%%%%%
%  _____   ____  _____                                          %
% |_   _| /  __||  __ \    Institute of Computitional Physics   %
%   | |  |  /   | |__) |   Zuercher Hochschule Winterthur       %
%   | |  | (    |  ___/    (University of Applied Sciences)     %
%  _| |_ |  \__ | |        8401 Winterthur, Switzerland         %
% |_____| \____||_|                                             %
%%%%%%%%%%%%%%%%%%%%%%%%%%%%%%%%%%%%%%%%%%%%%%%%%%%%%%%%%%%%%%%%%
%
% Project     : BA Welti Keller
% Title       : 
% File        : titlepage.tex Rev. 00
% Date        : 15.09.2014
% Author      : Tobias Welti
%
%%%%%%%%%%%%%%%%%%%%%%%%%%%%%%%%%%%%%%%%%%%%%%%%%%%%%%%%%%%%%%%%%

\begin{titlepage}

% Logo
\ThisTileWallPaper{\paperwidth}{\paperheight}{images/logos/InES.pdf} % {}images/logos/*.pdf}
% Wählen Sie aus folenden pdf Files: ICP, IDP, IEFE, IMES, IMPE, IMS, INE, InES, InIT, KSR, SoE, ZAMP, ZAV, ZIL, ZPP, ZSN

\begin{minipage}[b]{0.117\textwidth}
\hskip 0.05cm
\end{minipage}
\begin{minipage}[b]{0.91\textwidth}
\begin{tiny}.\end{tiny}\vskip 2.8cm
	{\huge
	
	% Projekt Name
	\textbf{\underline{Bachelorarbeit}}
	
	% Projekt Titel
	Messstation zur Registrierung von Geschiebe-Bewegungen im Fluss
	\vskip 0.5cm}
	
	\begin{minipage}[b]{0.27\textwidth}
	\hrule\vskip 0.5cm
		\textbf{Autoren}\\
		\\
	\end{minipage}
	\begin{minipage}[b]{0.03\textwidth}
	\hskip 0.5cm
	\end{minipage}
	\begin{minipage}[b]{0.7\textwidth}
	\hrule\vskip 0.5cm
		Tobias Keller\\
		Tobias Welti\\
	\end{minipage}
	
	\begin{minipage}[b]{0.27\textwidth}
	\hrule\vskip 0.5cm
		\textbf{Betreuer}\\
		\\
	\end{minipage}
	\begin{minipage}[b]{0.03\textwidth}
	\hskip 0.5cm
	\end{minipage}
	\begin{minipage}[b]{0.7\textwidth}
	\hrule\vskip 0.5cm
		Prof. Hans-Joachim Gelke, Dipl. El. Ing FH\\
		ZHAW Institute for Embedded Systems\\
	\end{minipage}
	
	\begin{minipage}[b]{0.27\textwidth}
	\hrule\vskip 0.5cm
		\textbf{Partner}\\
		\\
	\end{minipage}
	\begin{minipage}[b]{0.03\textwidth}
	\hskip 0.5cm
	\end{minipage}
	\begin{minipage}[b]{0.7\textwidth}
	\hrule\vskip 0.5cm
	  Carlos Rodrigo Wyss\\
		Eidg. Forschungsanstalt für Wald, Schnee und Landschaft WSL\\
	\end{minipage}
	
	\begin{minipage}[b]{0.27\textwidth}
	\hrule\vskip 0.5cm
		\textbf{Datum}
	\end{minipage}
	\begin{minipage}[b]{0.03\textwidth}
	\hskip 0.5cm
	\end{minipage}
	\begin{minipage}[b]{0.7\textwidth}
	\hrule\vskip 0.5cm
		\today
	\end{minipage}
\end{minipage}
\vskip 0.5cm

\end{titlepage}

\includepdf{images/Erklaerung_BA.pdf}
% !TeX spellcheck = de_CH
%%%%%%%%%%%%%%%%%%%%%%%%%%%%%%%%%%%%%%%%%%%%%%%%%%%%%%%%%%%%%%%%%
%  _____   ____  _____                                          %
% |_   _| /  __||  __ \    Institute of Computitional Physics   %
%   | |  |  /   | |__) |   Zuercher Hochschule Winterthur       %
%   | |  | (    |  ___/    (University of Applied Sciences)     %
%  _| |_ |  \__ | |        8401 Winterthur, Switzerland         %
% |_____| \____||_|                                             %
%%%%%%%%%%%%%%%%%%%%%%%%%%%%%%%%%%%%%%%%%%%%%%%%%%%%%%%%%%%%%%%%%
%
% Project     : BA Welti Keller
% Title       : 
% File        : vorwort.tex Rev. 00
% Date        : 15.09.2014
% Author      : Tobias Welti
%
%%%%%%%%%%%%%%%%%%%%%%%%%%%%%%%%%%%%%%%%%%%%%%%%%%%%%%%%%%%%%%%%%

\chapter*{Vorwort}\label{chap.vorwort}
\todo{Vorwort: Hier ein bisschen blabla von dir und mir}

\setcounter{page}{1}
\include{content/kontakt}


%Inhaltsverzeichnis
\tableofcontents
\listoftodos
\newpage


%Kapitel
%\setcounter{page}{1}
%\pagenumbering{arabic}

% !TeX spellcheck = de_CH
%%%%%%%%%%%%%%%%%%%%%%%%%%%%%%%%%%%%%%%%%%%%%%%%%%%%%%%%%%%%%%%%%
%  _____   ____  _____                                          %
% |_   _| /  __||  __ \    Institute of Computitional Physics   %
%   | |  |  /   | |__) |   Zuercher Hochschule Winterthur       %
%   | |  | (    |  ___/    (University of Applied Sciences)     %
%  _| |_ |  \__ | |        8401 Winterthur, Switzerland         %
% |_____| \____||_|                                             %
%%%%%%%%%%%%%%%%%%%%%%%%%%%%%%%%%%%%%%%%%%%%%%%%%%%%%%%%%%%%%%%%%
%
% Project     : BA Welti Keller
% Title       : 
% File        : einleitung.tex Rev. 00
% Date        : 15.09.2014
% Author      : Tobias Welti
%
%%%%%%%%%%%%%%%%%%%%%%%%%%%%%%%%%%%%%%%%%%%%%%%%%%%%%%%%%%%%%%%%%



\chapter{Einleitung}\label{chap.einleitung}

\section{Ausgangslage}\label{sec.ausgangslage}
Die \gls{wsl} betreibt Messstationen zur Registrierung von Geschiebe-Bewegungen im Fluss mittels \glspl{geophon}n, die unter Stahlplatten montiert sind. Diese Platten sind in einer Betonkonstruktion eingelassen, um sie im Flussbett zu fixieren. Die Betonkonstruktion dient gleichzeitig als Kabelkanal. Jedes \glspl{geophon} ist über ein Kabel mit einem Auswertungs-Rechner (Embedded PC) verbunden. Der Rechner wertet die Signale aller angeschlossen \glspl{geophon} kontinuierlich aus, um die Ereignisse zu detektieren. Bei mehreren \glspl{geophon}n ist hier ein recht leistungsfähiger Rechner nötig, der eine entsprechend hohe Leistungsaufnahme hat. Die baulichen Massnahmen für die Installation der \glspl{geophon}, der Auswertungsstation sowie der Stromversorgung sind sehr teuer. Da viele dieser Messstationen in Gebirgsbächen installiert werden, fallen hohe Transportkosten für die schweren Materialien an.

Wenn die Sensoren in einer Elastomerplatte integriert werden könnten und die Komplexität der Verkabelung reduziert würde, liessen sich viele dieser baulichen Massnahmen umgehen. Dies würde die Kosten drastisch senken.

\todo{was gibt es bis jetzt? wie wirds gemacht? was gäbe es für alternativen?Mu
\\LITERATURVERWEISE!}


% !TeX spellcheck = de_CH
%%%%%%%%%%%%%%%%%%%%%%%%%%%%%%%%%%%%%%%%%%%%%%%%%%%%%%%%%%%%%%%%%
%  _____   ____  _____                                          %
% |_   _| /  __||  __ \    Institute of Computitional Physics   %
%   | |  |  /   | |__) |   Zuercher Hochschule Winterthur       %
%   | |  | (    |  ___/    (University of Applied Sciences)     %
%  _| |_ |  \__ | |        8401 Winterthur, Switzerland         %
% |_____| \____||_|                                             %
%%%%%%%%%%%%%%%%%%%%%%%%%%%%%%%%%%%%%%%%%%%%%%%%%%%%%%%%%%%%%%%%%
%
% Project     : BA Welti Keller
% Title       : 
% File        : aufgabenstellung.tex Rev. 00
% Date        : 15.09.2014
% Author      : Tobias Welti
%
%%%%%%%%%%%%%%%%%%%%%%%%%%%%%%%%%%%%%%%%%%%%%%%%%%%%%%%%%%%%%%%%%

\chapter{Aufgabenstellung}\label{chap.aufgabenstellung}

Die offizielle Aufgabenstellung befindet sich im Anhang \ref{app.aufgabenstellung}.

\section{Aufgabenstellung}\label{sec.aufgabenstellung}
Im Rahmen dieser Bachelorarbeit soll eine Lösung erarbeitet werden, um zukünftige Installationen günstiger zu machen. Da solche Messanlagen an sehr vielen Orten auf der ganzen Welt aufgebaut werden, kann durch eine Vereinfachung der Installation viel Aufwand gespart werden. 

Die Projektidee stammt von Bruno Fritschi (WSL). Sein Vorschlag sieht vor, die aufgezeichneten Signale direkt am \gls{sensor} auszuwerten und nur die gewünschten \gls{ereignis}-Daten zu übertragen und zu speichern. Somit könnten die Daten über ein \gls{bussys} übertragen werden und der Rechner für die Sammlung der Daten bräuchte weniger Rechenleistung.

Dank der Bustopologie kommt das Messsystem mit weniger Leitungen aus und kann einfacher installiert werden. Denkbar wäre die Integration in einer Elastomerplatte anstelle der Stahl- und Betonkonstruktion, da viel weniger Leitungen nötig sind. Die Elastomerplatte könnte einfacher im Bachbett verankert werden.

Ziel der Arbeit ist die Entwicklung der Auswertungshardware und des \gls{bussys}s. Die Qualität der gemessenen Signale soll mindestens erhalten werden. Die Auswertungsalgorithmen sind nicht Bestandteil der Arbeit und werden vom WSL zur Verfügung gestellt.

Die von der bisherigen Anlage gemachten Messdaten enthalten die Dauer und Intensität jedes Aufschlags (\gls{ereignis}) auf der Sensorplatte, sowie die Anzahl Ausschläge (Peaks) pro Aufschlag. Pro Minute wird ein Histogramm über die Intensitäten der Peaks gebildet und abgespeichert.

Denkbar wäre es, einen Prototyp für Vergleichsmessungen im Erlenbach (Alptal, SZ) an einer bestehenden Schwelle zu implementieren.


\subsection{Musskriterien}
\begin{itemize}
\item Die Anlage zeichnet den Geschiebetransport im Bachbett auf. Die bisherige Aufzeichnungsrate von 10'000 Messpunkten pro Sekunde soll nicht unterschritten werden.
\item Die Anlage liefert eine minütliche Zusammenfassung über die \glspl{ereignis} an jedem \gls{sensor}. Diese Zusammenfassung enthält die Anzahl \glspl{ereignis}, Dauer und Intensität der einzelnen \glspl{ereignis} sowie ein Histogramm über die Intensitätsverteilung.
\item Die Messstation ist fähig, mindestens zehn \glspl{sensor} zu betreiben und ihre Messignale aufzuzeichnen.
\item Es ist möglich, die kompletten Rohdaten von einem \gls{sensor} über eine Dauer von 30 Minuten aufzuzeichnen. Während einer solchen Messung dürfen die anderen \glspl{sensor} ihre Messung einstellen.
\item Die \glspl{sensor} können über bis zu fünfzehn Meter im Bachbett verteilt sein.
\item Die Leistungsaufnahme der Anlage beim Betrieb von 10 \glspl{sensor} ist kleiner als zehn Watt.
\item Die Datenaufzeichnung erfolgt in einem eigens entwickelten \gls{logger}.
\item Am \gls{logger} kann ein Laptop angeschlossen werden, um Kontrollparameter der Messanlage zu setzen und um den Status der Anlage abzufragen.
\item Die erfassten Messdaten werden im \gls{logger} auf einer Speicherkarte gespeichert. Dies ermöglicht ein einfaches Abholen der Daten im Feld, indem die Speicherkarte ausgetauscht wird.
\end{itemize}
\subsection{Wunschkriterien}
\begin{itemize}
\item Die Anlage liefert für jedes \gls{ereignis} die Rohdaten in voller zeitlicher Auflösung.
\item Der Sensoraufbau ermöglicht es, die \glspl{sensor} in einer Elastomerplatte zu verpacken. Die Elastomerplatte kann ohne Betonkonstruktion im Bachbett verankert werden.
\item Am \gls{logger} kann ein Laptop angeschlossen werden, um die erfassten Messdaten herunterzuladen.
\end{itemize}
\subsection{Abgrenzungskriterien}
\begin{itemize}
\item Es würde den Rahmen dieser Arbeit sprengen, die Messeinheiten zur Produktreife zu bringen. Es wird lediglich aufgezeigt, wie solche Messeinheiten realisiert werden könnten.
\item Eine Testinstallation in einem Bach ist nicht möglich. Allenfalls kann in der Versuchsanstalt für Wasserbau, Hydrologie und Glaziologie der ETH Zürich ein kleiner Testlauf stattfinden.
\item
\end{itemize}


% !TeX spellcheck = de_CH
%%%%%%%%%%%%%%%%%%%%%%%%%%%%%%%%%%%%%%%%%%%%%%%%%%%%%%%%%%%%%%%%%
%  _____   ____  _____                                          %
% |_   _| /  __||  __ \    Institute of Computitional Physics   %
%   | |  |  /   | |__) |   Zuercher Hochschule Winterthur       %
%   | |  | (    |  ___/    (University of Applied Sciences)     %
%  _| |_ |  \__ | |        8401 Winterthur, Switzerland         %
% |_____| \____||_|                                             %
%%%%%%%%%%%%%%%%%%%%%%%%%%%%%%%%%%%%%%%%%%%%%%%%%%%%%%%%%%%%%%%%%
%
% Project     : BA Welti Keller
% Title       : 
% File        : vorgehen.tex Rev. 00
% Date        : 15.09.2014
% Author      : Tobias Welti
%
%%%%%%%%%%%%%%%%%%%%%%%%%%%%%%%%%%%%%%%%%%%%%%%%%%%%%%%%%%%%%%%%%

\chapter{Vorgehen}\label{chap.vorgehen}
\section{Überblick}\label{sec.ueberblick}
Das zu entwickelnde Messsystem kann grob in drei Komponenten aufgeteilt werden. 
\begin{enumerate}
\item \gls{logger}
\item \gls{sensoreinh}
\item \gls{bussys}
\end{enumerate}
Der \gls{logger} hat die Aufgabe, von mehreren \glspl{sensoreinh} registrierte \glspl{ereignis} zu empfangen und zu speichern. Die \glspl{sensoreinh} messen kontinuierlich die Beschleunigung, werten die Signale aus und erkennen \glspl{ereignis}, die einer vordefinierten Signalform entsprechen. Alle \glspl{sensoreinh} sind über ein \gls{bussys} mit dem \gls{logger} verbunden, um miteinander kommunizieren zu können. Der prinzipielle Aufbau ist in Abbildung \ref{fig.situationskroki} ersichtlich. Die Stromversorgung der Anlage wird am \gls{logger} angeschlossen. Parallel zum Kabel des Datenbusses wird die Stromversorgung der \glspl{sensoreinh} geführt.

\begin{figure}
	\centering
		\includegraphics[width=0.8\textwidth]{images/visio/Situationskroki.pdf}
	\caption{Eine Messstation mit einem \gls{logger}, der mehrere \glspl{sensoreinh} im Bach steuert.}
	\label{fig.situationskroki}
\end{figure}

Diese drei Einheiten werden im Folgenden genauer definiert.

\section{Datenlogger}
Der \gls{logger} hat verschiedene Aufgaben zu erfüllen:
\begin{itemize}
\item Sammeln und speichern der Messdaten der \glspl{sensoreinh}.
\item Kontrolle über das \gls{bussys}.
\item Steuerung des Betriebs der Anlage.
\item Schnittstelle für die Konfiguration der Anlage und für das Auslesen der Messdaten.
\end{itemize}


\subsection{Messdaten sammeln}
Für jede angeschlossene \gls{sensoreinh} führt der \gls{logger} eine Datensammlung, in der die registrierten \glspl{ereignis} zeitlich sortiert abgespeichert werden. Die Datensammlungen werden in Dateien abgelegt, die auf einem leicht auswechselbaren Medium abgespeichert werden. So können die Messdaten auf einfache Art für die weitere Auswertung abgeholt werden.


\subsection{Kontrolle über das Bussystem}
Als Busmaster hat der \gls{logger} die Aufgabe, allen angeschlossenen Einheiten eine eindeutige \gls{id} zuzuweisen. Über diese \gls{id} erkennt der \gls{logger}, von welcher \gls{sensoreinh} Daten übertragen werden. Anhand der \gls{id} kann der Datenlogger Konfigurationsnachrichten an bestimmte \glspl{sensoreinh} addressieren. Für die Zuordnung der Messdaten zu einem bestimmten \gls{sensor} benötigen die \glspl{sensoreinh} ein fixes Erkennungsmerkmal, z.B. eine Seriennummer, die mit den Messdaten abgespeichert werden soll. Wurde einer \gls{sensoreinh} einmal eine \gls{id} zugeteilt, wird diese Zugehörigkeit in einer Konfigurationsdatei auf dem \gls{logger} abgespeichert, um bei einem Neustart des Systems die gleichen \gls{id}s zu vergeben.


\subsection{Steuerung des Betriebs}
Die Messstation hat verschiedene \glspl{modus}, die über den \gls{logger} angewählt werden können. Der \gls{logger} steuert die einzelnen \glspl{sensoreinh} entsprechend an. Je nach \gls{modus} werden mehr oder weniger detailreiche Daten über die Ereignisse abgespeichert.


\subsection{Schnittstelle nach Aussen}
Über eine Schnittstelle am \gls{logger} kann ein \gls{compi} angeschlossen werden. Per \gls{cli} wird die Messstation konfiguriert, der Zustand überprüft und der \gls{modus} gewählt.


\section{Sensoreinheit}
Die Aufgaben der \gls{sensoreinh} umfassen:
\begin{itemize}
\item Erfassung von Messdaten.
\item Erkennung von \glspl{ereignis}n.
\item Übertragung der Ereignisdaten an den \gls{logger}.
\end{itemize}


\subsection{Messdatenerfassung}
Der \gls{sensor} zur Erfassung der Daten wird mit einer vordefinierten Abtastrate ausgelesen. Die Abtastrate muss so gewählt werden, dass einzelne \glspl{ereignis} erkannt werden können, ohne unnötig viel Messdaten zu generieren.

\subsection{Ereigniserkennung}
Im Mikroprozessor werden die Messdaten fortlaufend analysiert. Überschreitet das gemessene Signal einen gewissen Schwellenwert (\gls{threshold}), markiert dies den Beginn eines \gls{ereignis}ses. Das \gls{ereignis} ist beendet, wenn der Signalpegel für eine gewisse Zeit (\gls{timeout}) unterhalb des \gls{threshold} bleibt. Für jedes \gls{ereignis} wird abgespeichert, wann es aufgetreten ist (\gls{timestamp}), wie lange es gedauert hat, wie hoch der Signalpegel maximal ausschlug und wie viele Signalspitzen (\glspl{peak}) aufgetreten sind. Allenfalls können auch die Höhen und \glspl{timestamp} aller \glspl{peak} übertragen werden.

\subsection{Datenübertragung}
Die \gls{sensoreinh} sendet die Messdaten regelmässig über das \gls{bussys} an den \gls{logger}. Nach Bestätigung des Erhalts werden die Messdaten aus dem Speicher der \gls{sensoreinh} gelöscht.

\section{Bussystem}
Das \gls{bussys} verbindet die Einheiten der Messstation miteinander. Die gesamten Messdaten und Steuerkommandos werden über den Bus übertragen. Das \gls{bussys} muss die Datenmenge der angeschlossenen \glspl{sensor} bewältigen können, über die geforderte Distanz funktionieren und möglichst robust gegenüber äusseren Einflüssen sein. Der Busmaster hat die Möglichkeit, laufende Übertragungen von \glspl{sensoreinh} zu unterbrechen, um Steuerkommandos zu senden.
%%%%%%%%%%%%%%%%%%%%%%%%%%%%%%%%%%%%%%%%%%%%%%%%%%%%%%%%%%%%%%%%%
%  _____   ____  _____                                          %
% |_   _| /  __||  __ \    Institute of Computitional Physics   %
%   | |  |  /   | |__) |   Zuercher Hochschule Winterthur       %
%   | |  | (    |  ___/    (University of Applied Sciences)     %
%  _| |_ |  \__ | |        8401 Winterthur, Switzerland         %
% |_____| \____||_|                                             %
%%%%%%%%%%%%%%%%%%%%%%%%%%%%%%%%%%%%%%%%%%%%%%%%%%%%%%%%%%%%%%%%%
%
% Project     : LaTeX doc Vorlage für Windows ProTeXt mit TexMakerX
% Title       : 
% File        : abstract.tex Rev. 00
% Date        : 23.4.12
% Author      : Remo Ritzmann
% Feedback bitte an Email: remo.ritzmann@pfunzle.ch
%
%%%%%%%%%%%%%%%%%%%%%%%%%%%%%%%%%%%%%%%%%%%%%%%%%%%%%%%%%%%%%%%%%

\thispagestyle{empty}
\chapter{Funktionale Anforderungen}\label{sec.funktionale}
\section{Datenlogger}
\begin{description}
\item{Busmaster}: Der Datenlogger übernimmt die Kontrolle des CAN-Bus. 

\item{Sensorerkennung}: Die angeschlossenen Sensoren werden vom Datenlogger erkannt und mit einer ID versehen. Anhand der ID wird die Priorität bei der Datenübertragung festgelegt und der Sensor identifiziert.\\
Können wir die Seriennummer des Boards auslesen? Damit die ID immer gleich ist... über UART-Kommandozeile muss die ID jedes Sensors gesetzt werden können.



\item{Daten sammeln}: Der Datenlogger fragt in regelmässigen Abständen bei den Sensoreinheiten an, ob Ereignisdaten zur Übertragung bereit sind.\\
Token vergeben? Übermitteln in Zeitfenster, gewisse Datenmenge oder bis Daten fertig? Dürfen andere Sensoren verhungern=überlaufen? Welches Polling-Intervall?

\item{Timestamp verteilen}: Der Datenlogger sendet ein Signal an alle Sensoreinheiten, dass der Timestamp auf Null gestellt werden soll. Ab dann beziehen sich die Timestamps auf die Dauer seit dem jetztigen Zeitpunkt.\\
Wie wird Uhrzeit eingestellt? Setupfile auf SD-Karte? UART? Wie genau kann die Uhrzeit an die Sensoren übergeben werden?

\item{Schnittstelle zum Steuerrechner}: Der Datenlogger bietet eine Schnittstelle, wo ein Steuerrechner (Laptop, PC) angeschlossen werden kann. Über diese Schnittstelle kann der Betrieb der ganzen Anlage gesteuert werden.\\
 UART? Können hierüber auch Daten ausgelesen werden? Oder nur über SD-Karte?

\item{Daten speichern}: Die Daten werden vom Datenlogger auf einer Speicherkarte in Dateien abgelegt. Nach entsprechenden Befehlen vom Steuerrechner kann die Karte entfernt und ausgetauscht werden, um die Daten abzuholen.\\
Alle Files auf Karte schliessen, damit diese gewechselt werden kann.\\
Bei Blackout: was passiert mit den Daten? Worst Case: Datenlogger crasht, wie können Daten ausgelesen werden?

\item{Steuerung Betriebsmodus}: Der Betriebsmodus der Sensoren wird vom Datenlogger aus gesteuert: Wie viele und welche Art von Daten gesammelt werden soll und ob alle Sensoren oder nur bestimmte aktiv sein sollen. \\
Nur Ereignisdaten von allen Sensoren oder Rohdaten von einem einzelnen Sensor? Rohdaten evtl. nur bei angeschlossenem Rechner wegen Speicherplatz, direkt auf Rechner übermitteln? (eher nicht, braucht wieder Protokoll, evtl noch Tool auf dem Rechner)

\end{description}

\section{Sensoreinheit}
\begin{description}
\item{Detektion}: Die Sensoreinheit liest den Sensor in definierten Zeitabständen aus und wertet die Messdaten aus. Zu jedem Ereignis werden folgende Daten gespeichert: Zeitpunkt (Timestamp), Dauer, Anzahl Peaks und höchster Peak. In einem zweiten Betriebsmodus können alle Messpunkte während einem Ereignis gespeichert werden.

\item{Datenübertragung}: Die Daten zu den Ereignissen werden an den Datenlogger übermittelt.

\item{Rohdatenaufzeichnung}: In einem Sondermodus werden alle Messpunkte gespeichert und an den Datenlogger übertragen. In diesem Betriebsmodus  Aufzeichnung und Speicherung von Rohdaten. Annahme 12 bit/Sample bei 10kHz => 120 kbit/s => Kein Problem, CAN Bus kann bis 1 Mbit/s => es könnten unter Umständen sogar mehrere Sensoren Rohdaten übermitteln. Mit 1 GB Speicher auf dem Logger könnten etwas mehr als 19 Stunden aufgezeichnet werden.
\end{description}

%%%%%%%%%%%%%%%%%%%%%%%%%%%%%%%%%%%%%%%%%%%%%%%%%%%%%%%%%%%%%%%%%
%  _____   ____  _____                                          %
% |_   _| /  __||  __ \    Institute of Computitional Physics   %
%   | |  |  /   | |__) |   Zuercher Hochschule Winterthur       %
%   | |  | (    |  ___/    (University of Applied Sciences)     %
%  _| |_ |  \__ | |        8401 Winterthur, Switzerland         %
% |_____| \____||_|                                             %
%%%%%%%%%%%%%%%%%%%%%%%%%%%%%%%%%%%%%%%%%%%%%%%%%%%%%%%%%%%%%%%%%
%
% Project     : Pflichtenheft für WeKeBA
% Title       : 
% File        : nichtfunktionale.tex Rev. 00
% Date        : 15.09.2014
% Author      : Tobias Welti
%
%%%%%%%%%%%%%%%%%%%%%%%%%%%%%%%%%%%%%%%%%%%%%%%%%%%%%%%%%%%%%%%%%

\thispagestyle{empty}
\chapter{Nichtfunktionale Anforderungen}\label{chap.nichtfunktionale}

\begin{itemize}
\item Die gesamte Messstation soll eine geringere Leistungsaufnahme haben als eine aktuelle Messstation mit Geophonen. Für zehn Geophone sind dies zur Zeit zehn Watt.
\item Die Installation einer Messstation soll weniger bauliche Massnahmen erfordern als eine aktuelle Messstation mit Geophonen.
\item Die erfassten Daten sollen mehr Details enthalten als die gegenwärtig erfassten Daten.
\item Sensoren müssen wasserdicht verpackt werden können.
\
\end{itemize}
% !TeX spellcheck = de_CH
%%%%%%%%%%%%%%%%%%%%%%%%%%%%%%%%%%%%%%%%%%%%%%%%%%%%%%%%%%%%%%%%%
%  _____   ____  _____                                          %
% |_   _| /  __||  __ \    Institute of Computitional Physics   %
%   | |  |  /   | |__) |   Zuercher Hochschule Winterthur       %
%   | |  | (    |  ___/    (University of Applied Sciences)     %
%  _| |_ |  \__ | |        8401 Winterthur, Switzerland         %
% |_____| \____||_|                                             %
%%%%%%%%%%%%%%%%%%%%%%%%%%%%%%%%%%%%%%%%%%%%%%%%%%%%%%%%%%%%%%%%%
%
% Project     : BA Welti Keller
% Title       : 
% File        : grundlagen.tex Rev. 00
% Date        : 15.09.2014
% Author      : Tobias Welti
%
%%%%%%%%%%%%%%%%%%%%%%%%%%%%%%%%%%%%%%%%%%%%%%%%%%%%%%%%%%%%%%%%%

\chapter{Grundlagen}\label{chap.grundlagen}
etwas über mems
\todo{etwas über signalerfassung. }
\todo{etwas über signalverarbeitung (aufwand hilbert etc.)}
% !TeX spellcheck = de_CH
%%%%%%%%%%%%%%%%%%%%%%%%%%%%%%%%%%%%%%%%%%%%%%%%%%%%%%%%%%%%%%%%%
%  _____   ____  _____                                          %
% |_   _| /  __||  __ \    Institute of Computitional Physics   %
%   | |  |  /   | |__) |   Zuercher Hochschule Winterthur       %
%   | |  | (    |  ___/    (University of Applied Sciences)     %
%  _| |_ |  \__ | |        8401 Winterthur, Switzerland         %
% |_____| \____||_|                                             %
%%%%%%%%%%%%%%%%%%%%%%%%%%%%%%%%%%%%%%%%%%%%%%%%%%%%%%%%%%%%%%%%%
%
% Project     : BA Welti Keller
% Title       : 
% File        : hardware.tex Rev. 00
% Date        : 15.09.2014
% Author      : Tobias Welti
%
%%%%%%%%%%%%%%%%%%%%%%%%%%%%%%%%%%%%%%%%%%%%%%%%%%%%%%%%%%%%%%%%%

\chapter{Hardware}\label{chap.hardware}


\section{Hardware-Architektur}\label{sec.hw_arch}

Anhand der funktionalen Vorgaben für die Messstation werden der \gls{logger}, die \gls{sensoreinh} und das Bussytem im folgenden genauer spezifiziert und die Komponenten ausgewählt.

\subsection{Datenlogger}
Das \gls{hardware}-Konzept des \gls{logger}s ist in Abbildung \ref{fig.hwkonzept_logger} dargestellt.
Der \gls{logger} sammelt die Daten der \glspl{sensoreinh} über das \gls{bussys} ein und speichert sie ab. Dafür benötigt er das \gls{bussys}, einen \gls{mc}, einen internen Speicher und ein leicht auswechselbares \gls{speichermedium}. Ausserdem soll über eine Schnittstelle ein Computer angeschlossen werden können, um den Betrieb der Messstation zu steuern. Der \gls{logger} wird in einem wasserdichten Gehäuse untergebracht. Für den Austausch des \gls{speichermedium}s wäre eine verschraubbare Öffnung denkbar.

\begin{figure}
	\centering
		\includegraphics[width=0.8\textwidth]{images/visio/hardwarekonzept_logger.pdf}
	\caption{Hardwarekonzept des \gls{logger}s.}
	\label{fig.hwkonzept_logger}
\end{figure}

\subsection{Sensoreinheit}
Die \gls{sensoreinh} benötigt einen Beschleunigungssensor, um die Einschläge von Geschiebe zu messen. Über einen Analog-Digital-Wandler (\gls{adwandler}, Englisch \gls{adc}) werden die Messsignale digitalisiert. Die gemessenen Signale werden von einem \gls{mc} verarbeitet, im internen Speicher zwischengespeichert und über das \gls{bussys} an den \gls{logger} übertragen. Abbildung \ref{fig.hwkonzept_sensor} zeigt das \gls{hardware}-Konzept der \gls{sensoreinh}.

\begin{figure}
	\centering
		\includegraphics[width=0.8\textwidth]{images/visio/hardwarekonzept_sensor.pdf}
	\caption{Hardwarekonzept der \gls{sensoreinh}.}
	\label{fig.hwkonzept_sensor}
\end{figure}

\subsection{Bussystem}
Das \gls{bussys} muss die Daten und Befehle zwischen \gls{logger} und \glspl{sensoreinh} übertragen. Die Reichweite des \gls{bussys}s muss genügen, um alle Komponenten der Messinstallation zu verbinden. Die Datenbandbreite muss die Übertragung der Messresultate aller \glspl{sensor} erlauben.


\section{Komponentenauswahl}

\subsection{Mikroprozessor}
Bei der Auswahl des \gls{mc}s werden folgende Kriterien berücksichtigt:

\begin{itemize}
\item Genügend Rechenleistung für allfällige zusätzliche Anforderungen.
\item \gls{adwandler} mit genügender Abtastrate und Auflösung (\gls{bitbreite}).
\item \gls{dspgloss} integriert für die schnelle Verarbeitung der Messdaten.
\item \gls{nvic}.
\item Ein-/Ausgänge für das \gls{bussys}.
\item Ein-/Ausgänge für den externen Speicher.
\item möglichst geringer Stromverbrauch.
\end{itemize}

Für die vorliegende Anwendung eignen sich Mobile-Prozessoren sehr gut. Sie sind für den Einsatz in mobilen Geräten konzipiert, d.h. für den Batteriebetrieb, sind aber trotzdem sehr leistungsfähig. Die \emph{ARM Cortex}-Reihe bietet ein breites Spektrum an Prozessoren an. Im Mobile-Segment der \emph{ARM Cortex}-Reihe sind vier Prozessoren erhältlich. Da nur in einem Modell ein \gls{dsp} integriert ist, fällt die Entscheidung leicht. Der \emph{ARM Cortex-M4} ist auch der neueste Prozessor aus dem Mobile-Segment. Mit dem Ausblick, das Messsystem in einem zukünftigen Projekt zur Serienreife zu bringen, macht es nur Sinn, den neuesten Prozessor zu verwenden. Ein Auszug aus dem Datenblatt des \emph{ARM Cortex-M4} befindet sich im Anhang \ref{ds.lpc4088}. Abbildung \ref{fig.M4diagramm} zeigt die Fähigkeiten des \emph{ARM Cortex-M4}.

\paragraph{\gls{nvicgloss}} Ein Prozessor mit NVIC kann auf verschiedene Ereignisse reagieren, indem Interrupts ausgelöst werden. Jedem Interrupt kann eine Priorität zugewiesen werden, um festzulegen, ob ein Interrupt einen anderen unterbrechen darf, der gerade vom Prozessor abgearbeitet wird. Für das Messsystem wird ein \gls{nvic} benötigt, da mehrere zeitkritische Prozesse parallel ablaufen sollen. Einerseits muss die Abtastrate der Messwerterfassung genau eingehalten werden. Andererseits darf die Verarbeitung der Messwerte nicht zu lange unterbrochen werden, um einen Überlauf der Queue zu vermeiden. Die Daten der \glspl{ereignis} müssen parallel dazu an den \gls{logger} übermittelt werden. Die Prioritäten dieser Prozesse müssen richtig gewählt werden. Im Abschnitt \ref{subsec.sw_ueberblick} \emph{ff.} wird näher darauf eingegangen.

\paragraph{\gls{dsp}} Der \emph{ARM Cortex-M4} verfügt über \gls{dsp}-Funktionen, eine sog. single-cycle \gls{mac}. In einer single-cycle \gls{macgloss} können Multiplikationen in einem einzigen Prozessorzyklus ausgeführt werden. Normalerweise benötigt ein Prozessor für eine Multiplikation bis zu mehreren zehn Zyklen, bis das Resultat vorliegt. Mit einem \gls{dspgloss} ist es daher möglich, z.B. eine Filterfunktion viel effizienter zu berechnen als mit einem normalen Prozessor, da für diese viele Multiplikationen und Additionen ausgeführt werden müssen.

\paragraph{\gls{fpu}} Eine \gls{fpugloss} führt Berechnungen mit Dezimalbrüchen sehr rasch aus. Für unsere Anwendung ist eine \gls{fpu} keine Voraussetzung. Falls in einem späteren Projekt komplexere Filter oder andere Anwendungen berechnet werden müssen, könnte die \gls{fpu} aber ein Vorteil sein.

\paragraph{Wake Up Interrupt Controller Interface} Ein Wake Up Interrupt Controller Interface ermöglicht es, einen Prozessor in einen Stromsparmodus zu versetzen und ihn durch ein definiertes Signal wieder zu wecken. Damit ist es möglich, eine \gls{sensoreinh} praktisch ganz abzuschalten, wenn sie keine Messungen durchführt. Durch eine Nachricht über das Bussystem kann die \gls{sensoreinh} wieder eingeschaltet werden. Der Cortex-M4 verfügt über 240 mögliche Wake Up Interrupts, kann also für 240 Ereignisse programmiert werden, die ihn aufwecken oder in einen Stromsparmodus senden können (vgl. \cite{armcortex}).

\begin{figure}
	\centering
		\includegraphics[width=0.8\textwidth]{images/datasheets/Cortex-M4-chip-diagram-LG.png}
	\caption{Chipdiagramm der \emph{ARM Cortex-M4} Architektur \cite{armcortex}.}
	\label{fig.M4diagramm}
\end{figure}

\paragraph{Rechenleistung} Die Rechenleistung des \emph{ARM Cortex-M4} hängt von der Implementation ab. Die Firma ARM produziert den \emph{ARM Cortex-M4} nicht selbst, sondern lizenziert Chip-Hersteller für die Verwendung der Architektur in ihren Prozessoren. Vom \emph{ARM Coretx-M4} sind mehrere Ausführungen erhältlich. Verschiedene Chip-Hersteller implementieren eine Architektur des \emph{ARM Cortex-M4} in ihren Prozessoren.

\paragraph{\gls{adwandler}} Bei den bestehenden Messstationen wird die Datenmenge stark reduziert, indem über eine Minute ein Histogramm mit den Peakintensitäten als Klassen berechnet wird. Die Intensitäten werden in 18 Klassen logarithmischer Abstufung eingeteilt. Das entspricht einer \gls{bitbreite} von etwas mehr als vier Bit (4 Bit ermöglichen 16 Werte). Da für die Klasseneinteilung eine logarithmische Abstufung gewählt wurde, muss ein linearer \gls{adwandler} trotzdem eine höhere \gls{bitbreite} als vier aufweisen, um die gleiche Auflösung wie in den unteren logarithmischen Klassen zu erreichen. Mit 12 Bit Auflösung sind 4096 Stufen unterscheidbar, was für diese Anwendung genügt.

\paragraph{Peripherie} Damit der Prozessor über das Bussystem kommunizieren und auf ein externes Speichermedium schreiben kann, sind genügend Ein- und Ausgabe-\glspl{pin} nötig.

\paragraph{Wahl eines Prozessors} Da der Entscheid für eine Hardware schon zu Beginn des Projekts gefällt werden musste, wurde auf eine grosszügige Sicherheitsmarge in Sachen Rechenleistung und \gls{bitbreite} geachtet.  Um Kosten und Baugrösse der \gls{sensoreinh} klein zu halten, suchten wir nach einem Evaluationsboard mit \emph{ARM Cortex-M4} Prozessor. Das \emph{LPC4088 QuickStart Board} von \emph{NXP Semiconductors} hat genügend Arbeitsspeicher für den Prozessor, verfügt über die benötigten \glspl{pin} für die Peripherie und hat bei weitem genügend Rechenleistung. Die Fähigkeiten des NXP LPC4088FET208 Prozessors sind in Tabelle \ref{table.lpc4088} dargestellt. 

\begin{table}
\begin{center}
\begin{tabular}{|l|l|}
\hline
\begin{minipage}{38mm}Taktfrequenz\end{minipage} & \begin{minipage}{44mm}bis 120 MHz\end{minipage} \\
\hline
NVIC & vorhanden \\
\hline
FPU & vorhanden \\
\hline
Programmspeicher & 512 kByte \\
\hline
Arbeitsspeicher (intern) & 96 kByte \\
\hline
CAN-Bus & 2 \\
\hline
USB & 2 \\
\hline
SD-Card & Anschlüsse vorhanden \\
\hline
\gls{adwandler} & 8 Eingänge, 12 Bit \\
\hline
\end{tabular}
\caption{Fähigkeiten des NXP LPC4088 Prozessors \cite{nxplpc4088}.}
\label{table.lpc4088}
\end{center}
\end{table}

Auf dem \emph{NXP LPC4088 QuickStart Board} sind zusätzliche Bauteile verbaut, z.B. Speicher (\gls{sdram} und \gls{flash}) und \gls{pin}-Steckleisten, um CAN-Bus-\glspl{pin} oder A/D-Eingänge anzuschliessen. Tabelle \ref{table.nxplpc4088qsb} listet die für dieses Projekt relevanten, zusätzlichen Eigenschaften auf.

\begin{table}
\begin{center}
\begin{tabular}{|l|l|}
\hline
\begin{minipage}{38mm}Prozessor\end{minipage} & \begin{minipage}{44mm}NXP LPC4088FET208\end{minipage}\\
\hline
Taktfrequenz & bis 120 MHz \\
\hline
Flash-Speicher & 8 MByte \\
\hline
SDRAM & 32 MByte \\
\hline
\gls{adwandler} & 6 Eingänge nutzbar, 12 Bit \\
\hline
\end{tabular}
\caption{Zusätzliche Fähigkeiten des NXP LPC4088 QuickStart Boards von \emph{Embedded Artists}  \cite{nxplpc4088qsb}.}
\label{table.nxplpc4088qsb}
\end{center}
\end{table}

\subsection{Bus-System}
Das \gls{bussys} für die Messanlage muss die folgenden Kriterien erfüllen:
\begin{itemize}
\item Übertragungsbandbreite genügend für fortlaufende Übertragung von Rohdaten einer \gls{sensoreinh}.
\item Reichweite mindestens 20 Meter.
\item Robust gegenüber äusseren Einflüssen.
\item Mindestens zwanzig Busteilnehmer möglich.
\end{itemize}

\begin{table}
\begin{tabular}{|l|l|l|l|l|}
\hline  & \textbf{Bitrate}      & \textbf{Distanz} & \textbf{Clients} & \textbf{Besonderheiten}\\ 
\hline \textbf{CAN} & \begin{minipage}{2cm}
1 MBit/s\\ 125 kBit/s
\end{minipage} & \begin{minipage}{1.5cm}40 m\\500 m\end{minipage} & > 20 & \begin{minipage}{6cm}
\mbox{ }\\+ \gls{cdet} umgehen mit \gls{polling} durch Master.\\
+ Bei synchronem CAN wird \gls{cdet} durch ID gelöst.\\
+ CAN Controller sendet Interrupt Request bei erhaltener Nachricht.\\
+ CAN Filter blendet irrelevante Nachrichten aus.
\end{minipage} \\ 
\hline \textbf{SPI} & ..100 MBit/s & < 1 m & \begin{minipage}{1cm}
slave select
\end{minipage} & \begin{minipage}{6cm}
\mbox{ }\\- Pro Client eine Slave Select Leitung\\
- alternativ: \gls{daisy} $\Rightarrow $alle \gls{mcacr} beschäftigt.\\
- Bei Ausfall eines \gls{mcacr} ganzer Bus unterbrochen.\\
\end{minipage} \\ 
\hline \textbf{RS485} & \begin{minipage}{2cm}
35 MBit/s\\100 kBit/s
\end{minipage} & \begin{minipage}{1.5cm}
10 m\\1200 m
\end{minipage} & >32 & \begin{minipage}{6cm}
\mbox{ }\\- Master am besten in der Mitte des Bus $\Rightarrow$ ungünstig.\\
- Braucht 2--4 Drähte (bei Full Duplex)\\
- braucht pull-up und pull-down Widerstände $\Rightarrow$ mehr Leistungsaufnahme.\\
\end{minipage} \\ 
\hline \textbf{Ethernet} & 100 MBit/s & 100 m & > 20 & \begin{minipage}{6cm}
\mbox{ }\\+ Stromversorgung bei Power over Ethernet (PoE) integriert.\\
- kein Bus sondern allenfalls \gls{daisy}.\\
- bei \gls{daisy} kein PoE möglich.\\
\end{minipage} \\ 
\hline \textbf{Feldbus} &  &  &  & \begin{minipage}{6cm}
\mbox{ }\\ist eine Familie von Bussen, z.B. CAN-Bus\\
\end{minipage} \\ 
\hline \textbf{I2C} & 0.4..5 Mbit/s & wenige Meter & < 20 & \begin{minipage}{6cm}
\mbox{ }\\nur für kurze Distanzen, Bitrate nimmt mit zunehmender Distanz rasch ab.\\
\end{minipage}\\
\hline 
\end{tabular}
\caption{Entscheidungsmatrix für die Auswahl des \gls{bussys}s.}
\label{table.bussystem}
\end{table} 

In Tabelle \ref{table.bussystem} sind die Eigenschaften diverser \glspl{bussys} aufgeführt.

\paragraph{Kommentare}
SPI und I2C sind nur für kurze Distanzen geeignet und sind deshalb keine Option.
Die Verwendung von Ethernet zur Datenübertragung würde zwei Schnittstellen auf jeder \gls{sensoreinh} voraussetzen, um die \glspl{sensor} hintereinander zusammenzuhängen (\gls{daisy}). Jedes Paket müsste vom \gls{mc} weitergeleitet werden, wenn es für einen anderen Empfänger bestimmt ist. Dies führte zu einer zusätzlichen Belastung der Microcontroller. Stromversorgung über Ethernet ist mit PowerOverEthernet (PoE) zwar möglich, erfordert aber spezielle Geräte zur Speisung über den Stecker des Datenkabels. Dies verunmöglicht eine \gls{daisy} mit PoE, neben dem Datenkabel wäre noch ein Kabel für die Stromversorgung notwendig.

\paragraph{Vergleich CAN-Bus und RS485} Die beiden Bussysteme CAN und RS485 sind nicht einfach zu vergleichen, da der RS485-Standard nur die elektrischen Eigenschaften des Systems beschreibt (OSI Layer 1). Der CAN-Standard beschreibt auch den Data Link Layer (OSI Layer 2). Der Data Link Layer beschreibt Methoden, die die Übertragung zuverlässig machen. Ein Beispiel dafür ist eine Prüfsumme, die es ermöglicht, eine fehlerhafte Übertragung im Empfänger festzustellen. Der Empfänger bestätigt bei korrekter Prüfsumme den Empfang. Stellt einer der Empfänger einen Fehler fest, sendet er eine Fehlermeldung über den Bus und stört damit die Übertragung. So ist es nicht möglich, dass einige Empfänger die Nachricht lesen konnten und andere nicht. Der Sender ist bei erfolgter Bestätigung sicher, dass seine Nachricht erfolgreich an alle Empfänger übertragen wurde (vgl. \cite{ixxatcan}). 

CAN-Bus definiert auch eine Adressierung, Kollisions-Erkennung und -Auflösung, Prioritäten der Busteilnehmer und ein Nachrichtenformat. Bei RS485 besteht eine Nachricht aus einem einzelnen Zeichen. Der Data Link Layer muss gänzlich in Software gelöst werden. Dafür ist es möglich, das Protokoll komplett selbst zu definieren. Dies erlaubt beliebig lange Übertragungen von einem Busteilnehmer. CAN-Bus limitiert die Nachrichtenlänge auf 8 Bytes. Die Übertragung längerer Nachrichten muss über Software geregelt werden (vgl. \cite{ixxatcan}).

Bei beiden Standards ist der Stecker nicht definiert. Das lässt die komplette Freiheit für die Wahl eines wasserdichten Steckverbinders.

Da der CAN-Bus bereits mit dem Standard viele benötigte Merkmale mitbringt, fällt die Entscheidung nicht schwer.

\paragraph{Entscheidung}
CAN-Bus erfüllt alle Kriterien und erlaubt es, den Busmaster am Ende des Bus zu platzieren. Dies ist ein weiterer Vorteil gegenüber RS485, wo der Master in der Mitte platziert werden sollte. CAN-Bus bietet bereits \gls{cdetgloss} und Fehlererkennung, während dies bei RS485 in der Software gelöst werden muss. Für CAN-Bus sind Bus-Treiber (\gls{transceiver}) erhältlich, die mit hohen Spannungen umgehen können, was das \gls{bussys} robuster gegenüber Umwelteinflüssen macht. Die Grösse der Datenpakete ist bei CAN-Bus auf 8 Byte begrenzt, bei RS485 werden die Datenpakete über die Software frei definiert, was einen Vorteil von RS485 darstellt. Insgesamt überwiegen die Vorteile von CAN-Bus klar. 

\paragraph{CAN-Bus} CAN-Bus ist ein Bussystem über eine differentielle Leitung. Auf zwei Drähten liegt eine bestimmte Spannung an. Der Empfänger misst die Spannungsdifferenz zwischen den beiden Drähten. Die Differenz wird vom Sender entweder klein oder gross gehalten, um ein 'dominantes' oder ein 'rezessives' Bit zu senden. 'Rezessiv' bedeutet dabei, dass ein anderer Busteilnehmer durch Anlegen eines 'dominanten' Bits das 'rezessive' Bit überschreiben kann. Welche Spannungsdifferenzen 'dominant' und 'rezessiv' darstellen, ist vom Standard nicht definiert und lässt dem Entwickler damit freie Hand, die elektrischen Eigenschaften seiner CAN-Implementation seinen Bedürfnissen entsprechend zu definieren. Um Signalreflexionen am Ende der Leitungen zu vermeiden, wird ein Terminierungswiderstand an beiden Enden benötigt, der die beiden Leitungen abschliesst. (vgl. \cite{boschcanspec2}).

Die beiden Drähte der differentiellen Leitung sind verdrillt. Dadurch wird das Signal der Leitung besser vor äusseren Einflüssen geschützt. Elektrische und magnetische Felder induzieren in einem Draht einen Strom. Da das äussere Feld aber in beiden Drähten praktisch den gleichen Strom induziert, ändert sich an der Spannungsdifferenz in den Drähten kaum etwas (siehe Abbildung \ref{fig.diff}). Die beiden Drähte stellen auch eine Spule dar. Durch das Verdrillen der Drähte ändert sich die Ausrichtung der Spule im äusseren Feld auf kurzen Abständen, induzierte Störungen heben sich so gegenseitig auf (vgl. \cite[Kap. 2, S. 14]{muellerkt1}).

\begin{figure}
	\centering
		\includegraphics[width=0.8\textwidth]{images/differential.pdf}
	\caption{Differentielle Leitung \cite{{muellerkt1}}.}
	\label{fig.diff}
\end{figure}

Die Kollisions-Erkennung und -Auflösung wird über ID-Nummern der Teilnehmer gelöst. Vor dem Start einer Übertragung prüft der Teilnehmer, dass der Bus zur Zeit frei ist. Dann sendet der Teilnehmer seine ID-Nummer. Falls zwei Teilnehmer gleichzeitig zu senden beginnen, werden sie irgendwann unterschiedliche Bits senden. Der Teilnehmer, der dann das 'dominante' Bit sendet, liest vom Bus den gleichen Wert, den er gerade sendet. Für ihn ist die Übertragung nicht gestört. Der Teilnehmer mit dem 'rezessiven' Bit liest aber das 'dominante' Bit des anderen Senders und muss seine Übertragung sofort abbrechen. Diese Art der Kollisionsauflösung hat den Vorteil, dass keine Übertragungszeit verloren geht, da einer der Teilnehmer seine Nachricht senden darf. Der Verlierer wartet, bis der Bus wieder frei ist und probiert dann erneut, die Nachricht zu senden (vgl. \cite{boschcanspec2}).

Die korrekte Übertragung wird mittels eines \gls{crc}-Codes überprüft. Der Empfänger berechnet bereits während dem Empfang der Nachricht die \gls{crcgloss}-Prüfsumme. Sobald die vom Sender mitgeschickte CRC-Prüfsumme übertragen ist, kann der Empfänger diese mit der berechneten \gls{crc} vergleichen. Stimmen die Prüfsummen überein, bestätigt der Empfänger die korrekte Übertragung (vgl. \cite{boschcanspec2}).

\subsection{Speichermedium}
\paragraph{Kriterien} Das externe Speichermedium soll möglichst klein sein, wenig Stromverbrauch haben und einfach auswechselbar sein. Bei Inaktivität sollte das Medium wenn möglich keinen Strom verbrauchen. Für einen mehrwöchigen unabhängigen Betrieb einer Messstation muss genügend Speicherkapazität bereitgestellt werden.

\paragraph{Datenmenge} Pro \gls{sensor} werden bei hohem Geschiebeaufkommen maximal hundert \glspl{ereignis} pro Sekunde erwartet. Ein solches Geschiebeaufkommen stellt jedoch die Ausnahme dar. Ein \gls{ereignis} benötigt je nach verlangtem Detailgrad und Dauer des \gls{ereignis}ses 10--200 Byte Speicherplatz. Für den normalen Betriebsmodus werden 50 Byte/\gls{ereignis} gerechnet, bei 5 \glspl{ereignis}n pro Sekunde. Damit ergibt sich eine Datenrate von 250 Byte/s, die es pro \gls{sensor} abzuspeichern gilt. Mit zehn \glspl{sensor} im Einsatz müssen 2.5 kByte/s gespeichert werden. 

\paragraph{Unabhängige Betriebsdauer} Pro Gigabyte Speicherplatz können 111 Stunden Daten für zehn \glspl{sensor} gespeichert werden. Bei hohem Geschiebeaufkommen mit zwanzig Mal mehr \glspl{ereignis}n bleiben immer noch 5 Stunden Aufzeichnungszeit pro Gigabyte. Begnügt man sich mit weniger Details, fallen pro \gls{sensor} in zehn Sekunden rund 400 Byte Daten an. Bei dieser Datenrate reicht ein Gigabyte für rund 700 Stunden. Auch bei hohem Geschiebeaufkommen kann die Anlage über mehrere Tage Daten speichern. 

\paragraph{Kapazität} Heute sind Speichermedien mit Kapazitäten bis über 128 GB erhältlich, so dass die Detailrate kein entscheidendes Kriterium mehr darstellt.

\paragraph{Datentransfer} Für den Transfer der Daten aus dem \gls{logger} auf einen Computer gibt es grundsätzlich zwei Varianten. Entweder man liest die Daten über eine Schnittstelle auf den Computer aus, oder man tauscht das Speichermedium aus. Das Auslesen via Schnittstelle benötigt zusätzlich Strom, das Wechseln des Speichermediums setzt einen mehr oder weniger komfortablen und trotzdem wasserdichten Zugang zum Medium voraus. Da heute Speichermedien mit kleinem Platzbedarf erhältlich sind, kann ein solcher Zugang recht einfach mit einem Schraubverschluss realisiert werden.

\paragraph{Vergleich} In Tabelle \ref{table.speichermedium} werden verschiedene Speichermedien miteinander verglichen. In der Spalte 'Breite' ist aufgelistet, wie gross eine Öffnung mindestens sein muss, um das Speichermedium wechseln zu können. 'Pins' gibt an, wie viele Leitungen für den Anschluss des Mediums am Microcontroller nötig sind. Der Stromverbrauch in Klammern gilt für den Standby-Modus des Speichermediums.

\begin{table}
\begin{tabular}{|l|l|l|l|l|}
	\hline
	                      & \textbf{Breite} & \textbf{Pins} & \textbf{Stromverbrauch} & \textbf{Bemerkungen}        \\ \hline
	\textbf{SD-Card}      & 24 mm           & 9             & 20..100 mA (0.2 mA)     & 4 bit breiter serieller Bus \\ \hline
	\textbf{CompactFlash} & 43 mm           & 50            & max. 70 mA (k.A.)       & paralleler Bus              \\ \hline
	\textbf{USB-Stick}    & min. 12 mm      & 4             & typ. 70 mA (k.A.) &  \\ \hline
\end{tabular} 
\caption{Entscheidungsmatrix zur Auswahl des Speichermediums \cite{sdstd,cfstd,usbwiki}.}
\label{table.speichermedium}
\end{table} 

\paragraph{Entscheid} Für einen verschraubbaren Verschluss ist die CompactFlash-Karte zu breit, das Gehäuse würde dadurch sehr gross. Die SD-Karte und der USB-Stick sind vergleichbar in der Grösse. Von der SD-Karte sind auch kleinere Varianten erhältlich. Eine Öffnung für den Austausch des Speichermediums kann eine gewisse Grösse ohnehin nicht unterschreiten, damit hineingegriffen werden kann. Da die SD-Karte im Standby den geringeren Stromverbrauch hat, wird der \gls{logger} mit einem SD-Kartenleser ausgestattet.

\subsection{Sensor}
Da für den AD-Wandler des verwendeten Boards ein maximaler Messbereich zwischen 0 und 3.3 Volt empfohlen wird, musste ein Beschleunigungssensor ausgewählt werden, der in diesem Bereich arbeitet und eine möglichst genaue Auflösung bietet. Zusätzlich sollte der gewählte Sensor auf einem Evaluationsboard bestellt werden können, um den Implementationsaufwand gering zu halten. Der ADXL001-70 von Analog Devices lässt sich im geforderten Bereich betreiben und liefert eine Auflösung von 16mV/g bei einem Messbereich von -70g bis + 70g. Für den finalen Einsatz muss ein Sensor mit einem weit grösseren Bereich gewählt werden, bereits bei den Laborversuchen konnte festgestellt werden, dass 70g sehr schnell erreicht werden.

\subsection{Schnittstelle}
Das gewählte \emph{NXP LPC4088 QuickStart Board} verfügt über einen \gls{usb}-Anschluss, über den eine serielle Schnittstelle angesprochen werden kann. Für eine einfache Kommandozeile oder ein Konfigurationsmenü ist diese Schnittstelle ausreichend. Die Schnittstelle kann mit einer Übertragunsrate von 9600 bis 115200 Baud betrieben werden. 


\subsection{Gehäuse}
Um den \gls{logger} und die \glspl{sensoreinh} wasserdicht zu verpacken, wurden Gehäuse und Komponenten mit der Schutzklasse IP68 gesucht. Die wasserdichten Gehäuse der Firma \emph{FIBOX} sind in verschiedenen Ausführungen erhältlich. Da diese Arbeit kein serienreifes Produkt zum Ziel hatte, wurden Gehäuse aus ABS-Kunststoff mit transparenten Deckeln gewählt. Die Gehäuse messen 180x130x75~mm.

Die Stecker wurden ebenfalls mit Schutzklasse IP68 gewählt. Als zusätzliches Kriterium galt der Durchmesser des Kabelmantels. Die Stecker \emph{Lumberg 0332 05-1} verfügen über fünf Pole, die vierpolige Variante war zur Zeit nicht lieferbar.

Für den \gls{usb}-Anschluss gibt es wasserdichte Ausführungen, an denen sowohl innen als auch aussen Standard-USB-Kabel angeschlossen werden können. Die \emph{Buccaneer PCXP6043/B}-Gerätebuchse von \emph{Bulgin} erfüllt die Anforderungen. Für diese Buchse ist auch eine wasserdichte Kappe erhältlich.

Die verschraubbare Öffnung für die SD-Karte besteht aus einer M36/M32-Gewindebuchse, in die ein M32-Schraubdeckel eingeschraubt wird.


\section{Datenlogger}
Die Hardware-Architektur des \gls{logger}s ist in Abbildung \ref{fig.hw_logger} schematisch dargestellt. Kernstück ist der \gls{mc} \emph{NXP LPC4088}. Der \gls{mc} ist über eine serielle Schnittstelle mit dem CAN-Transceiver verbunden. Der Transceiver sendet Konfigurations- und Kontrolldaten und empfängt Messdaten über den CAN-Bus.

Dem \gls{mc} verfügt über einen interner Programm- und Datenspeicher, dieser ist aber sehr beschränkt. Damit Konfigurations- und Messdaten zwischengespeichert werden können, steht dem \gls{mc} auf dem \emph{NXP LPC4088 QuickStartBoard} (gestrichelter Kasten) externer Speicher zur Verfügung: 8~MByte \gls{flash} und 32~MByte \gls{sdram}. Dieser Speicher genügt, um Daten von mehreren \glspl{sensoreinh} zwischenzuspeichern.

Über das \gls{mci} ist eine SD-Karte an den \gls{mc} angebunden. Auf der SD-Karte werden sowohl die Konfigurationsdaten als auch die Messdaten gespeichert. Heute sind SD-Karten mit bis zu 256~GByte Speicherplatz erhältlich (SDXC-Karten). Zur Zeit unterstützt die verwendete Bibliothek zur Verwendung der SD-Karten nur SDHC-Karten mit bis zu 32~GByte Speicherplatz. 

Für die Kommunikation mit einem \gls{compi} verfügt der \gls{logger} über einen \gls{usb}-Anschluss. Über einen \gls{terminalemu} wie \emph{PuTTY} kann via serielle Schnittstelle auf ein Konfigurationsmenü zugegriffen werden, um die Messanlage zu überwachen und zu steuern.

\begin{figure}
	\centering
		\includegraphics[width=0.8\textwidth]{images/visio/hardware_logger.pdf}
	\caption{Schematischer Hardware-Aufbau des \gls{logger}s.}
	\label{fig.hw_logger}
\end{figure}



\section{Sensoreinheit}
Die \gls{sensoreinh} ist ähnlich aufgebaut wie der \gls{logger}. Das Schema in Abbildung \ref{fig.hw_sensor} zeigt die Hardware. Wie im \gls{logger} ist der \emph{NXP LPC4088} \gls{mc} mit zusätzlichem \gls{sdram} und \gls{flash} ausgerüstet. Der Anschluss an den CAN-Bus ist identisch. Die SD-Karte wird in der \gls{sensoreinh} nicht benötigt und wurde deshalb weggelassen. Auch eine USB-Schnittstelle für die Konfiguration ist nicht nötig, da die Konfiguration über den CAN-Bus erfolgt.

Der Beschleunigungs-\gls{sensor} wird vom \emph{QuickStart Board} mit Spannung versorgt und gibt die gemessene Beschleunigung als analoge Spannung aus. Diese Spannung wird vom \gls{adwandler} des \emph{NXP LPC4088} \gls{mc} gemessen.

\begin{figure}
	\centering
		\includegraphics[width=0.8\textwidth]{images/visio/hardware_sensor.pdf}
	\caption{Schematischer Hardware-Aufbau der \gls{sensoreinh}.}
	\label{fig.hw_sensor}
\end{figure}


%%%%%%%%%%%%%%%%%%%%%%%%%%%%%%%%%%%%%%%%%%%%%%%%%%%%%%%%%%%%%%%%%
%  _____   ____  _____                                          %
% |_   _| /  __||  __ \    Institute of Computitional Physics   %
%   | |  |  /   | |__) |   Zuercher Hochschule Winterthur       %
%   | |  | (    |  ___/    (University of Applied Sciences)     %
%  _| |_ |  \__ | |        8401 Winterthur, Switzerland         %
% |_____| \____||_|                                             %
%%%%%%%%%%%%%%%%%%%%%%%%%%%%%%%%%%%%%%%%%%%%%%%%%%%%%%%%%%%%%%%%%
%
% Project     : LaTeX doc Vorlage für Windows ProTeXt mit TexMakerX
% Title       : 
% File        : einleitung.tex Rev. 00
% Date        : 23.4.12
% Author      : Remo Ritzmann
% Feedback bitte an Email: remo.ritzmann@pfunzle.ch
%
%%%%%%%%%%%%%%%%%%%%%%%%%%%%%%%%%%%%%%%%%%%%%%%%%%%%%%%%%%%%%%%%%

\chapter{Software-Konzept}\label{chap.software}


\section{Software-Stack}\label{sec.sw_stack}


\subsection{Überblick}\label{subsec.sw_ueberblick}


\subsection{Messdatenerfassung}\label{subsec.sw_messen}


\subsection{Ereigniserkennung}\label{subsec.sw_ereignis}


\subsection{Timestamp}\label{subsec.sw_timestamp}


\subsection{Busprotokoll}\label{subsec.sw_busprotokoll}


\subsection{Filesystem}\label{subsec.sw_filesystem}


\subsection{UART-Kommandozeile}\label{subsec.sw_uart}


\section{Funktionalität}\label{sec.sw_funktionalitaet}


\section{Konfiguration}\label{sec.sw_konfiguration}

% !TeX spellcheck = de_CH
%%%%%%%%%%%%%%%%%%%%%%%%%%%%%%%%%%%%%%%%%%%%%%%%%%%%%%%%%%%%%%%%%
%  _____   ____  _____                                          %
% |_   _| /  __||  __ \    Institute of Computitional Physics   %
%   | |  |  /   | |__) |   Zuercher Hochschule Winterthur       %
%   | |  | (    |  ___/    (University of Applied Sciences)     %
%  _| |_ |  \__ | |        8401 Winterthur, Switzerland         %
% |_____| \____||_|                                             %
%%%%%%%%%%%%%%%%%%%%%%%%%%%%%%%%%%%%%%%%%%%%%%%%%%%%%%%%%%%%%%%%%
%
% Project     : BA Welti Keller
% Title       : 
% File        : resultate.tex Rev. 00
% Date        : 15.09.2014
% Author      : Tobias Welti
%
%%%%%%%%%%%%%%%%%%%%%%%%%%%%%%%%%%%%%%%%%%%%%%%%%%%%%%%%%%%%%%%%%

\chapter{Resultate}\label{chap.resultate}

\section{Testfälle}
Die wichtigsten Testfälle für die grundsätzliche Funktionalität der Messstation konnten erfolgreich abgeschlossen werden. Für einige Tests blieb jedoch nicht genügend Zeit vor der Fertigstellung des Berichts. Diese Tests werden so weit möglich noch nachgeholt.

Im Kapitel \ref{chap.tests} ab Seite \pageref{chap.tests} sind die Testfälle und die Testergebnisse aufgeführt.

\section{Ereigniserkennung}
Der \gls{adwandler} des \emph{NXP LPC4088} \gls{mc} wurde wie geplant in Betrieb genommen und liefert Messdaten, die sich sehr gut mit dem analogen Ausgangssignal des \gls{sensor}s decken. Abbildung \ref{fig.comparison2} zeigt den Vergleich zwischen der Messung von Oszilloskop (blau) und der \gls{sensoreinh} (grün). Die \gls{sensoreinh} arbeitete mit einer \gls{fs} von 10000~\ensuremath{Hz}, das Oszilloskop zeichnete mit einer \gls{fs} von 3.125~MHz auf. Um ein Ereignis zu simulieren, wurde ein Golfball auf den Testaufbau (siehe im Verzeichnis Fotos/Testaufbau/ auf der beiliegenden CD) fallen gelassen. Der Vergleich zeigt, dass die beiden Kurven gut übereinstimmen. 

\begin{figure}
	\centering
		\includegraphics[width=0.8\textwidth]{images/comparison/comparison.png}
	\caption{Vergleichsmessung mit Oszilloskop und \gls{sensoreinh}.}
	\label{fig.comparison2}
\end{figure}


Der Betrieb mit 10000~Hz war für unseren Testaufbau genügend, die Peaks traten mit einer Frequenz von ungefähr 2000~Hz auf. Die Frequenz der Peakspitzen variiert sowohl mit der Plattenkonstruktion als auch mit der Korngrösse, Kornbeschaffenheit und der Art des Aufschlags. Daher muss für den tatsächlichen Messbetrieb eine Kalibrierung gemacht werden, um die geeigneten Parameter zu finden.

Für Versuche mit verschiedenen \gls{fs}n blieb keine Zeit mehr. So können wir zur Zeit nicht sagen, was die höchstmögliche \gls{fs} mit der aktuellen Software ist.

\section{Daten-Reduktion und -Speicherung}
Die Datenreduktion durch Wahl der verschiedenen Detail-Level ist effizient gelöst und sehr effektiv. Pro Messwert müssen lediglich zwei Vergleiche für die Bestimmung des Signalpegels gemacht werden, sowie zwei Vergleiche mit je einem vorhergehenden Wert für die Bestimmung des Peak-Maximums und des Ereignis-Maximums.

Die Wahl des Detail-Levels beeinflusst lediglich, welche Daten übertragen werden. Auf die Zwischenspeicherung hat sie keinen Einfluss. Vor und nach der Übertragung finden keine komplexen Umrechnungen an den Messdaten statt. Die \gls{bitbreite} wird von 12 Bit auf 8 Bit reduziert. Die grösste Datenreduktion erfolgt durch die Auswahl der relevanten Daten.

\subsection{Rohdaten (raw)}
Es werden alle Messpunkte übertragen und gespeichert. Es findet keine Datenreduktion statt. Pro Messpunkt fällt 1 Byte an. Bei einer \gls{fs} von 10000~Hz resultiert ein Datenstrom von 10000~Byte/s.

\subsection{Detaillierte Ereignisdaten (detailed)}
Es werden nur die Messpunkte der Ereignisse gespeichert. Messpunkte, die nicht zu einem Ereignis gehören, werden verworfen. Der Datenstrom variiert daher mit der Häufigkeit der Ereignisse. Liegt zu 10 \% der Messzeit ein Ereignis vor, wird dies von der \gls{wsl} als hohes Geschiebeaufkommen eingestuft. Eine solche Periode kann über mehrere Stunden anhalten, ist aber nicht jeden Tag zu erwarten.

Bei 10 \% Ereigniszeit wird in diesem Modus eine Datenreduktion um 90 \% erzielt. Für die durchschnittliche Ereigniszeit nehmen wir 5 \% an. Dann resultiert pro \gls{sensor} ein Datenstrom von 500~Byte/s.

\subsection{Peaks (peaks only)}
Der Timestamp, die Dauer des Ereignisses, die Anzahl \glspl{peak} und alle Peakspitzen werden gespeichert. Pro Ereignis fallen 8 Byte an für die Eckdaten und 2 Byte pro Peak. Ein Ereignis hat im Normalfall weniger als 20 Peaks. Wir nehmen somit 48 Byte pro Ereignis an. Bei einem Ereignis pro Sekunde resultiert ein Datenstrom von rund 50 Byte/s.

\subsection{Minimale Daten (sparse)}
Es werden nur der Timestamp, die Dauer des Ereignisses, die Anzahl \glspl{peak} und der maximale Ausschlag des Ereignisses übertragen, das sind 8 Byte pro Ereignis. Der Datenstrom erreicht so lediglich 8 Byte/s.

\subsection{Messdauer}
Je nach Modus fallen sehr unterschiedliche Datenströme pro Sensor an. Ein Vergleich, wie lange mit einem GByte Speicherplatz gemessen werden kann, ist in Tabelle \ref{table.datarate} aufgeführt. Anhand dieser Werte kann abgeschätzt werden, wie lange eine Messstation ohne Wechsel des Speichermediums betrieben werden kann.

\begin{table}
\begin{center}
\begin{tabular}{|l|l|l|}
\hline \textbf{Detail-Level} & \textbf{Byte/s} & Messzeit/GByte\\ 
\hline raw                   & 10000 & 27~h \\
\hline detailed              &   500 & 23~d \\
\hline peaks only            &    50 & 231~d \\
\hline sparse                &     8 & 3.9~yr \\
\hline 
\end{tabular}
\caption{Vergleich des Datenaufkommens verschiedener Detail-Levels.}
\label{table.datarate}
\end{center}
\end{table} 

\section{Hardware}
Für den Datenlogger und die Sensoreinheiten wurde mit der Hilfe von Erich Ruff (ZHAW InES) und Valentin Schlatter (ZHAW InES) eine Leiterplatte entworfen und zwei Gehäusetypen gebaut. Die Leiterplatte wurde so entworfen, dass über die Bestückung entschieden wird, ob ein Datenlogger oder eine Sensoreinheit gebaut wird. Für einen Datenlogger wird die Leiterplatte mit einem SD-Karten-Slot bestückt. Die \gls{sensoreinh} wird mit ein Tiefpassfilter und dem Anschluss für den Sensor bestückt. Beide Varianten enthalten die Spannungsversorgung (12~V auf 5~V), einen CAN-Transceiver und die Anschlüsse für die Kabel. Der Schaltplan und das Leiterplattenlayout befinden sich im Anhang \ref{app.pcb}
% !TeX spellcheck = de_CH
%%%%%%%%%%%%%%%%%%%%%%%%%%%%%%%%%%%%%%%%%%%%%%%%%%%%%%%%%%%%%%%%%
%  _____   ____  _____                                          %
% |_   _| /  __||  __ \    Institute of Computitional Physics   %
%   | |  |  /   | |__) |   Zuercher Hochschule Winterthur       %
%   | |  | (    |  ___/    (University of Applied Sciences)     %
%  _| |_ |  \__ | |        8401 Winterthur, Switzerland         %
% |_____| \____||_|                                             %
%%%%%%%%%%%%%%%%%%%%%%%%%%%%%%%%%%%%%%%%%%%%%%%%%%%%%%%%%%%%%%%%%
%
% Project     : BA Welti Keller
% Title       : 
% File        : diskussion.tex Rev. 00
% Date        : 15.09.2014
% Author      : Tobias Welti
%
%%%%%%%%%%%%%%%%%%%%%%%%%%%%%%%%%%%%%%%%%%%%%%%%%%%%%%%%%%%%%%%%%

\chapter{Diskussion}\label{chap.diskussion}
\todo{haben wir erfüllt?}
\todo{wo gabs schwierigkeiten?}
\todo{worauf sind wir stolz}
\todo{was könnte man jetzt weiter noch machen?}
\todo{was ist noch geplant?}

% !TeX spellcheck = de_CH
%%%%%%%%%%%%%%%%%%%%%%%%%%%%%%%%%%%%%%%%%%%%%%%%%%%%%%%%%%%%%%%%%
%  _____   ____  _____                                          %
% |_   _| /  __||  __ \    Institute of Computitional Physics   %
%   | |  |  /   | |__) |   Zuercher Hochschule Winterthur       %
%   | |  | (    |  ___/    (University of Applied Sciences)     %
%  _| |_ |  \__ | |        8401 Winterthur, Switzerland         %
% |_____| \____||_|                                             %
%%%%%%%%%%%%%%%%%%%%%%%%%%%%%%%%%%%%%%%%%%%%%%%%%%%%%%%%%%%%%%%%%
%
% Project     : BA Welti Keller
% Title       : 
% File        : bedienung.tex Rev. 00
% Date        : 15.09.2014
% Author      : Tobias Welti
%
%%%%%%%%%%%%%%%%%%%%%%%%%%%%%%%%%%%%%%%%%%%%%%%%%%%%%%%%%%%%%%%%%

\chapter{Bedienungsanleitung}\label{chap.bedienung}

\section{Produktbeschrieb}\label{sec.manualproduct}


\section{Aufbau der Messstation}\label{sec.manualoverview}
Stromversorgung, Verdrahtung, Can-Bus, Terminator, R2D2, C3PO

\section{Datenlogger}\label{sec.manuallogger}


\section{Sensor}\label{sec.manualsensor}


\section{Ereignis}\label{sec.manualimpact}
\todo{Beschreibe hier die Ereignisse, die Detaillevel inklusive den tollen grafiken}


\section{Konfiguration}\label{sec.manualkonfig}


\subsection{Anschluss eines Computers}
Am USB-Anschluss des \gls{logger}s kann ein \gls{compi} angeschlossen werden, um auf die serielle Schnittstelle des \gls{logger}s zuzugreifen. Um die serielle Schnittstelle zu verwenden, wird ein \gls{terminalemu} wie \emph{PuTTY} oder \emph{minicom} benötigt. um mit \emph{PuTTY} eine Verbindung aufzubauen, muss die Schnittstelle und die Übertragungsrate (Baud) angegeben werden. Die Übertragungsrate ist 9600 baud, die Schnittstelle kann variieren. 

\paragraph{Windows} Unter \emph{Windows} erfolgt die Verbindung auf eine der COMx-Schnittstellen. Die Nummer der COM-Schnittstelle kann im Geräte-Manager herausgesucht werden, die Bezeichnung lautet 'mbed Serial Port (COMx)', wobei 'x' eine Nummer ist. In PuTTY muss nur 'COMx' angegeben werden.

\paragraph{Linux} Unter \emph{Linux} findet man die Schnittstellenbezeichnung mit dem Befehl 'ls /dev/ttyACM*' heraus, in \emph{PuTTY} wird dann '/dev/ttyACMx' angegeben. 

\paragraph{Mac OS X} Unter \emph{Mac OS X} lautet der Befehl 'ls /dev/tty.usbmodem*', der in einem Terminal eingegeben werden muss. Als \gls{terminalemu} kann 'screen' verwendet werden. Auf Apple Mac Computern mit USB 3.0 kann es zu Schwierigkeiten mit der Verbindung kommen. Den Herstellern des Prozessorboards ist dies bekannt, sie arbeiten an einer Lösung.

Die Einstellungen für die serielle Schnittstelle sind normalerweise bereits korrekt gesetzt. Es werden 8 Datenbits verwendet, 1 Stopbit und keine Parität (parity).

Weitere Hilfe für die Verwendung eines \gls{terminalemu}s findet man unter \url{http://developer.mbed.org/handbook/Terminals}.


\subsection{Menü}\label{ssec.menu}
Beim Herstellen der Verbindung über einen \gls{terminalemu} wird das Basis-Menü angezeigt. Durch Eingabe der Zahl wird der entsprechende Menü-Eintrag gewählt. Im Folgenden wird das gesamte Menü im Detail beschrieben.

Das Basis-Menü (siehe Listing \ref{list.basemenu}) listet alle Überwachungs- und Konfigurations-Möglichkeiten auf. 

\begin{lstlisting}[caption=, label=list.basemenu]
 1) list files
 2) format SD card
 3) mount SD card
 4) unmount SD card
 5) logger status
 6) start/stop logging
 7) sensor parameters
 8) sensor states
 9) reset timestamp
10) internal clock
11) config file
\end{lstlisting}

\paragraph{Dateien auflisten} Mit dem Befehl 'list files' wird eine Liste aller Dateien auf der SD-Karte angezeigt. Die Liste enthält die Dateigrösse sowie den Dateinamen, siehe Abschnitt \ref{sssec.listfiles}.

\paragraph{SD-Karte formatieren} Um die SD-Karte für den ersten Gebrauch vorzubereiten, sollte sie formatiert werden. Dies erfolgt von Vorteil auf einem \gls{compi}, kann aber auch im \gls{logger} mit dem Befehl 'format SD card' gemacht werden, siehe Abschnitt \ref{sssec.sdformat}.

\paragraph{SD-Karte anmelden} Nach dem Einsetzen einer SD-Karte erkennt der \gls{logger} dies normalerweise automatisch. Es kann jedoch vorkommen, dass der \gls{logger} auf die neue Karte aufmerksam gemacht werden muss. Dies erfolgt mit dem Befehl 'mount SD card', siehe Abschnitt \ref{sssec.sdmount}.

\paragraph{SD-Karte abmelden} Vor dem Entfernen der SD-Karte müssen alle Dateien geschlossen werden. Dies erfolgt mit dem Befehl 'unmount SD card', siehe Abschnitt \ref{sssec.sdunmount}.

\paragraph{Status des Datenloggers} Mit dem Befehl 'logger status' werden einige Betriebszustandsdaten des Datenloggers angezeigt, siehe Abschnitt \ref{sssec.loggerstate}.

\paragraph{Aufzeichnung starten/stoppen} Um die Aufzeichnung im ganzen System zu starten oder zu stoppen wird der Befehl 'start/stop logging' verwendet, siehe Abschnitt \ref{sssec.startstop}.

\paragraph{Sensor-Einstellungen} Mit dem Befehl 'sensor parameters' kann eine einzelne \gls{sensoreinh} oder alle \glspl{sensoreinh} zusammen konfiguriert werden. Siehe Abschnitt \ref{sssec.sensorparam}.

\paragraph{Status der Sensoreinheiten} Der Betriebszustand aller angeschlossenen \glspl{sensoreinh} kann mit dem Befehl 'sensor state' (siehe \ref{sssec.sensorstate}) aufgelistet werden.

\paragraph{Timestamp zurücksetzen} Um den Timestamp in allen \glspl{sensoreinh} auf Null zurückzustellen, wird der Befehl 'reset timestamp' verwendet. Siehe Abschnitt \ref{sssec.timestamp}.

\paragraph{Interne Uhr} Die interne Uhr wird mit den Befehl 'internal clock' eingestellt, Abschnitt \ref{sssec.intclock} beschreibt dies im Detail.

\paragraph{Konfigurations-Datei} Mit dem Befehl 'config file' wird die Konfiguration der Sensoren abgespeichert oderr aus einer Datei eingelesen, siehe Abschnitt \ref{sssec.configfile}.

\subsection{Befehle}\label{ssec.befehle}

\subsubsection{Dateiliste}\label{ssec.listfiles}
\begin{lstlisting}[caption=, label=list.]
HIER LISTE DER FILES EINFueGEN
 0) exit
\end{lstlisting}


\subsubsection{SD-Karte formatieren}\label{sssec.sdformat}
\begin{lstlisting}[caption=Untermenü SD-Karte formatieren, label=list.sdformat]
 1) confirm formatting of SD card.
    All data will be erased.
 0) cancel
\end{lstlisting}

Beim Formattieren werden alle Dateien auf der SD-Karte gelöscht, inklusive der Konfigurationsdatei mit allen Sensor-Einstellungen. Der Befehl 'format SD card' holt vor der Ausführung nochmals eine Bestätigung ein, ob sich der Benutzer wirklich sicher ist, dass er alle Dateien löschen will (Listing \ref{list.sdformat}). Während dem Formatieren wird die Meldung \ref{list.sdformatting} angezeigt.

\begin{lstlisting}[caption=Statusmeldung SD formatieren, label=list.sdformatting]
formatting SD Card
\end{lstlisting}

Sind beim Formatieren Fehler aufgetreten, erhält man die Fehlermeldung \ref{list.sdformatfail}. In diesem Fall sollte die Karte in einem \gls{compi} geprüft und formatiert werden.

\begin{lstlisting}[caption=Fehlermeldung SD formatieren, label=list.sdformatfail]
Formatting SD card FAILED. Please use a Computer to format the card.
\end{lstlisting}

Bei erfolgreicher Formatierung wird die Meldung \ref{list.sdformatsuccess} ausgegeben. 

\begin{lstlisting}[caption=Erfolgsmeldung SD formatieren, label=list.sdformatsuccess]
Formatting done
Returning to base menu.
\end{lstlisting}


\subsubsection{SD-Karte anmelden}\label{sssec.sdmount}
Damit Dateien auf die SD-Karte geschrieben werden können, muss sie vorher erkannt werden. Normalerweise geschieht dies, sobald die Karte eingesetzt wird. Wenn keine SD-Karte erkannt wird, wird dies im Basismenü angezeigt wie im Listing \ref{list.sdmissing}. Durch den Aufruf des Befehls 'mount SD card' im Basismenü (Listing \ref{list.basemenu}) kann die eingesetzte SD-Karte angemeldet werden.

Nach erfolgreicher Anmeldung der SD-Karte wird das Basismenü angezeigt. 

Wenn keine SD-Karte erkannt werden kann, wird eine Fehlermeldung \ref{list.sdmountfail} ausgegeben.

\begin{lstlisting}[caption=Fehlermeldung SD-Karte anmelden, label=list.sdmountfail]
No SD card detected! Please insert card and try again!
\end{lstlisting}


\subsubsection{SD-Karte abmelden}\label{sssec.sdunmount}
Bevor die SD-Karte aus dem \gls{logger} entfernt wird, sollte sie abgemeldet werden. Der \gls{logger} schliesst bei diesem Vorgang alle geöffneten Dateien, um Datenverlust zu vermeiden. Da beim Abmelden der Karte die Aufzeichnung der Daten gestoppt wird, wird vorher eine Bestätigung verlangt (Listing \ref{list.sdunmount}).

\begin{lstlisting}[caption=Untermenü SD-Karte abmelden, label=list.sdunmount]
 1) unmount SD card
    This will stop logging and close all data files.
 0) cancel
\end{lstlisting}

Falls die Konfiguration der Sensoren verändert, aber noch nicht in die Konfigurationsdatei geschrieben wurde, wird eine Warnung angezeigt (Listing \ref{list.sdunmountwarn}). Die Konfiguration bleibt im Speicher des \gls{logger}s erhalten, so lange die Spannungsversorgung angeschlossen ist und kann auch auf der neuen SD-Karte gespeichert werden.

\begin{lstlisting}[caption=Warnung vor SD-Karte abmelden bei ungespeicherter Konfiguration, label=list.sdunmountwarn]
******************************************************************
* WARNING: sensor configuration data has not been saved to file! *
* If you want to save config to file, cancel now.                *
******************************************************************
\end{lstlisting}

\subsubsection{Logger-Status}\label{sssec.loggerstate}
Mit dem Befehl 'logger status' können einige Information über den \gls{logger} angezeigt werden.

\todo{logger status listing}
\begin{lstlisting}[caption=Untermenü Logger-Status, label=list.loggerstatus]

\end{lstlisting}


\subsubsection{Starten und stoppen der Aufzeichnung}\label{sssec.startstop}
Um die Datenspeicherung im \gls{logger} zu unterbrechen oder wieder zu starten wird der Befehl 'start/stop logger' verwendet. Beim Aufruf des Befehls wird ein Untermenü gemäss Listing \ref{list.startstop} angezeigt.

\begin{lstlisting}[caption=Untermenü Stoppen der Aufzeichnung, label=list.startstop1]
 Logger is running.
 1) stop the logging.
 0) cancel
\end{lstlisting}

Da sich das Menü dem gegenwärtigen Zustand anpasst, sieht es bei gestoppter Aufzeichnung aus wie in Listing \ref{list.startstop2}.

\begin{lstlisting}[caption=Untermenü Starten der Aufzeichnung, label=list.startstop2]
 Logger is stopped.
 1) start the logging.
 0) cancel
\end{lstlisting}

Wenn die Aufzeichnung am Logger gestoppt wird, wird an alle \glspl{sensoreinh} der Befehl zum Aufzeichnungsstopp gesendet. Die Einstellungen zum Detailmodus bleiben in der \gls{sensoreinh} aber erhalten. Beim erneuten Starten der Aufzeichnung im Logger werden auch die \glspl{sensoreinh} wieder gestartet. Es besteht auch die Möglichkeit, einzelne \glspl{sensoreinh} zu stoppen (siehe Abschnitt \ref{sssec.sensorparam}). Eine gestoppte \gls{sensoreinh} bleibt auch beim Starten der Aufzeichnung am \gls{logger} gestoppt, da sie schon vor dem Stopp in diesem Zustand war.

\subsubsection{Sensor-Parameter}\label{sssec.sensorparam}
Die Parameter der Datenerfassung und Ereigniserkennung können für alle \glspl{sensoreinh} gemeinsam oder für jede \gls{sensoreinh} individuell eingestellt werden. Die Auswahl einer einzelnen oder aller \glspl{sensoreinh} erfolgt beim Einstieg in das Untermenü der Sensor-Parameter. Listing \ref{list.sensorsel} zeigt die Auswahl der Sensoren. Die Auswahlliste enthält gleich die aktuellen Werte der Parameter, damit man eine Übersicht hat.

\todo{sensorauswahl listet die sensoren auf, das noch einfügen}

\begin{lstlisting}[caption=Untermenü Sensor-Auswahl, label=list.sensorsel]
 #) Select a sensor from the list.
99) Select all sensors.
 0) cancel
\end{lstlisting}

Nach der Auswahl einer \gls{sensoreinh} gelangt man zur Auswahl des anzupassenden Parameters, Listing \ref{list.sensorparam}. Ist ein einzelner Sensor ausgewählt, werden hier noch einmal die aktuellen Werte der Parameter angezeigt.

\begin{lstlisting}[caption=Untermenü Sensor-Parameter, label=list.sensorparam]
 1) set sampling rate (current: 10000 Hz)
 2) set threshold value (current: 200)
 3) set baseline value (current: 2047)
 4) set timeout (current: 30)
 5) set detail level (current: peaks only)
 6) start or stop recording (current: started)
 0) exit
\end{lstlisting}

\paragraph{Abtastrate} Die Abtastrate legt fest, wie oft pro Sekunde ein Messwert vom Beschleunigungssensor eingelesen werden soll. Die \glspl{sensoreinh} können in einem Bereich zwischen \ensuremath{100 Hz} und \ensuremath{200000 Hz} messen. Die Abtastrate kann in Schritten von \ensuremath{100 Hz} eingestellt werden, Listing \ref{list.paramfs}.

Die Abtastrate hat einen wesentlichen Einfluss auf die zu übertragende und zu speichernde Datenmenge, die benötigte Rechenleistung. Davon wiederum hängt die Zeitspanne ab, wie lange die Messstation ohne Wartung betrieben werden kann. Es wird deshalb empfohlen, die Abtastrate nur so hoch einzustellen, wie es wirklich benötigt wird. Wie hoch dieser Wert ist, hängt stark von den geplanten Auswertungen ab.

\begin{lstlisting}[caption=Untermenü Abtastrate, label=list.paramfs]
 #) Enter sampling rate in Hz. (multiple of 100 Hz in range 100..200'000 Hz)
 0) cancel
\end{lstlisting}

Bei Eingabe einer ungültigen oder nicht unterstützten Abtastrate wird eine Fehlermeldung ähnlich Listing \ref{list.paramfserror} angezeigt.

\begin{lstlisting}[caption=Fehlermeldung bei ungültiger Abtastrate, label=list.paramfserror]
Sampling rate 220000 Hz not supported, too high.
\end{lstlisting}

\paragraph{Ereigniserkennung} Die \gls{ereignisdet} hat drei Parameter, die die Form der gesuchten \glspl{ereignis} bestimmen.  Abbildung \ref{fig.params} illustriert die Zusammenhänge zwischen \gls{threshold} und \gls{nullpegel} sowie die Funktionsweise des \gls{timeout}s.

\paragraph{Threshold} Der \gls{threshold} (Schwellenwert) ist ein Parameter der Ereigniserkennung. Er bestimmt, ab welcher Abweichung vom Nullwert ein Signal als Peak betrachtet werden soll. Bei der Wahl des \gls{threshold}s ist zu beachten, dass der \gls{threshold} auf beide Seiten des Nullwerts gilt. Daher darf die Summe des \gls{nullpegel}s und des \gls{threshold}s nicht den maximalen Wert (4096) des \gls{adwandler}s überschreiten. Ebenso muss der Wert des \gls{threshold}s kleiner sein als der \gls{nullpegel}, damit kein negativer Messwert anliegen müsste, um einen Peak zu erzeugen. Diese Einschränkungen werden bei der Eingabe noch einmal angezeigt, Listing \ref{list.paramthres}.

\begin{figure}
	\centering
		\includegraphics[width=0.8\textwidth]{images/impact_params.png}
	\caption{Zusammenhänge der Parameter der \gls{ereignisdet}.}
	\label{fig.params}
\end{figure}

\begin{lstlisting}[caption=Untermenü Threshold, label=list.paramthres]
 #) Enter threshold value.
    baseline + threshold must not exceed 4096
    and
    baseline - threshold must not be below 0
 0) cancel
\end{lstlisting}

Bei Verletzung der Kriterien für den \gls{threshold} wird eine entsprechende Fehlermeldung angezeigt, Listing \ref{list.paramthresfail}. Da die gleichen Kriterien auch bei der Einstellung des \gls{nullpegel}s gelten, empfiehlt es sich, zuerst einen kleinen Wert für den \gls{threshold} zu wählen. Dann kann der \gls{nullpegel} ohne grosse Einschränkung eingestellt werden. Danach setzt man den passenden \gls{threshold}.

\begin{lstlisting}[caption=Fehlermeldung ungültiger Threshold, label=list.paramthresfail]
 Invalid threshold value:
 threshold + baseline must not exceed 4096
 and
 threshold must be smaller than baseline value.
\end{lstlisting}

\paragraph{Nullpegel} Der \gls{nullpegel} wird mit einer ähnlichen Maske (Listing \ref{list.parambase}) wie der \gls{threshold} eingestellt, auch die Einschränkungen für den Wertebereich sind die Gleichen.

\begin{lstlisting}[caption=Untermenü Null-Level, label=list.parambase]
 #) Enter baseline value (default: 2047).
 0) cancel
\end{lstlisting}

Die Fehlermeldung bei ungültigen Werten für den \gls{nullpegel} ist in Listing \ref{list.parambaseerror} aufgeführt.

\begin{lstlisting}[caption=Fehlermeldung ungültiger Nullpegel, label=list.parambaseerror]
Invalid baseline value:
threshold + baseline must not exceed 4096.
and
threshold - baseline must not be below 0 value
\end{lstlisting}

\paragraph{Timeout} Der \gls{timeout} definiert, wie viele Samples der Signalwert unterhalb des \gls{threshold}s liegen kann, bevor das \gls{ereignis} als beendet betrachtet wird (Listing \ref{list.paramtimeout}). Die einzige Einschränkung an den \gls{timeout} ist, dass er die Länge des Ereignispuffers nicht überschreiten darf.
\todo{Länge des Ereignispuffers}

\begin{lstlisting}[caption=Untermenü Timeout, label=list.paramtimeout]
 #) Enter timeout in samples.
 0) cancel
\end{lstlisting}

Bei zu langem (Listing \ref{list.paramtimeoutlong}) oder sehr kurzem (Listing \ref{list.paramtimeoutshort}) \gls{timeout} wird eine Fehlermeldung resp. Warnung angezeigt.

\begin{lstlisting}[caption=Fehlermeldung zu langer Timeout, label=list.paramtimeoutlong]
Timeout too long, can not exceed 512.
\end{lstlisting}

\begin{lstlisting}[caption=Warnung kurzer Timeout, label=list.paramtimeoutshort]
Timeout 0 will end impact after each peak.
Timeout 0 in effect.
\end{lstlisting}

\paragraph{Detaillevel} Über die Wahl des Detaillevels wird bestimmt, wie viele und welche Daten von jedem \gls{ereignis} übertragen und gespeichert werden sollen (Listing \ref{list.detail}). Die Detaillevel sind geordnet nach anfallender Datenmenge, beginnend mit dem grössten Aufwand. Die Detaillevel sind in Abschnitt \ref{sec.manualimpact}, Seite \pageref{sec.manualimpact} beschrieben.

\begin{lstlisting}[caption=Untermenü Detail-Level, label=list.detail]
 1) raw (continuous data)
 2) detailed (start time, all samples of impact)
 3) peaks only (start time, nr of peaks, peaks
 4) sparse (only start time, duration, nr of peaks, max peak)
 5) off
 0) cancel
\end{lstlisting}

\paragraph{Start/Stop Sensor} Jede \gls{sensoreinh} kann einzeln gestartet oder gestoppt werden, vorausgesetzt der \gls{logger} ist gestartet. Im Untermenü ist ersichtlich, in welchem Zustand die ausgewählte \gls{sensoreinh} gerade ist (listing \ref{list.started_one}). Falls die Konfigurationsänderung alle \glspl{sensoreinh} betreffen soll, wird die Anzahl gestarteter und gestoppter Sensoren angezeigt (listing \ref{list.started_all}).

\begin{lstlisting}[caption=Untermenü Start/Stop einzeln, label=list.started_one]
Selected sensor is currently stopped.
 1) start
 2) stop
 0) cancel
\end{lstlisting}

\begin{lstlisting}[caption=Untermenü Start/Stop alle Sensoren, label=list.started_all]
Started sensors: 3
Stopped sensors: 0
 1) start
 2) stop
 0) cancel
\end{lstlisting}

Wenn ein Sensor gestartet wird, muss für die anfallenden Daten eine Datei erzeugt werden. Schlägt dies fehl, wird dies mit der Fehlermeldung \ref{list.sensorerror} angezeigt. Der Sensor wird dann nicht gestartet. Es wird empfohlen, in diesem Fall die SD-Karte zu überprüfen. Möglicherweise verfügt die SD-Karte nicht mehr über genügend Speicherplatz.

\begin{lstlisting}[caption=Fehlermeldung beim Starten eines Sensors, label=list.sensorerror]
Could not create or open file. Please check SD card for free space.
\end{lstlisting}

\subsubsection{Sensor-Status}\label{sssec.sensorstate}
\todo{Liste von Sensor-Stati ins Listing einfügen}
\begin{lstlisting}[caption=Untermenü Sensor-Status, label=list.sensorstatus]
Listing sensor config:
HIER EINFueGEN
 0) continue
\end{lstlisting}


\subsubsection{Timestamp zurücksetzen}\label{sssec.timestamp}
\begin{lstlisting}[caption=Untermenü Timestamp zurücksetzen, label=list.timestamp]
 1) re-synchronize timestamp
 0) cancel
\end{lstlisting}


\subsubsection{Interne Uhr}\label{sssec.intclock}
\todo{Beispiel einer Uhrzeit einfügen}
\begin{lstlisting}[caption=Untermenü interne Uhr, label=list.intclock]
 1) adjust date
 2) adjust time
 0) exit

 current time: HIER EINFueGEN
\end{lstlisting}

\begin{lstlisting}[caption=Untermenü Datum einstellen, label=list.setdate]
adjust internal date
 1) adjust date +365 days
 2) adjust date -365 days
 3) adjust date + 30 days
 4) adjust date - 30 days
 5) adjust date + 10 days
 6) adjust date - 10 days
 7) adjust date +  1 day
 8) adjust date -  1 day
 0) exit

 current time: HIER EINFueGEN
\end{lstlisting}

\begin{lstlisting}[caption=Untermenü Uhrzeit einstellen, label=list.settime]
adjust internal time
 1) adjust time +1 hour
 2) adjust time -1 hour
 3) adjust time +10 minute
 4) adjust time -10 minute
 5) adjust time +1 minute
 6) adjust time -1 minute
 7) adjust time +1 second
 8) adjust time -1 second
 0) exit

 current time: HIER EINFueGEN
\end{lstlisting}

\subsubsection{Konfigurations-Datei}
\begin{lstlisting}[caption=Untermenü Konfigurationsdatei, label=list.config]
 1) read configuration from file and set up all sensors accordingly.
 2) store current configuration in file. Old config file will be overwritten.
 0) cancel
\end{lstlisting}

\begin{lstlisting}[caption=Fehlermeldung beim Speichern der Konfigurationsdatei, label=list.configerror]
The config file could not be written. Please check the SD card in a computer.
The configuration data will remain stored in the logger unless you turn off power.
\end{lstlisting}

\subsection{Konfigurationsdatei}
\begin{lstlisting}[caption=Beispiel einer Konfigurationsdatei, label=list.configfile]
{7,
{BLABLABLA}
{BLABLABLA}
{BLABLABLA}
}
\end{lstlisting}

Config has been modified but not saved to SD card.

\section{Betrieb}


\section{Technische Daten}


\chapter{Verzeichnisse}\label{chap.verzeichnisse}
 %%%%%%%%%%%%%%%%%%%%%%%%%%%%%%%%%%%%%%%%%%%%%%%%%%%%%%%%%%%%%%%%%%
%  _____   ____  _____                                          %
% |_   _| /  __||  __ \    Institute of Computitional Physics   %
%   | |  |  /   | |__) |   Zuercher Hochschule Winterthur       %
%   | |  | (    |  ___/    (University of Applied Sciences)     %
%  _| |_ |  \__ | |        8401 Winterthur, Switzerland         %
% |_____| \____||_|                                             %
%%%%%%%%%%%%%%%%%%%%%%%%%%%%%%%%%%%%%%%%%%%%%%%%%%%%%%%%%%%%%%%%%
%
% Project     : LaTeX doc Vorlage für Windows ProTeXt mit TexMakerX
% Title       : 
% File        : literatur.tex Rev. 00
% Date        : 23.4.12
% Author      : Remo Ritzmann
% Feedback bitte an Email: remo.ritzmann@pfunzle.ch
%
%%%%%%%%%%%%%%%%%%%%%%%%%%%%%%%%%%%%%%%%%%%%%%%%%%%%%%%%%%%%%%%%%

\begin{thebibliography}{99}
\addcontentsline{toc}{section}{Literaturverzeichnis}\label{cha:literaturverzeichnis}

% How to make a Literaturlist nach www.ieee.org/documents/ieeecitationref.pdf
% Erklärung

%Books
\bibitem{robotvision} B. Klaus and P. Horn, Robot Vision. Cambridge, MA: MIT Press, 1986.
%Einzelne Seiten aus Buch
\bibitem{randompatterns} L. Stein, ">Random patterns,"> in Computers and You, J. S. Brake, Ed. New York: Wiley, 1994, pp. 55-70.



\end{thebibliography}

\Urlmuskip=0mu plus 1mu\relax
 \bibliographystyle{plain}
 \begingroup
 \raggedright
 \bibliography{literatur}
 \endgroup \newpage
 \listoffigures
 %\addcontentsline{toc}{section}{(Abbildungsverzeichnis)}
 \listoftables
 %\addcontentsline{toc}{section}{(Tabellenverzeichnis)}
 \printglossary
 %\addcontentsline{toc}{section}{(Glossar)}
 \printglossary[type=acronym]
 %\addcontentsline{toc}{section}{(Abkürzungen)}
 \lstlistoflistings
 %\addcontentsline{toc}{section}{(Listingverzeichnis)}
 
% !TeX spellcheck = de_CH
%%%%%%%%%%%%%%%%%%%%%%%%%%%%%%%%%%%%%%%%%%%%%%%%%%%%%%%%%%%%%%%%%
%  _____   ____  _____                                          %
% |_   _| /  __||  __ \    Institute of Computitional Physics   %
%   | |  |  /   | |__) |   Zuercher Hochschule Winterthur       %
%   | |  | (    |  ___/    (University of Applied Sciences)     %
%  _| |_ |  \__ | |        8401 Winterthur, Switzerland         %
% |_____| \____||_|                                             %
%%%%%%%%%%%%%%%%%%%%%%%%%%%%%%%%%%%%%%%%%%%%%%%%%%%%%%%%%%%%%%%%%
%
% Project     : BA Welti Keller
% Title       : 
% File        : anhang.tex Rev. 00
% Date        : 15.09.2014
% Author      : Tobias Welti
%
%%%%%%%%%%%%%%%%%%%%%%%%%%%%%%%%%%%%%%%%%%%%%%%%%%%%%%%%%%%%%%%%%


\pagenumbering{Roman}

\appendix
\chapter{Anhang}\label{chap.anhang}


\section{Offizielle Aufgabenstellung}\label{app.aufgabenstellung}
\includepdf{images/BA_welti_keller_hs14.pdf}


\section{Projektmanagement}\label{app.projektmanagement}

\subsection{Zeitplanung und Meilensteine}
Abbildung \ref{fig.gantt} zeigt den Gantt-Chart des Projekts. Die Konzeptphase wurde gemeinsam bearbeitet. Danach wurden die Arbeiten aufgeteilt, T. Keller entwickelte das CAN-Protokoll und die zugehörigen \gls{ipc}-Abläufe. Die Konfiguration und Bedienung der Messstation wurde von T. Welti entwickelt. Die Aufteilung aller Arbeitspakete ist im Gantt-Chart auf Seite \pageref{fig.gantt} aufgeführt.

\begin{figure}
	\centering		\includegraphics[angle=270,width=1\textwidth]{images/gantt.pdf}
	\caption{Gantt-Chart des Projekts.}
	\label{fig.gantt}
\end{figure}

\subsubsection{Meilenstein 1: Hardware in Betrieb}
Ein erstes Konzept wurde pünktlich zum Meilenstein 1 erstellt. Bezüglich der Art, wie die Dokumentation (vor allem welche Diagramme am besten geeignet wären) im Detail auszusehen hat, bestanden noch Unsicherheiten. Soweit sinnvoll soll UML verwendet werden, damit die Diagramme verständlich sind.

\subsubsection{Meilenstein 2: Gehäuse und Leiterplatte}
Die Leiterplatte wurde in unserem Auftrag und unter Mithilfe von T. Welti durch Valentin Schlatter (ZHAW InES) entwickelt. Beim Entwurf und dem Bau der Gehäuse stand Erich Ruff (ZHAW InES) als Berater zur Seite und übernahm die Bohrarbeiten. Dank ihrer Mithilfe konnte der Meilenstein 2 termingerecht abgeschlossen werden.

\subsubsection{Meilenstein 3: Software}
Der dritte Meilenstein konnte nicht eingehalten werden. Die ersten proof-of-concept-Implementationen sahen zwar vielversprechend aus, allerdings tauchten bei der Detailimplementation des CAN-Protokolls Schwierigkeiten auf, die einige Umstellungen notwendig machten. So konnte der Meilenstein erst mit drei Wochen Verspätung erreicht werden.

\subsubsection{Meilenstein 4: Testing}
Aufgrund der Verzögerung des zweiten Meilensteins konnte auch dieser Meilenstein nicht fristgereicht erreicht werden. Zudem ist das Testing rudimentär ausgefallen, da nach der Verzögerung kaum mehr Zeit für intensive Tests blieb. Wir waren aus terminlichen Gründen gezwungen, die Arbeit auf den MS~5 zu fokussieren.

\subsubsection{Meilenstein 5: Dokumentation}
Der letzte Meilenstein konnte fristgerecht erreicht werden.


\subsection{Besprechungsprotokolle}

\subsubsection{Projektskizzierung an der WSL, Birmensdorf}
Die erste Sitzung zum Projekt fand am 16. Juli 2014 an der \gls{wsl} in Birmensdorf statt. Anwesende waren Bruno Fritschi (Projektidee), Carlos Wyss (Doktorand), Alexandre Badoux (Gruppenleiter) sowie Tobias Keller und Tobias Welti.

Die Projektidee wurde von Bruno Fritschi noch einmal erklärt. Carlos Wyss zeigte einige Beispiele von bestehenden Messresultaten und erläuterte den Algorithmus der \gls{ereignisdet}, die mittels Hilbert-Transformation gelöst wird. Die Datenreduktion erfolgt durch Berechnung eines Histogramms der Peakintensitäten über eine Minute und die Speicherung der Dauer, Intensität und Anzahl \glspl{peak} jedes \gls{ereignis}ses. Als minimale Anforderung an die Messdaten wurde vereinbart, dass die \gls{fs} mindestens 10~kHz beträgt und mindestens die gleichen Werte gespeichert werden wie bisher.

Als Wunsch wurde von Carlos Wyss geäussert, die Originaldaten von Anfang bis Ende der \gls{ereignis} speichern zu können.


\subsubsection{Zwischenbesprechung an der ZHAW, Winterthur}
Eine Zwischenbesprechung am 28. Oktober 2014 an der ZHAW Winterthur verlief sehr zufriedenstellend. Anwesend waren Prof. Hans Gelke (Betreuer), Bruno Fritschi sowie Tobias Keller und Tobias Welti.

\paragraph{Projektstand} An der Zwischenbesprechung konnte bereits der Testaufbau mit einer \gls{sensoreinh} vorgeführt werden. Die Signale wurden auf einem Oszilloskop visualisiert und entsprachen den Erwartungen von Bruno Fritschi. Die Dimensionen der Geräte waren jedoch wesentlich grösser als erwartet. 

\paragraph{Beschlüsse} In der Diskussion wurde entschieden, die bereits im Aufbau befindlichen Gehäuse trotzdem fertig zu bauen. Das Projekt war zu diesem Zeitpunkt auf einem guten Weg, weshalb keine Massnahmen oder Änderungen an den Zielen beschlossen wurden.

\subsubsection{Zwischenbesprechung an der VAW, ETH Hönggerberg}
Am 12. November 2014 fand eine zweite Projektsitzung statt, diesmal an der \gls{vaw} an der ETH Hönggerberg. Anwesend waren Prof. Hans Gelke, Prof. Dieter Rickenmann (WSL), Bruno Fritschi, Carlos Wyss sowie Tobias Keller und Tobias Welti.

\paragraph{Projektstand} Die Methode der Ereigniserkennung mittels einer State Machine mit konfigurierbaren Parametern wurde vorgestellt und von den Anwesenden gelobt. Die verschiedenen Detail-Level wurden intensiv diskutiert und Vor- und Nachteile aufgeführt. 

\paragraph{Beschlüsse} Es wurde beschlossen, das Projekt mit allen vorgeschlagenen Detail-Levels weiterzuführen, da für jeden Modus ein plausibles Szenario gefunden wurde. Da das Projekt mit der \gls{ereignisdet} und Teilen des CAN-Protokolls gut im Zeitplan lag, wurden keine Änderungen der Ziele definiert. Die geplanten Tests an der \gls{vaw} wurden aber auf einen Zeitpunkt nach dem Abschluss der Bachelorarbeit verschoben.

Für die Tests in der Versuchsrinne ist vorgesehen, die Sensoren ausserhalb der Gehäuse der \glspl{sensoreinh} anzuschliessen und unter die bestehende Stahlplatte zu montieren. Dann können Vergleichsmessungen mit Geophonen und \gls{mems}-Beschleunigungssensoren durchgeführt werden.

Als Folgeprojekt oder möglicher Entwicklungsauftrag an die ZHAW wurde die Miniaturisierung  der \gls{sensoreinh} und deren Entwicklung zur Produktreife ins Auge gefasst.

Im Anschluss an die Sitzung wurde die \gls{vaw} besichtigt. Hier konnten sich die Entwickler zum ersten Mal ein (eindrückliches!) Bild der bestehenden Messinstallationen mit Geophonen machen.

\clearpage
\section{Schaltpläne}\label{app.pcb}
\includepdf[angle=90]{images/pcb/Schematic.pdf}
\includepdf[pages={1,2},angle=90]{images/pcb/PCB_Layout.pdf}


\section{Datenblätter}\label{app.datasheets}
Um die Dokumentation übersichtlich zu halten, wird der Grossteil der Datenblätter nicht mit der Dokumentation ausgedruckt, sondern auf der beiliegenden \gls{cd} mitgeliefert.


\subsection{NXP LPC4088 32-bit ARM Cortex-M4 microcontroller}
\includepdf[pages={7}]{images/datasheets/LPC408X_7X.pdf}\label{ds.lpc4088}


\subsection{Embedded Artists NXP LPC4088 QuickStart Board}

\begin{figure}
	\centering		\includegraphics[width=0.7\textwidth]{images/datasheets/lpc4088_qsb_key_components_reva.png}
	\caption{Hauptkomponenten des NXP LPC4088 QuickStart Boards von Embedded Artists.}
	\label{fig.NXP_LPC4088_QSB_comps}
\end{figure}

\begin{figure}
	\centering		\includegraphics[width=0.7\textwidth]{images/datasheets/LPC4088_QSB_pinning_revA_800x769.png}
	\caption{Pins des NXP LPC4088 QuickStart Boards von Embedded Artists.}
	\label{fig.NXP_LPC4088_QSB_pinout}
\end{figure}

\includepdf{images/datasheets/LPC4088qsb.pdf}

\subsection{Analog Devices ADXL001 Beschleunigungssensor}
\includepdf[pages={3}]{images/datasheets/ADXL001.pdf}


\section{Fotos}\label{sec.foto.testaufbau}
Die Fotos befinden sich auf der beiliegenden CD und können auch aus dem Git-Repository unter \url{https://github.com/tokeller/WeKeBA} heruntergeladen werden.

\section{Source Code, Daten und Multimedia}\label{app.cd}
Da der Source Code sehr umfangreich ist, wird darauf verzichtet, ihn ausgedruckt zur Verfügung zu stellen. Er befindet sich auf der beiliegenden \gls{cd}.


CD mit dem vollständigen Bericht als pdf-File inklusive Film- und Fotomaterial}
\begin{lstlisting}[caption=Inhaltsverzeichnis CD, label=list.cd]
Inhaltsverzeichnis CD
=====================
1_Dokumentation
  gesamte Dokumentation als .pdf-Datei sowie als LaTeX source code.
2_Projektplanung
  Microsoft Project Datei mit der Projektplanung
3_Datenblätter
  Datenblätter aller verwendeten Komponenten. 
4_Busprotokoll
  Excel-Datei mit den Message-Formaten
5_Leiterplatte
  Gerber-Daten, Layout, Schema, Stückliste der Leiterplatte.
6_source_code
  Der komplette Source Code des Projekts inkl. Projektfile für Keil uVision 5. 
  Der selbst verfasste Source Code liegt im Verzeichnis 
  mbed_BusHandlerV2/source/app/
7_Literatur
  pdf-Dateien der zitierten Literatur.
8_Fotos
  Fotos des Projekts, sortiert in einer Ordnerstruktur
9_Messdaten
  Messdaten von Versuchen am Testaufbau
\end{lstlisting}



\end{document}
