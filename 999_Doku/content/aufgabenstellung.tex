% !TeX spellcheck = de_CH
%%%%%%%%%%%%%%%%%%%%%%%%%%%%%%%%%%%%%%%%%%%%%%%%%%%%%%%%%%%%%%%%%
%  _____   ____  _____                                          %
% |_   _| /  __||  __ \    Institute of Computitional Physics   %
%   | |  |  /   | |__) |   Zuercher Hochschule Winterthur       %
%   | |  | (    |  ___/    (University of Applied Sciences)     %
%  _| |_ |  \__ | |        8401 Winterthur, Switzerland         %
% |_____| \____||_|                                             %
%%%%%%%%%%%%%%%%%%%%%%%%%%%%%%%%%%%%%%%%%%%%%%%%%%%%%%%%%%%%%%%%%
%
% Project     : BA Welti Keller
% Title       : 
% File        : aufgabenstellung.tex Rev. 00
% Date        : 15.09.2014
% Author      : Tobias Welti
%
%%%%%%%%%%%%%%%%%%%%%%%%%%%%%%%%%%%%%%%%%%%%%%%%%%%%%%%%%%%%%%%%%

\chapter{Aufgabenstellung}\label{chap.aufgabenstellung}

Die offizielle Aufgabenstellung befindet sich im Anhang \ref{app.aufgabenstellung}.

\section{Aufgabenstellung}\label{sec.aufgabenstellung}
Im Rahmen dieser Bachelorarbeit soll eine Lösung erarbeitet werden, um zukünftige Installationen günstiger zu machen. Da solche Messanlagen an sehr vielen Orten auf der ganzen Welt aufgebaut werden, kann durch eine Vereinfachung der Installation viel Aufwand gespart werden. 

Die Projektidee stammt von Bruno Fritschi (WSL). Sein Vorschlag sieht vor, die aufgezeichneten Signale direkt am \gls{sensor} auszuwerten und nur die gewünschten \gls{ereignis}-Daten zu übertragen und zu speichern. Somit könnten die Daten über ein \gls{bussys} übertragen werden und der Rechner für die Sammlung der Daten bräuchte weniger Rechenleistung.

Dank der Bustopologie kommt das Messsystem mit weniger Leitungen aus und kann einfacher installiert werden. Denkbar wäre die Integration in eine Elastomerplatte anstelle der Stahl- und Betonkonstruktion, da viel weniger Leitungen nötig sind. Die Elastomerplatte könnte einfacher im Bachbett verankert werden.

Ziel der Arbeit ist die Entwicklung der Auswertungshardware und des \gls{bussys}s. Die Qualität der gemessenen Signale soll mindestens erhalten werden. Die Auswertungsalgorithmen sind nicht Bestandteil der Arbeit und werden vom WSL zur Verfügung gestellt.

Die von der bisherigen Anlage gemachten Messdaten enthalten die Dauer und Intensität jedes Aufschlags (\gls{ereignis}) auf der Sensorplatte, sowie die Anzahl Ausschläge (Peaks) pro Aufschlag. Pro Minute wird ein Histogramm über die Intensitäten der Peaks gebildet und abgespeichert.

Denkbar wäre es, einen Prototyp für Vergleichsmessungen im Erlenbach (Alptal, SZ) an einer bestehenden Schwelle zu implementieren.


\subsection{Musskriterien}
\begin{itemize}
\item Die Anlage zeichnet den Geschiebetransport im Bachbett auf. Die bisherige Aufzeichnungsrate von 10'000 Messpunkten pro Sekunde soll nicht unterschritten werden.
\item Die Anlage liefert eine minütliche Zusammenfassung über die \glspl{ereignis} an jedem \gls{sensor}. Diese Zusammenfassung enthält die Anzahl \glspl{ereignis}, Dauer und Intensität der einzelnen \glspl{ereignis} sowie ein Histogramm über die Intensitätsverteilung.
\item Die Messstation ist fähig, mindestens zehn \glspl{sensor} zu betreiben und ihre Messsignale aufzuzeichnen.
\item Es ist möglich, die kompletten Rohdaten von einem \gls{sensor} über eine Dauer von 30 Minuten aufzuzeichnen. Während einer solchen Messung dürfen die anderen \glspl{sensor} ihre Messung einstellen.
\item Die \glspl{sensor} können über bis zu fünfzehn Meter im Bachbett verteilt sein.
\item Die Leistungsaufnahme der Anlage beim Betrieb von 10 \glspl{sensor} ist kleiner als zehn Watt.
\item Die Datenaufzeichnung erfolgt in einem eigens entwickelten \gls{logger}.
\item Am \gls{logger} kann ein Laptop angeschlossen werden, um Kontrollparameter der Messanlage zu setzen und um den Status der Anlage abzufragen.
\item Die erfassten Messdaten werden im \gls{logger} auf einer Speicherkarte gespeichert. Dies ermöglicht ein einfaches Abholen der Daten im Feld, indem die Speicherkarte ausgetauscht wird.
\end{itemize}
\subsection{Wunschkriterien}
\begin{itemize}
\item Die Anlage liefert für jedes \gls{ereignis} die Rohdaten in voller zeitlicher Auflösung.
\item Der Sensoraufbau ermöglicht es, die \glspl{sensor} in einer Elastomerplatte zu verpacken. Die Elastomerplatte kann ohne Betonkonstruktion im Bachbett verankert werden.
\item Am \gls{logger} kann ein Laptop angeschlossen werden, um die erfassten Messdaten herunterzuladen.
\end{itemize}
\subsection{Abgrenzungskriterien}
\begin{itemize}
\item Es würde den Rahmen dieser Arbeit sprengen, die Messeinheiten zur Produktreife zu bringen. Es wird lediglich aufgezeigt, wie solche Messeinheiten realisiert werden könnten.
\item Eine Testinstallation in einem Bach ist nicht möglich. Allenfalls kann ausserhalb der Bachelorarbeit in der Versuchsanstalt für Wasserbau, Hydrologie und Glaziologie der ETH Zürich ein kleiner Testlauf stattfinden.
\end{itemize}

