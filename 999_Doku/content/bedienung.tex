% !TeX spellcheck = de_CH
%%%%%%%%%%%%%%%%%%%%%%%%%%%%%%%%%%%%%%%%%%%%%%%%%%%%%%%%%%%%%%%%%
%  _____   ____  _____                                          %
% |_   _| /  __||  __ \    Institute of Computitional Physics   %
%   | |  |  /   | |__) |   Zuercher Hochschule Winterthur       %
%   | |  | (    |  ___/    (University of Applied Sciences)     %
%  _| |_ |  \__ | |        8401 Winterthur, Switzerland         %
% |_____| \____||_|                                             %
%%%%%%%%%%%%%%%%%%%%%%%%%%%%%%%%%%%%%%%%%%%%%%%%%%%%%%%%%%%%%%%%%
%
% Project     : BA Welti Keller
% Title       : 
% File        : bedienung.tex Rev. 00
% Date        : 15.09.2014
% Author      : Tobias Welti
%
%%%%%%%%%%%%%%%%%%%%%%%%%%%%%%%%%%%%%%%%%%%%%%%%%%%%%%%%%%%%%%%%%

\chapter{Bedienungsanleitung}\label{chap.bedienung}

\section{Produktbeschrieb}


\section{Aufbau der Messstation}
Stromversorgung, Verdrahtung, Can-Bus, Terminator, R2D2, C3PO

\section{Datenlogger}


\section{Sensor}


\section{Konfiguration}


\subsection{Menü}
\begin{lstlisting}[caption=, label=list.] 1) list files
 2) format SD card

 3) mount SD card
 4) unmount SD card
 5) logger status\n");
 6) start/stop logging

 7) sensor parameters
 8) sensor state
 9) reset timestamp
10) internal clock

11) config file
\end{lstlisting}


\subsubsection{Dateiliste}
\begin{lstlisting}[caption=, label=list.]HIER LISTE DER FILES EINFüGEN
 0) exit
\end{lstlisting}


\subsubsection{SD-Karte formatieren}
\begin{lstlisting}[caption=Untermenü SD-Karte formatieren, label=list.sdformat]
 1) confirm formatting of SD card.
    All data will be erased
 0) cancel
\end{lstlisting}

\begin{lstlisting}[caption=Statusmeldung SD formatieren, label=list.sdformatting]
formatting SD Card

\end{lstlisting}

\begin{lstlisting}[caption=Fehlermeldung SD formatieren, label=list.sdformatfail]
Formatting SD card FAILED. Please use a Computer to format the card.
\end{lstlisting}

\begin{lstlisting}[caption=Erfolgsmeldung SD formatieren, label=list.sdformatsuccess]
Formatting done
Returning to base menu.
\end{lstlisting}


\subsubsection{SD-Karte anmelden}
\begin{lstlisting}[caption=Untermenü SD-Karte anmelden, label=list.sdmount]

\end{lstlisting}


\subsubsection{SD-Karte abmelden}
\begin{lstlisting}[caption=Untermenü SD-Karte abmelden, label=list.sdunmount]

\end{lstlisting}


\subsubsection{Logger-Status}
\begin{lstlisting}[caption=Untermenü Logger-Status, label=list.loggerstatus]

\end{lstlisting}


\subsubsection{Starten und stoppen der Aufzeichnung}
\begin{lstlisting}[caption=Untermenü Starten und Stoppen der Aufzeichnung, label=list.startstop]
\end{lstlisting}


\subsubsection{Sensor-Parameter}
\begin{lstlisting}[caption=Untermenü Sensor-Auswahl, label=list.sensorsel]
 #) Select a sensor from the list.
99) Select all sensors.
 0) cancel
\end{lstlisting}

\begin{lstlisting}[caption=Untermenü Sensor-Parameter, label=list.sensorparam]
 1) set sampling rate (current: 10000 Hz)
 2) set threshold value (current: 200)
 3) set baseline value (current: 2047)
 4) set timeout (current: 30)
 5) set detail level (current: peaks only)
 6) start or stop recording (current: started)
 0) exit
\end{lstlisting}

\begin{lstlisting}[caption=Untermenü Abtastrate, label=list.paramfs]
 #) Enter sampling rate in Hz. (multiple of 100 Hz in range 100..200'000 Hz)
 0) cancel
\end{lstlisting}

Sampling rate 220000 Hz not supported, too high.

Sampling rate 0 Hz not supported.
To turn off sensor, set the detail level to 'off'
or use start/stop menu




\begin{lstlisting}[caption=Untermenü Threshold, label=list.paramthres]
 #) Enter threshold value.
    baseline + threshold must not exceed 4096
    and
    baseline - threshold must not be below 0
 0) cancel
\end{lstlisting}

\begin{lstlisting}[caption=Fehlermeldung ungültiger Threshold, label=list.paramthres]
		Invalid threshold value:
		threshold + baseline must not exceed 4096
		and
		threshold must be smaller than baseline value.
\end{lstlisting}

\begin{lstlisting}[caption=Untermenü Null-Level, label=list.parambase]
 #) Enter baseline value (default: 2047).
 0) cancel
\end{lstlisting}

\begin{lstlisting}[caption=Fehlermeldung ungültiger Null-Level, label=list.parambaseerror]
Invalid baseline value:\n");
threshold + baseline must not exceed 4096.
and
threshold - baseline must not be below 0 value
\end{lstlisting}

\begin{lstlisting}[caption=Untermenü Timeout, label=list.paramtimeout]
 #) Enter timeout in samples.
 0) cancel
\end{lstlisting}

\begin{lstlisting}[caption=Fehlermeldung zu langer Timeout, label=list.paramtimeoutlong]
Timeout too long, can not exceed 512.
\end{lstlisting}

\begin{lstlisting}[caption=Warnung kurzer Timeout, label=list.paramtimeoutshort]
Timeout 0 will end impact after each peak.
Timeout 0 in effect.
\end{lstlisting}

\begin{lstlisting}[caption=Untermenü Detail-Level, label=list.detail]
 1) raw (continuous data)
 2) detailed (start time, all samples of impact)
 3) peaks only (start time, nr of peaks, peaks
 4) sparse (only start time, duration, nr of peaks, max peak)
 5) off
 0) cancel
\end{lstlisting}

\begin{lstlisting}[caption=Untermenü Start/Stop einzeln, label=list.started_one]
Selected sensor is currently stopped.
 1) start
 2) stop
 0) cancel
\end{lstlisting}

\begin{lstlisting}[caption=Untermenü Start/Stop alle Sensoren, label=list.started_all]
At least one sensor is active.
 1) start
 2) stop
 0) cancel
 
-- oder --

No sensors are active.
 1) start
 2) stop
 0) cancel
\end{lstlisting}

\begin{lstlisting}[caption=Fehlermeldung beim Starten eines Sensors, label=list.sensorerror]
Could not create or open file. Please check SD card for free space.
\end{lstlisting}

\subsubsection{Sensor-Status}
\todo{Liste von Sensor-Stati ins Listing einfügen}
\begin{lstlisting}[caption=Untermenü Sensor-Status, label=list.sensorstatus]
Listing sensor config:
HIER EINFüGEN
 0) continue
\end{lstlisting}


\subsubsection{Timestamp zurücksetzen}
\begin{lstlisting}[caption=Untermenü Timestamp zurücksetzen, label=list.timestamp]
 1) re-synchronize timestamp
 0) cancel
\end{lstlisting}


\subsubsection{Interne Uhr}
\todo{Beispiel einer Uhrzeit einfügen}
\begin{lstlisting}[caption=Untermenü interne Uhr, label=list.intclock]
 1) adjust date
 2) adjust time
 0) exit

 current time: HIER EINFüGEN
\end{lstlisting}

\begin{lstlisting}[caption=Untermenü Datum einstellen, label=list.setdate]
adjust internal date
 1) adjust date +365 days
 2) adjust date -365 days
 3) adjust date + 30 days
 4) adjust date - 30 days
 5) adjust date + 10 days
 6) adjust date - 10 days
 7) adjust date +  1 day
 8) adjust date -  1 day
 0) exit

 current time: HIER EINFüGEN
\end{lstlisting}

\begin{lstlisting}[caption=Untermenü Uhrzeit einstellen, label=list.settime]
adjust internal time
 1) adjust time +1 hour
 2) adjust time -1 hour
 3) adjust time +10 minute
 4) adjust time -10 minute
 5) adjust time +1 minute
 6) adjust time -1 minute
 7) adjust time +1 second
 8) adjust time -1 second
 0) exit

 current time: HIER EINFüGEN
\end{lstlisting}

\subsubsection{Konfigurations-Datei}
\begin{lstlisting}[caption=Untermenü Konfigurationsdatei, label=list.config]
 1) read configuration from file and set up all sensors accordingly.
 2) store current configuration in file. Old config file will be overwritten.
 0) cancel
\end{lstlisting}

\begin{lstlisting}[caption=Fehlermeldung beim Speichern der Konfigurationsdatei, label=list.configerror]
The config file could not be written. Please check the SD card in a computer.
The configuration data will remain stored in the logger unless you turn off power.
\end{lstlisting}

\subsection{Konfigurationsdatei}
\begin{lstlisting}[caption=Beispiel einer Konfigurationsdatei, label=list.configfile]
{7,
{BLABLABLA}
{BLABLABLA}
{BLABLABLA}
}
\end{lstlisting}

\section{Betrieb}


\section{Technische Daten}