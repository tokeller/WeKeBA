%%%%%%%%%%%%%%%%%%%%%%%%%%%%%%%%%%%%%%%%%%%%%%%%%%%%%%%%%%%%%%%%%
%  _____   ____  _____                                          %
% |_   _| /  __||  __ \    Institute of Computitional Physics   %
%   | |  |  /   | |__) |   Zuercher Hochschule Winterthur       %
%   | |  | (    |  ___/    (University of Applied Sciences)     %
%  _| |_ |  \__ | |        8401 Winterthur, Switzerland         %
% |_____| \____||_|                                             %
%%%%%%%%%%%%%%%%%%%%%%%%%%%%%%%%%%%%%%%%%%%%%%%%%%%%%%%%%%%%%%%%%
%
% Project     : BA Welti Keller
% Title       : 
% File        : abstract.tex Rev. 00
% Date        : 15.09.2014
% Author      : Tobias Welti
%
%%%%%%%%%%%%%%%%%%%%%%%%%%%%%%%%%%%%%%%%%%%%%%%%%%%%%%%%%%%%%%%%%

\thispagestyle{empty}
\chapter*{Zusammenfassung}\label{chap.zusammenfassung}
Zurzeit kommen zur Messung von Geschiebebewegungen in Gewässern diverse Lösungen zur Anwendung. Bei der Eidg. Forschungsanstalt für Wald, Schnee und Landschaft (WSL) werden hauptsächlich Geophone verwendet, welche kostspielig sind und aufwändige bauliche Massnahmen sowie eine verhältnismässig komplexe IT-Infrastruktur bedingen. Das Erstellen eines Prototypen, der die Vereinfachung dieses Systems durch die Konzeption eines geeigneten Bussystems und der Verwendung kompakter MEMS-Beschleunigungssensoren zur Ereigniserkennung demonstriert, war das Hauptziel dieser Arbeit.

Als Basis für die im Einsatz befindlichen Geräte dienen Boards mit einem ARM Cortex-M4 Mikrocontroller. Das konzipierte System besteht im wesentlichen aus zwei Komponenten: einem Datenlogger und den Sensoreinheiten. Der Datenlogger zeichnet die Messwerte auf einer SD-Karte auf und übernimmt die Konfiguration der angehängten Sensoreinheiten. Im Unterschied zum normalen CAN-Bus muss der Absender einer Meldung für das Bus-Protokoll identifizierbar sein. Deshalb fungiert der Datenlogger auch als Busmaster und weist den Teilnehmern am CAN-Bus eindeutige Identifikationen zu. Die Sensoreinheiten bestehen aus den MEMS-Beschleunigungssensoren und dem Mikrocontroller, der die Messdaten je nach gewähltem Modus verarbeitet. Möglich ist dabei das Aufzeichnen der wichtigsten Kenndaten eines Ereignisses, die detailreiche Aufzeichnung eines Einschlages in zwei Detailstufen oder das Senden der Rohdaten eines Sensors. Durch die direkt im Sensor stattfindende Verarbeitung der Messdaten wird einerseits der Bus entlastet. Es werden nur die verarbeiteten Daten übermittelt. Andererseits fällt für den Datenlogger weniger Arbeit an, da dieser keine Sensordaten auswerten muss und nur noch die fertigen Daten speichert.

Da das ganze System unter anderem in Gebirgsbächen eingesetzt werden soll, musste ein besonderes Augenmerk auf die Stabilität und die Autarkie der Lösung gelegt werden. Wurde die Konfiguration des System einmal manuell vorgenommen, so können die Werte abgespeichert und bei einem Neustart des Systems automatisch wieder geladen werden. Die Verwendung eines CAN-Busses garantiert zudem die fehlerfreie Übertragung der Daten. 

Der erstellte Prototyp erfüllt die Erwartungen in Bezug auf die Vereinfachung des Aufbaus und den Verbrauch des ganzen Systems. Er könnte somit als Basis für ein finales Produkt dienen.



\newpage
\thispagestyle{empty}
\chapter*{Abstract}\label{abstract}
There are different ways to measure the transportation of bed load. The most common one, also used by the Swiss Federal Institute for Forest, Snow and Landscape Research (WSL), being the mounting of geophones in the river bed. There are several disadvantages to this, the main ones being the costs involved and the necessary structural measures. 

Additionally, the IT infrastructure becomes rather complex, since the geophones are connected by one cable each to an industrial, rugged PC. The goal of this project is to plan, design and implement a bus-system with MEMS-accelerometers and provide a prototype of a simpler and more cost-efficient measurement installation than the usage of geophones.

All used devices are based on the ARM Cortex-M4 microcontroller. The system implemented consists of two types of devices, the data logger and the sensor units. The data logger  receives and stores the processed data transmitted by the sensors. Additionally, the logger acts as a bus master and configures all the connected devices by sending them unique identifiers, thusly allowing for a proper assignment of the received messages to each sensor. The sensor units are composed of a MEMS-accelerometer and a microcontroller that processes the measurements from the MEMS, packaging the data depending on the detail level specified by the user, effectively reducing the traffic on the bus and the workload for the data logger. The detail level can be set dynamically by the user to either log the basic data of an impact, the detailed data of an impact (two levels of details are possible) or to send unprocessed raw data over a time range.

Since the whole system will be installed in rivers it should be rugged and self-sustaining. Once configured, the settings for each sensor can be stored on the SD card of the logger and, if a reset should occur, will be read automatically and sent to the sensors. The usage of a CAN-bus guarantees the error free transmission of the measured data.

The built prototype fulfills the expectations concerning the simplification of the system and the resource usage of the system and therefore could be used as a base for a final product.



