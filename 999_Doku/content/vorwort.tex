% !TeX spellcheck = de_CH
%%%%%%%%%%%%%%%%%%%%%%%%%%%%%%%%%%%%%%%%%%%%%%%%%%%%%%%%%%%%%%%%%
%  _____   ____  _____                                          %
% |_   _| /  __||  __ \    Institute of Computitional Physics   %
%   | |  |  /   | |__) |   Zuercher Hochschule Winterthur       %
%   | |  | (    |  ___/    (University of Applied Sciences)     %
%  _| |_ |  \__ | |        8401 Winterthur, Switzerland         %
% |_____| \____||_|                                             %
%%%%%%%%%%%%%%%%%%%%%%%%%%%%%%%%%%%%%%%%%%%%%%%%%%%%%%%%%%%%%%%%%
%
% Project     : BA Welti Keller
% Title       : 
% File        : vorwort.tex Rev. 00
% Date        : 15.09.2014
% Author      : Tobias Welti
%
%%%%%%%%%%%%%%%%%%%%%%%%%%%%%%%%%%%%%%%%%%%%%%%%%%%%%%%%%%%%%%%%%

\chapter*{Vorwort}\label{chap.vorwort}
Durch eine Studienkollegin kamen wir im Sommer 2013 mit Bruno Fritschi von der Eidgenössischen Forschungsanstalt für Wald, Schnee und Landschaft (WSL) in Kontakt und konnten uns von ihm einige äusserst interessante Ideen für Bachelorarbeiten im Bereich Embedded Systems vorstellen lassen. Ziemlich schnell waren wir uns einig, dass die Entwicklung eines Bussystems zur Messung von Geschiebetransport nach einer spannenden und herausfordernden Aufgabe tönt und entschlossen uns, diese als Bachelorarbeit anzugehen. Wie wir feststellen mussten, ist die Planung und Implementation eines ganzen Systems eine ziemliche Herausforderung, zumal wir beide bis jetzt nur wenig Erfahrung in diesem Bereich gesammelt hatten. Dank dem fachlichen Know-how von Bruno Fritschi und Hans Gelke konnten wir zum Schluss einen funktionierenden Prototyp erstellen, der zumindest die Machbarkeit dieser Lösung bestätigt. Es gibt aber durchaus noch einige Punkte, die man verbessern oder erweitern muss, bevor man von einem finalen Produkt sprechen kann.