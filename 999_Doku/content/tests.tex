% !TeX spellcheck = de_CH
%%%%%%%%%%%%%%%%%%%%%%%%%%%%%%%%%%%%%%%%%%%%%%%%%%%%%%%%%%%%%%%%%
%  _____   ____  _____                                          %
% |_   _| /  __||  __ \    Institute of Computitional Physics   %
%   | |  |  /   | |__) |   Zuercher Hochschule Winterthur       %
%   | |  | (    |  ___/    (University of Applied Sciences)     %
%  _| |_ |  \__ | |        8401 Winterthur, Switzerland         %
% |_____| \____||_|                                             %
%%%%%%%%%%%%%%%%%%%%%%%%%%%%%%%%%%%%%%%%%%%%%%%%%%%%%%%%%%%%%%%%%
%
% Project     : BA Welti Keller
% Title       : 
% File        : tests.tex Rev. 00
% Date        : 16.12.2014
% Author      : Tobias Welti
%
%%%%%%%%%%%%%%%%%%%%%%%%%%%%%%%%%%%%%%%%%%%%%%%%%%%%%%%%%%%%%%%%%


\chapter{Tests}\label{chap.tests}

Dieses Kapitel listet die Testfälle auf, die zu Beginn des Projekts definiert wurden. Für jeden Testfall sind, wo nötig, Vorbedingungen definiert. Ein Testfall beschreibt Aktionen, die vorgenommen werden müssen sowie das zu erreichende Resultat.

Die Testergebnisse sind am Ende jedes Testfalls angegeben.

\section{Testaufbau}
Die Verifikation der Sensoreinheiten macht einen Testaufbau nötig. Da das finale Produkt in einem Bach verbaut werden soll und die dort verwendeten Stahlplatten eine gewisse Grösse und aufgrund der Dicke von 15mm auch ein erhebliches Gewicht aufweisen, wurde zu Testzwecken ein kleinerer, einfacherer Aufbau realisiert. Statt einer Stahlplatte wurde eine Aluminiumplatte verwendet, was zur Folge hat dass bereits schwache Einschläge Signale generieren (Abbildung \ref{fig.testaufbau1}). Der \gls{sensor} ist unter der Aluminiumplatte verschraubt (Abbildung \ref{fig.testaufbau2}). Um eine präzise Messung zu ermöglichen, wurde der Sensor im Zentrum der Platte montiert. Da der verwendete Sensor nur auf einer Achse misst, musste er horizontal montiert werden, damit die Vibration entlang der Achse am grössten ist. Der Sensor wurde mit einem 5-poligen Flachbandkabel am Board angeschlossen, was der Übersicht im Gehäuse und der Einfachheit der Montage dienlich ist.  

\begin{figure}
	\centering
		\includegraphics[width=0.8\textwidth]{images/fotos/testaufbau1.jpg}
	\caption{Übersicht des Testaufbaus.}
	\label{fig.testaufbau1}
\end{figure}

\begin{figure}
	\centering
		\includegraphics[width=0.8\textwidth]{images/fotos/testaufbau2.jpg}
	\caption{Montage des Sensors im Testaufbau.}
	\label{fig.testaufbau2}
\end{figure}

\section{Datenlogger}
\subsection{T110 Busmaster}
\paragraph{Aktionen} Die Messstation wird an die Stromversorgung angeschlossen.

\paragraph{Resultate} Der \gls{logger} übernimmt die Kontrolle über den CAN-Bus und vergibt jedem angeschlossenen Sensor die eindeutige CAN-ID aus der Konfigurationsdatei.

\paragraph{Testergebnis} Die \glspl{sensoreinh} haben die vordefinierte CAN-ID erhalten. Der Test ist erfüllt.

\subsection{T120 Sensorerkennung}
\paragraph{Aktionen} Eine neue \gls{sensoreinh} wird an die Messstation angeschlossen, deren Seriennummer nicht in der Konfigurationsdatei erfasst ist. Die Messstation wird an die Stromversorgung angeschlossen.

\paragraph{Resultate} Der \gls{logger} erkennt die neue \gls{sensoreinh} und teilt ihr eine eindeutige CAN-ID zu.

\paragraph{Testergebnis} Die \gls{sensoreinh} hat eine bisher unbenutzte, unreservierte CAN-ID erhalten. Der Test ist erfüllt.

\subsection{T130 Uhrzeit}
\paragraph{Aktionen} Die Uhrzeit wird im laufenden Betrieb richtig eingestellt. Die Messstation wird während mind. 12~Stunden in Betrieb gehalten.

\paragraph{Resultate} Die interne Uhrzeit weist nicht mehr als 2~Sekunden Fehlgang innert 12~Stunden auf.

\paragraph{Testergebnis} Bis zur Berichterstellung konnte dieser Test noch nicht durchgeführt werden.

\subsection{T140 Timestamp zurücksetzen}
\paragraph{Aktionen} Über die Konfigurationsschnittstelle wird der Befehl zum Zurücksetzen der \gls{timestamp}s in den \glspl{sensoreinh} gegeben.

\paragraph{Resultate} Alle \glspl{sensoreinh} führen den Befehl innert weniger Sekunden aus und senden ab dann die Ereignisse mit dem neuen \gls{timestamp}.

\paragraph{Testergebnis} Bis zur Berichterstellung konnte dieser Test noch nicht durchgeführt werden.

\subsection{T160 Schnittstelle zum Steuerrechner}
\paragraph{Aktionen} Ein \gls{compi} wird an die Konfigurationsschnittstelle angeschlossen und ein \gls{terminalemu} gestartet.

\paragraph{Resultate} Das Konfigurationsmenü erscheint in der Terminal-Emulation und kann über Tastatureingaben bedient werden.

Die Eingaben werden vom \gls{logger} ausgeführt und bei Bedarf an die \glspl{sensoreinh} kommuniziert.

\paragraph{Testergebnis} Das Konfigurationsmenü funktioniert wie vorgesehen. Die Einstellungen werden an die \glspl{sensoreinh} übertragen.

\subsection{T170 Steuerung Detail-Level}
\paragraph{Aktionen} Am \gls{logger} wird für eine \gls{sensoreinh} ein anderer Detail-Level eingestellt.

\paragraph{Resultate} Die \gls{sensoreinh} wechselt den Detail-Level bei der Übertragung neuer Ereignisse.

\paragraph{Testergebnis} Bis zur Berichterstellung konnte dieser Test noch nicht durchgeführt werden.

\subsection{T180 Daten sammeln und speichern}
\paragraph{Aktionen} Mehrere \glspl{sensoreinh} senden Ereignisdaten an den \gls{logger}. 

\paragraph{Resultate} Der \gls{logger} legt für jede \gls{sensoreinh} eine Datei an und speichert die Ereignisdaten in der richtigen Datei.

\paragraph{Testergebnis} Bis zur Berichterstellung konnte dieser Test noch nicht durchgeführt werden.

\subsection{T190 Tokenvergabe}
\paragraph{Aktionen} Der \gls{logger} vergibt ein Token an eine \gls{sensoreinh}. Nach erhaltenen zehn Nachrichten verfällt der Token. Der \gls{logger} vergibt den Token der nächsten \gls{sensoreinh}.

\paragraph{Resultate} Token wird korrekt weitergegeben. Es sendet nur jene \gls{sensoreinh} Nachrichten, die den Token hält. Die andere \gls{sensoreinh} hält die Ereignisdaten so lange im Zwischenspeicher, bis sie den Token erhält.

\paragraph{Testergebnis} Das Übermittlungsprotokoll des \gls{logger}s und der \glspl{sensoreinh} ist in Listing \ref{t190} aufgeführt und zeigt den korrekten Ablauf des Tests.

\begin{lstlisting}[caption=T190 Tokenvergabe, label=t190]
//Logger                           //Sensor02
// sent Token to Sensor02          // received token
prepare message id: 4020101        Timestamp id  : 2ff0101
OK                                 Token received, start sending 10 msgs

                                   //Sensor03
                                   // no token, stops after timesync
                                   Timestamp id  : 2ff0101

//Logger                           //Sensor02
// receive 10 msgs from Sensor02   // send 10 packages to logger
Sensor data: 055f7cf52602e         Impact complete:
Sensor id  : 1c010201              Starttime: 5633999
Sensor len : 8                     Samples: 46
...                                Peaks: 6
...                                 Maximum: 594
...                                ***********
...                                sent payload 055f7cf52602e
...                                OK, Message sent to logger
...                                
...                                Impact complete:
...                                Starttime: 5829023
...                                Samples: 36
...                                Peaks: 5
...                                Maximum: 260
...                                ***********
...                                sent payload 058f19f45024
Sensor data: 058f19f45024          OK, Message sent to logger
Sensor id  : 1c010201              Token revoked
Sensor len : 8                     
//Logger                           //Sensor03
give token to SensorID 3 of 2      Timestamp id  : 2ff0101
prepare message id: 4030101        Token received, start sending 10 msgs
OK
                                   Impact complete:
                                   Starttime: 11874290
                                   Samples: 1
                                   Peaks: 1
                                   Maximum: 410
                                   ***********
                                   sent payload 0b52ff29a101
Sensor data: 0b52ff29a101          OK, Message sent to logger
Sensor id  : 1c010301              Token revoked
Sensor len : 8                     
                                   Impact complete:
give token to SensorID 2 of 2      Starttime: 12065974
prepare message id: 4020101        Samples: 25
                                   Peaks: 6
                                   Maximum: 334
                                   ***********
                                   sent payload 0b81cb64e6019
                                   No token, message stored
\end{lstlisting}


\section{Sensoreinheit}
\subsection{T400 Konfiguration}
\paragraph{Aktionen} Via Konfigurationsmenü des Datenloggers werden die Parameter des Sensors auf neue Werte gesetzt. 

\paragraph{Resultate} Die \gls{sensoreinh} übernimmt die neuen Einstellungen und passt die Ereigniserkennung entsprechend an.

\paragraph{Testergebnis} Die Konfiguration wurde erfolgreich auf die \gls{sensoreinh} übertragen. Die Konfiguration vor dem Test ist in Listing \ref{t400.1} gegeben, mit der Eingabe 99 werden alle Sensoren ausgewählt. Listings \ref{t400.2} und \ref{t400.3} zeigen, wie der zu konfigurierende Parameter ausgewählt und der neue Wert eingegeben wird. Am \gls{logger} wird im Testmodus das Absenden einer CAN-Nachricht protokolliert, Listing \ref{t400.4}. Die \glspl{sensoreinh} geben im Testmodus ebenfalls eine Bestätigung aus (Listings \ref{t400.5} und \ref{t400.6}), wenn sie eine neue Konfiguration erhalten. Die Änderung des \gls{threshold}s wurde also von beiden Sensoren übernommen.

\begin{lstlisting}[caption=T400 Vorbedingung und Auswahl aller Sensoren, label=t400.1]
Listing sensor config
 Nr  SID  serial    fs  threshold baseline timeout detail     started?
 0)  2  061bfdf6   100        200     2047      30 sparse     started
 1)  3  15117738   100        150     2040      30 peaks only started
 #) Select a sensor from the list.
99) Select all sensors.
 0) cancel
 >
your entry was: 99
\end{lstlisting}

\begin{lstlisting}[caption=T400 Auswahl des Parameters, label=t400.2]
 1) set sampling rate
 2) set threshold value
 3) set baseline value
 4) set timeout
 5) set detail level
 6) start or stop recording
 0) exit
 >
your entry was: 2
\end{lstlisting}

\begin{lstlisting}[caption=T400 Eingabe des neuen \gls{threshold}s, label=t400.3]
 #) Enter threshold value.
baseline + threshold must not exceed 4096
and
baseline - threshold must not be below 0
 0) cancel
 >
your entry was: 250
\end{lstlisting}

\begin{lstlisting}[caption=T400 Sendeprotokoll am \gls{logger}, label=t400.4]
prepare message id: 5020101
OK
prepare message id: 50301001
OK
\end{lstlisting}

\begin{lstlisting}[caption=T400 Konfiguration aus Sensor 3, label=t400.5]
threshold fa   -> geaenderter Wert, von 150 (0x96) auf 250 (0xfa)
fsampling 64   -> wie zuvor, bei 100
timeout 1e     -> wie zuvor, bei 30
baseline 7f8   -> wie zuvor, bei 2040
\end{lstlisting}

\begin{lstlisting}[caption=T400 Konfiguration aus Sensor 2, label=t400.6]
threshold fa   -> geaenderter Wert, von 200 (0xc8) auf 250 (0xfa)
fsampling 64   -> wie zuvor, bei 100
timeout 1e     -> wie zuvor, bei 30
baseline 7ff   -> wie zuvor, bei 2047
\end{lstlisting}


\subsection{T410 Ereignisdetektion}
\paragraph{Aktionen} Die \gls{sensoreinh} ist im Messbetrieb. Am \gls{sensor} wird ein Einschlag simuliert, indem ein Objekt auf die Metallplatte geschlagen wird. 

\paragraph{Resultate} Am Ausgang des \gls{sensor}s misst ein Oszilloskop den Spannungsverlauf. Die so gemessene Referenzkurve wird mit den von der \gls{sensoreinh} ausgegebenen Daten verglichen.

\paragraph{Testergebnis} Bis zur Berichterstellung konnte dieser Test noch nicht durchgeführt werden.

\subsection{T430 Datenübertragung}
\paragraph{Aktionen} Mehrere Einschläge werden simuliert und mit einem Oszilloskop aufgezeichnet, um Referenzdaten zu haben.

\paragraph{Resultate} Die \gls{sensoreinh} überträgt die Daten von Test T410 fehlerfrei an den \gls{logger}.

\paragraph{Testergebnis} Bis zur Berichterstellung konnte dieser Test noch nicht durchgeführt werden.

\subsection{T450 Rohdatenaufzeichnung}
\paragraph{Aktionen} Eine \gls{sensoreinh} wird in den Rohdatenmodus versetzt.

\paragraph{Resultate} Die \gls{sensoreinh} zeichnet Rohdaten während 60 Sekunden ohne einen Puffer-Überlauf auf und überträgt die Daten an den \gls{logger}.

\paragraph{Testergebnis} Bis zur Berichterstellung konnte dieser Test noch nicht durchgeführt werden.

\section{Nichtfunktionale Tests}
\subsection{T710 Stromverbrauch}
\paragraph{Aktionen} Der Stromverbrauch der Messanlage mit dem \gls{logger} und drei \glspl{sensoreinh} wird gemessen. 

\paragraph{Resultate} Der Stromverbrauch sollte unter 3~Watt liegen.

\paragraph{Testergebnis} Ein \gls{logger} und zwei \gls{sensoreinh} haben einen Stromverbrauch von 