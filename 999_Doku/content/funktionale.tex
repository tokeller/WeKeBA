% !TeX spellcheck = de_CH
% !TeX spellcheck = de_CH
%%%%%%%%%%%%%%%%%%%%%%%%%%%%%%%%%%%%%%%%%%%%%%%%%%%%%%%%%%%%%%%%%
%  _____   ____  _____                                          %
% |_   _| /  __||  __ \    Institute of Computational Physics   %
%   | |  |  /   | |__) |   Zuercher Hochschule Winterthur       %
%   | |  | (    |  ___/    (University of Applied Sciences)     %
%  _| |_ |  \__ | |        8401 Winterthur, Switzerland         %
% |_____| \____||_|                                             %
%%%%%%%%%%%%%%%%%%%%%%%%%%%%%%%%%%%%%%%%%%%%%%%%%%%%%%%%%%%%%%%%%
%
% Project     : BA Welti Keller
% Title       : 
% File        : funktionale.tex Rev. 00
% Date        : 15.09.2014
% Author      : Tobias Welti
%
%%%%%%%%%%%%%%%%%%%%%%%%%%%%%%%%%%%%%%%%%%%%%%%%%%%%%%%%%%%%%%%%%

\thispagestyle{empty}
\chapter{Funktionale Anforderungen}\label{chap.funktionale}
\section{\gls{logger} (F1\ldots)}


\subsection{F110 Busmaster}
Der \gls{logger} übernimmt die Kontrolle des \gls{bussys}. Bei Inbetriebnahme des Systems tastet der \gls{logger} den Bus nach \glspl{sensoreinh} ab und erteilt jeder \gls{sensoreinh} eine eindeutige Identifikationsnummer (ID). Die ID des \gls{logger}s soll so gewählt werden, dass er jederzeit Priorität hat, auf den Bus zu schreiben.


\subsection{F120 Sensorerkennung}
Die angeschlossenen \glspl{sensor} werden vom \gls{logger} erkannt und mit einer ID versehen. Anhand der ID wird die Priorität bei der Datenübertragung festgelegt und der \gls{sensor} identifiziert. Ein Sensor, der bereits am System angeschlossen war, erhält wieder die gleiche \gls{id}, sofern die Konfigurationsdatei nicht gelöscht wurde.


\subsection{F130 Uhrzeit}
Der \gls{logger} verfügt über eine interne Uhr, um die \glspl{ereignis} in den Dateien mit einem lesbaren Zeitstempel zu versehen.


\subsection{F140 \gls{timestamp} verteilen}
Der \gls{logger} sendet ein Signal an alle \glspl{sensoreinh}, dass der Zeitstempel (\gls{timestamp}) neu gestellt werden soll. Ab dann beziehen sich die \gls{timestamp}s auf die Dauer seit dem jetzigen Zeitpunkt.


\subsection{F160 Schnittstelle zum Steuerrechner}
Der \gls{logger} bietet eine Schnittstelle, an der ein Steuerrechner (Laptop, \gls{pc}) angeschlossen werden kann. Über diese Schnittstelle kann der Betrieb der ganzen Anlage gesteuert werden.


\subsection{F170 Steuerung Betriebsmodus}
Der Betriebsmodus der \glspl{sensor} wird vom \gls{logger} aus gesteuert: Wie viele und welche Art von Daten gesammelt werden soll und ob alle \glspl{sensor} oder nur bestimmte aktiv sein sollen. \\
Folgende Betreibsmodi sind verfügbar:
\begin{itemize}
\item Normaler Modus: Alle \glspl{sensor} übermitteln die verarbeiteten Ereignisdaten. Zeitpunkt, Intensität, Dauer und Anzahl Ausschläge jedes \gls{ereignis} werden gespeichert.
\item Detaillierter Modus: Alle \glspl{sensor} übermitteln die verarbeiteten Ereignisdaten sowie die gesamten Messdaten für die Dauer des \gls{ereignis}.
\item Rohdatenmodus: Ein \gls{sensor} übermittelt kontinuierlich Rohdaten, die anderen \glspl{sensor} werden vorübergehend abgeschaltet.
\end{itemize}


\subsection{F180 Daten sammeln}
Der \gls{logger} fragt in regelmässigen Abständen bei den \glspl{sensoreinh} an, ob Ereignisdaten zur Übertragung bereit sind. Diese übermitteln die vorliegenden Ereignisdaten.


\subsection{F190 Daten speichern}
Die Daten werden vom \gls{logger} auf einer Speicherkarte in Dateien abgelegt. Nach entsprechenden Befehlen vom Steuerrechner kann die Karte entfernt und ausgetauscht werden, um die Daten abzuholen.


\section{\gls{sensoreinh} (F4\ldots)}


\subsection{F410 Ereignisdetektion}
Die \gls{sensoreinh} liest den \gls{sensor} mit einer definierten \gls{fs} aus und wertet die Messdaten aus. Der Prozessor erkennt \glspl{ereignis} anhand definierter Kriterien. Zu jedem \gls{ereignis} werden folgende Daten gespeichert: Zeitpunkt (\gls{timestamp}), Dauer, Anzahl Peaks und höchster Peak. In einem zweiten Betriebsmodus können alle Messpunkte während einem \gls{ereignis} gespeichert werden.


\subsection{F430 Datenübertragung}
Die \gls{sensoreinh} übermittelt die Ereignisdaten über das \gls{bussys} an den \gls{logger}.


\subsection{F450 Rohdatenaufzeichnung}
In einem Sondermodus werden alle Messpunkte gespeichert und über das \gls{bussys} an den \gls{logger} übertragen. In diesem Betriebsmodus kann darf auch nur eine \gls{sensoreinh} aktiv sein, die anderen werden auf Standby geschaltet.


