% !TeX spellcheck = de_CH
% !TeX spellcheck = de_CH
%%%%%%%%%%%%%%%%%%%%%%%%%%%%%%%%%%%%%%%%%%%%%%%%%%%%%%%%%%%%%%%%%
%  _____   ____  _____                                          %
% |_   _| /  __||  __ \    Institute of Computational Physics   %
%   | |  |  /   | |__) |   Zuercher Hochschule Winterthur       %
%   | |  | (    |  ___/    (University of Applied Sciences)     %
%  _| |_ |  \__ | |        8401 Winterthur, Switzerland         %
% |_____| \____||_|                                             %
%%%%%%%%%%%%%%%%%%%%%%%%%%%%%%%%%%%%%%%%%%%%%%%%%%%%%%%%%%%%%%%%%
%
% Project     : BA Welti Keller
% Title       : 
% File        : funktionale.tex Rev. 00
% Date        : 15.09.2014
% Author      : Tobias Welti
%
%%%%%%%%%%%%%%%%%%%%%%%%%%%%%%%%%%%%%%%%%%%%%%%%%%%%%%%%%%%%%%%%%

\thispagestyle{empty}
\chapter{Funktionale Anforderungen}\label{chap.funktionale}
\section{Datenlogger (F1\ldots)}


\subsection{F110 Busmaster}
Der \gls{logger} übernimmt die Kontrolle des Bus. Bei Inbetriebnahme des Systems tastet der \gls{logger} den Bus nach Sensoreinheiten ab und erteilt jeder Sensoreinheit eine eindeutige Identifikationsnummer (ID). Die ID des \gls{logger}s soll so gewählt werden, dass er jederzeit Priorität hat, auf den Bus zu schreiben.


\subsection{F120 Sensorerkennung}
Die angeschlossenen Sensoren werden vom \gls{logger} erkannt und mit einer ID versehen. Anhand der ID wird die Priorität bei der Datenübertragung festgelegt und der Sensor identifiziert.\\
Können wir die Seriennummer des Boards auslesen? Damit die ID immer gleich ist... über UART-Kommandozeile muss die ID jedes Sensors gesetzt werden können.


\subsection{F130 Uhrzeit}
Der \gls{logger} verfügt über eine interne Uhr, um die Ereignisse in den Dateien mit einem lesbaren Zeitstempel zu versehen.


\subsection{F140 Timestamp verteilen}
Der \gls{logger} sendet ein Signal an alle Sensoreinheiten, dass der Zeitstempel (Timestamp) neu gestellt werden soll. Ab dann beziehen sich die Timestamps auf die Dauer seit dem jetzigen Zeitpunkt.


\subsection{F160 Schnittstelle zum Steuerrechner}
Der \gls{logger} bietet eine Schnittstelle, an der ein Steuerrechner (Laptop, PC) angeschlossen werden kann. Über diese Schnittstelle kann der Betrieb der ganzen Anlage gesteuert werden.


\subsection{F170 Steuerung Betriebsmodus}
Der Betriebsmodus der Sensoren wird vom \gls{logger} aus gesteuert: Wie viele und welche Art von Daten gesammelt werden soll und ob alle Sensoren oder nur bestimmte aktiv sein sollen. \\
Folgende Betreibsmodi sind verfügbar:
\begin{itemize}
\item Normaler Modus: Alle Sensoren übermitteln die verarbeiteten Ereignisdaten. Zeitpunkt, Intensität, Dauer und Anzahl Ausschläge jedes Ereignis werden gespeichert.
\item Detaillierter Modus: Alle Sensoren übermitteln die verarbeiteten Ereignisdaten sowie die gesamten Messdaten für die Dauer des Ereignis.
\item Rohdatenmodus: Ein Sensor übermittelt kontinuierlich Rohdaten, die anderen Sensoren werden vorübergehend abgeschaltet.
\end{itemize}


\subsection{F180 Daten sammeln}
Der \gls{logger} fragt in regelmässigen Abständen bei den Sensoreinheiten an, ob Ereignisdaten zur Übertragung bereit sind. Diese übermitteln die vorliegenden Ereignisdaten.


\subsection{F190 Daten speichern}
Die Daten werden vom \gls{logger} auf einer Speicherkarte in Dateien abgelegt. Nach entsprechenden Befehlen vom Steuerrechner kann die Karte entfernt und ausgetauscht werden, um die Daten abzuholen.


\section{Sensoreinheit (F4\ldots)}


\subsection{F410 Ereignisdetektion}
Die Sensoreinheit liest den Sensor mit einer definierten Abtastrate (Samplingrate) aus und wertet die Messdaten aus. Der Prozessor erkennt Ereignisse anhand definierter Kriterien. Zu jedem Ereignis werden folgende Daten gespeichert: Zeitpunkt (Timestamp), Dauer, Anzahl Peaks und höchster Peak. In einem zweiten Betriebsmodus können alle Messpunkte während einem Ereignis gespeichert werden.


\subsection{F430 Datenübertragung}
Die Sensoreinheit übermittelt die Ereignisdaten über das \gls{bussys} an den \gls{logger}.


\subsection{F450 Rohdatenaufzeichnung}
In einem Sondermodus werden alle Messpunkte gespeichert und über das \gls{bussys} an den \gls{logger} übertragen. In diesem Betriebsmodus kann nur eine Sensoreinheit aktiv sein, die anderen werden auf Standby geschaltet.


