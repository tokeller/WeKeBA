% !TeX spellcheck = de_CH
%%%%%%%%%%%%%%%%%%%%%%%%%%%%%%%%%%%%%%%%%%%%%%%%%%%%%%%%%%%%%%%%%
%  _____   ____  _____                                          %
% |_   _| /  __||  __ \    Institute of Computitional Physics   %
%   | |  |  /   | |__) |   Zuercher Hochschule Winterthur       %
%   | |  | (    |  ___/    (University of Applied Sciences)     %
%  _| |_ |  \__ | |        8401 Winterthur, Switzerland         %
% |_____| \____||_|                                             %
%%%%%%%%%%%%%%%%%%%%%%%%%%%%%%%%%%%%%%%%%%%%%%%%%%%%%%%%%%%%%%%%%
%
% Project     : BA Welti Keller
% Title       : 
% File        : nichtfunktionale.tex Rev. 00
% Date        : 15.09.2014
% Author      : Tobias Welti
%
%%%%%%%%%%%%%%%%%%%%%%%%%%%%%%%%%%%%%%%%%%%%%%%%%%%%%%%%%%%%%%%%%

\thispagestyle{empty}
\chapter{Nichtfunktionale Anforderungen}\label{chap.nichtfunktionale}

\begin{itemize}
\item Die gesamte Messstation soll eine geringere Leistungsaufnahme haben als eine aktuelle Messstation mit \glspl{geophon}n. Für zehn \glspl{geophon} sind dies zur Zeit ungefähr zehn Watt.

\item Die Installation soll weniger bauliche Massnahmen erfordern als eine aktuelle Messstation mit \gls{geophon}en.

\item Die erfassten Ereignisdaten sollen mindestens so detailliert sein wie von den bisherigen Installationen.

\item \glspl{sensoreinh} müssen wasserdicht verpackt werden können.
\
\end{itemize}