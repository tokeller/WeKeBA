% !TeX spellcheck = de_CH
%%%%%%%%%%%%%%%%%%%%%%%%%%%%%%%%%%%%%%%%%%%%%%%%%%%%%%%%%%%%%%%%%
%  _____   ____  _____                                          %
% |_   _| /  __||  __ \    Institute of Computitional Physics   %
%   | |  |  /   | |__) |   Zuercher Hochschule Winterthur       %
%   | |  | (    |  ___/    (University of Applied Sciences)     %
%  _| |_ |  \__ | |        8401 Winterthur, Switzerland         %
% |_____| \____||_|                                             %
%%%%%%%%%%%%%%%%%%%%%%%%%%%%%%%%%%%%%%%%%%%%%%%%%%%%%%%%%%%%%%%%%
%
% Project     : BA Welti Keller
% Title       : 
% File        : nichtfunktionale.tex Rev. 00
% Date        : 15.09.2014
% Author      : Tobias Welti
%
%%%%%%%%%%%%%%%%%%%%%%%%%%%%%%%%%%%%%%%%%%%%%%%%%%%%%%%%%%%%%%%%%

\thispagestyle{empty}
\chapter{Nichtfunktionale Anforderungen}\label{chap.nichtfunktionale}

\begin{itemize}
\item Die gesamte Messstation soll eine geringere Leistungsaufnahme haben als eine aktuelle Messstation mit Geophonen. Für zehn Geophone sind dies zur Zeit ungefähr zehn Watt.

\item Die Installation soll weniger bauliche Massnahmen erfordern als eine aktuelle Messstation mit Geophonen.

\item Die erfassten Ereignisdaten sollen mehr Details enthalten als mit den bisherigen Installationen.

\item Sensoreinheiten müssen wasserdicht verpackt werden können.
\
\end{itemize}