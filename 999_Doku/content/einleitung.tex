% !TeX spellcheck = de_CH
%%%%%%%%%%%%%%%%%%%%%%%%%%%%%%%%%%%%%%%%%%%%%%%%%%%%%%%%%%%%%%%%%
%  _____   ____  _____                                          %
% |_   _| /  __||  __ \    Institute of Computitional Physics   %
%   | |  |  /   | |__) |   Zuercher Hochschule Winterthur       %
%   | |  | (    |  ___/    (University of Applied Sciences)     %
%  _| |_ |  \__ | |        8401 Winterthur, Switzerland         %
% |_____| \____||_|                                             %
%%%%%%%%%%%%%%%%%%%%%%%%%%%%%%%%%%%%%%%%%%%%%%%%%%%%%%%%%%%%%%%%%
%
% Project     : BA Welti Keller
% Title       : 
% File        : einleitung.tex Rev. 00
% Date        : 15.09.2014
% Author      : Tobias Welti
%
%%%%%%%%%%%%%%%%%%%%%%%%%%%%%%%%%%%%%%%%%%%%%%%%%%%%%%%%%%%%%%%%%



\chapter{Einleitung}\label{chap.einleitung}

\section{Ausgangslage}\label{sec.ausgangslage}
Die \gls{wsl} betreibt Messstationen zur Registrierung von Geschiebe-Bewegungen im Fluss mittels \glspl{geophon}n, die unter Stahlplatten montiert sind. Diese Platten sind in einer Betonkonstruktion eingelassen, um sie im Flussbett zu fixieren. Die Betonkonstruktion dient gleichzeitig als Kabelkanal. Jedes \glspl{geophon} ist über ein Kabel mit einem Auswertungs-Rechner (Embedded PC) verbunden. Der Rechner wertet die Signale aller angeschlossen \glspl{geophon} kontinuierlich aus, um die Ereignisse zu detektieren. Bei mehreren \glspl{geophon}n ist hier ein recht leistungsfähiger Rechner nötig, der eine entsprechend hohe Leistungsaufnahme hat. Die baulichen Massnahmen für die Installation der \glspl{geophon}, der Auswertungsstation sowie der Stromversorgung sind sehr teuer. Da viele dieser Messstationen in Gebirgsbächen installiert werden, fallen hohe Transportkosten für die schweren Materialien an.

\subsection{Hydrologische Messungen}
Diese Messstationen benutzt das \gls{wsl}, um Daten über Geschiebetransport zu sammeln. Solche Datensammlungen spielen eine zentrale Rolle im Flussbau und der \gls{Hydrologie}. Die Vorhersage von Menge und Zeitpunkt des Auftretens von Geschiebetransport ist weiterhin schwierig, weshalb intensiv auf diesem Gebiet geforscht wird. Forschungsgruppen aus der ganzen Welt haben unterschiedlichste Sensoren und Messinstallationen entwickelt auf der Suche nach einem Instrument, mehr und bessere Daten sammeln zu können. Dabei kommen Piezoelektische Sensoren und Geophone zum Einsatz \cite{rickenmann2012ESPL}, aber auch Beschleunigungssensoren \cite{reid2007}. Hydrophone können ebenfalls eingesetzt werden, um die Geräusche der Geschiebekörner unter Wasser aufzuzeichnen. Dies erfordert aber eine viel höhere Abtastrate und daher mehr Rechenleistung \cite{wyss2014}. Um die Daten in Korrelation zu Menge, Korngrösse und Fliessgeschwindigkeit setzen zu können, gibt es Messanlagen mit weiteren Sensoren und Fangkörben (z.B. im Erlenbach im Alptal, Kanton Schwyz), die bei hohem Geschiebeaufkommen automatisch ins Flussbett gefahren werden, um das Geschiebe aufzufangen \cite{rickenmann2014ESPL}. Auch Fangnetze werden zu diesem Zweck installiert \cite{rickenmann2014ESPL}. 

Die hydrologische Forschung verfolgt unter Anderem das Ziel, bessere Vorhersagen zu Naturereignissen machen zu können. Stauungen in einem Bach durch Geschiebeablagerungen können zu drastischen Verschlimmerungen führen, z.B. Ausuferungen bei einem Hochwasser \cite{turowski2008}.

\section{Projektidee}
Bruno Fritschi (\gls{wsl}) ist bestrebt, die momentan sehr aufwändigen baulichen Massnahmen zu vereinfachen und den Sensoraufbau so zu modifizieren, dass die Messstation insgesamt einfacher aufgebaut werden kann.

Die wichtigste Idee sieht vor, die gemessenen Daten direkt am Sensor auszuwerten, und nur jene Daten zu übertragen und zu speichern, an denen die Forschung interessiert ist. Die Reduktion der Datenmenge an der Quelle ermöglicht die Verwendung eines Bussystems, also eines einzigen Kabels für alle Sensoren, statt eines eigenen Kabels für jeden Sensor. Da die Daten über das Bussystem aus dem Bach heraus auf einen Datensammler (\gls{logger}) übertragen werden können, würde dies weitere Möglichkeiten eröffnen:
\begin{itemize}
\item Stromversorgung der Sensoren über das Kabel des Bussystems. Dies würde die Laufzeit einer solchen Anlage beträchtlich erhöhen, resp. Batterien in den Sensoren unnötig machen.
\item Datensammlung an einem zugänglichen Gerät, statt Einsammeln der Sensoren aus dem Bach.
\item Überwachung der Anlage zu Laufzeit, Kontrolle über den Zustand der Datensammlung und Änderung der Konfiguration der Sensoren im Betrieb.
\end{itemize}

Zukünftig sollen die Geophone durch ein- oder mehr-achsige \gls{mems}-Beschleunigungssensoren ersetzt werden, da diese wesentlich kleiner sind. Die Verkleinerung der Sensoren ermöglicht neue Bauformen der Messanlagen:

Die Integration der Sensoren in einer Elastomerplatte statt der bisherigen Montage unter Stahlplatten würde die baulichen Massnahmen und damit die Kosten drastisch senken. Die Verkabelung eines \gls{bussys}s ist weit weniger Komplex als für ein sternförmiges System wie bisher. Dies würde die Variante mit Sensoren in Elastomerplatten begünstigen.

\section{Ziel}
In diesem Projekt soll daher ein Messsystem entwickelt werden, das \gls{mems}-Beschleunigungs-\glspl{sensor} in einem Bussystem einsetzt und die Signale direkt beim Sensor auswertet. Die Entwicklung zur Serienreife würde den Rahmen einer Bachelorarbeit für zwei Personen sprengen, weshalb nur ein Prototyp entwickelt werden soll. Anhand des Prototyps soll an der \gls{wsl} untersucht werden können, ob sich die Weiterentwicklung lohnt. 

Bei einem positiven Entscheid könnte in einer weiteren Bachelorarbeit die Weiterentwicklung zu einem Serienprodukt versucht werden.

