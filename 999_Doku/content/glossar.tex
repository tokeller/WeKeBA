% !TeX spellcheck = de_CH
%%%%%%%%%%%%%%%%%%%%%%%%%%%%%%%%%%%%%%%%%%%%%%%%%%%%%%%%%%%%%%%%%
%  _____   ____  _____                                          %
% |_   _| /  __||  __ \    Institute of Computitional Physics   %
%   | |  |  /   | |__) |   Zuercher Hochschule Winterthur       %
%   | |  | (    |  ___/    (University of Applied Sciences)     %
%  _| |_ |  \__ | |        8401 Winterthur, Switzerland         %
% |_____| \____||_|                                             %
%%%%%%%%%%%%%%%%%%%%%%%%%%%%%%%%%%%%%%%%%%%%%%%%%%%%%%%%%%%%%%%%%
%
% Project     : BA Welti Keller
% Title       : 
% File        : glossar.tex Rev. 00
% Date        : 15.09.2014
% Author      : Tobias Welti
%
%%%%%%%%%%%%%%%%%%%%%%%%%%%%%%%%%%%%%%%%%%%%%%%%%%%%%%%%%%%%%%%%%

% Abkürzungen
\newacronym{wsl}{WSL}{Eidg. Forschungsanstalt für Wald, Schnee und Landschaft}

\newacronym{vaw}{VAW}{Versuchsanlage für Wasserbau}

\newacronym{id}{ID}{Identifikationsnummer}

\newacronym{pc}{PC}{Personal Computer}

\newacronym{cpu}{CPU}{Central Processing Unit}

\newacronym{cd}{CD}{Compact Disc}

\newacronym{adc}{ADC}{Analog Digital Converter}

\newacronym{dft}{DFT}{dis\-kre\-te Fou\-rier-Trans\-for\-ma\-tion}

\newacronym{idft}{IDFT}{inverse diskrete Fourier-Transformation}

\newacronym{fsm}{FSM}{Finite State Machine}

\newacronym{dsp}{DSP}{Digitaler Signal-Prozessor}

\newacronym{nvic}{NVIC}{Nested Vectored Interrupt Controller}

\newacronym{mac}{MAC}{multiply-accumulate unit}

\newacronym{fpu}{FPU}{Floating Point Unit}

\newacronym{isr}{ISR}{Interrupt Service Routine}

\newacronym{cdet}{CD}{Collision Detection}

\newacronym{mcacr}{MC}{Microkontroller}

\newacronym{crc}{CRC}{Cyclic Redundancy Check}

\newacronym{usb}{USB}{Universal Serial Bus}

\newacronym{mci}{MCI}{Memory Card Interface}

\newacronym{irq}{IRQ}{Interrupt Request}

% Glossareinträge
\newglossaryentry{geophon}{name={Geophon},plural={Geophone},description={Ein Messgerät für Vibrationen des Bodens. Ein Geophon misst Bewegungen mittels einer magnetischen Masse, die beweglich in einer Spule aufgehängt ist. Wird das Geophon in Bewegung versetzt, schwingt die magnetische Masse aufgrund ihrer Trägheit und induziert dadurch einen Strom in der Spule. Durch Messung dieses Stroms kann die Bewegung registriert werden.}}


\newglossaryentry{logger}{name={Datenlogger},plural={Datenlogger},description={Ein Gerät zur Sammlung und Speicherung von Messdaten von mehreren Sensoreinheiten.}, see={sensoreinh}}

\newglossaryentry{speichermedium}{name={Speichermedium},plural={Speichermedien},description={Ein Stück Hardware, auf dem Daten gespeichert werden können.}}

\newglossaryentry{sensoreinh}{name={Sensoreinheit},plural={Sensoreinheiten},description={Ein kombiniertes elektronisches Gerät zur Messung von physikalischen Daten und der Verarbeitung dieser Daten. Das Gerät verfügt über einen Mikroprozessor und einen Sensor. Optional kann auch eine Schnittstelle für die Kommunikation mit einem Datenlogger vorhanden sein.},see={mc,sensor,logger}}

\newglossaryentry{sensor}{name={Sensor},plural={Sensoren},description={Ein Messgerät für physikalische Grössen wie Temperatur, Feuchtigkeit, Luftdruck oder Beschleunigung.}}

\newglossaryentry{bussys}{name={Bussystem},plural={Bussysteme},description={Ein elektrisches System für die Kommunikation zwischen mehreren Geräten. Ein Bussystem besteht aus Datenleitungen, über welche Signale gesendet werden, und aus Schnittstellen, an denen die Busteilnehmer angeschlossen werden. Die Besonderheit liegt darin, dass über ein Leitungssystem mehr als zwei Geräte miteinander kommunizieren können. Mittels eines Adressierungsschemas können die Empfänger ausgewählt werden.},see={signal}}

\newglossaryentry{ereignis}{name={Ereignis},plural={Ereignisse},description={Eine Abfolge von Messwerten, die einer vordefinierten Form entspricht. Es kann zum Beispiel ein Schwellenwert (engl. threshold) definiert sein. Das Überschreiten dieses Wertes kann dann den Beginn, das Unterschreiten des Schwellenwertes das Ende eines Ereignisses markieren.}}

\newglossaryentry{ereignisdet}{name={Er\-eig\-nis\-er\-ken\-nung},plural={Ereigniserkennungen},description={Auswertung von Messdaten um definierte Signalformen (\glspl{ereignis}) zu erkennen.},see={ereignis,signal}}

\newglossaryentry{threshold}{name={Threshold},plural={Thresholds},description={Englisch für Schwellenwert.}}

\newglossaryentry{timeout}{name={Timeout},plural={Timeouts},description={Englisch für Zeitüberschreitung.}}

\newglossaryentry{peak}{name={Peak},plural={Peaks},description={Englisch für Spitzenwert oder Signalspitze.}}

\newglossaryentry{signal}{name={Signal},plural={Signale},description={Ein Signal ist ein Informationsträger. Die Information wird einem Signalwert zugeordnet. Ein einfaches Beispiel ist die Spannung am Ausgang eines Beschleunigungs-Sensors. Der Sensor gibt über die Höhe der Spannung an, wie stark die Beschleunigung von einem definierten Referenzwert abweicht. Die Spannung trägt also eine Information über den Messwert und ist daher ein Signal. Aus dem Datenblatt kann herausgelesen werden, wie hoch die ausgegebene Spannung bei einer bestimmten Beschleunigung ist. So kann das Signal in Information umgewandelt werden.},see={sensor}}

\newglossaryentry{timestamp}{name={Timestamp},plural={Timestamps},description={Englisch für Zeitmarke. Mittels eines Timestamps kann ein Ereignis oder ein Messwert einem genauen Zeitpunkt zugeordnet werden. Der Timestamp wird dafür zu einem bestimmten Zeitpunkt auf null gesetzt (Reset) und in allen Messgeräten in vordefiniertem Takt erhöht. Der Timestamp gibt an, wie viel Zeit seit dem Reset vergangen ist. Durch die Wahl des Takts wird die zeitliche Auflösung definiert.}}

\newglossaryentry{compi}{name={Computer},description={Englisch für Elektronenrechner. Unter einem Computer versteht man umgangssprachlich einen Personal Computer (PC) oder einen tragbaren Computer (Laptop). Heute sind Computer so leistungsfähig und so stark miniaturisiert, dass sie mühelos in einer Aktentasche Platz finden. Weniger leistungsfähig, dafür noch kleiner sind Embedded Systems, eingebettete Systeme, die von Laien nicht als Computer erkannt werden.},see={pc,es}}

\newglossaryentry{pcomp}{name={PC},plural={PCs},description={\gls{pc} oder umgangssprachlich Computer. Eine elektronische Rechenmaschine mit der sehr viele Rechenschritte in sehr kurzer Zeit ausgeführt werden können. Mit der richtigen Programmierung (Software) können \gls{pc}s sehr unterschiedliche, umfangreiche Aufgaben lösen.},see={software}}

\newglossaryentry{es}{name={Embedded System},plural={Embedded Systems},description={Deutsch: eingebettetes System. Ein Computer, der nicht über die üblichen Ein- und Ausgabemöglichkeiten eines Computers wie Bildschirm und Tastatur verfügt. Oft werden deshalb Embedded Systems nicht als Computer wahrgenommen. Sie sind im Gegensatz zu PCs, die als Alleskönner konzipiert sind, auf eine bestimmte Aufgabe zugeschnitten und deshalb nur mit den nötigen Bedien-Elementen versehen. Oft genügen für die Aufgaben weniger leistungsfähige Prozessoren als in einem PC, sogenannte Mikroprozessoren.},see={compi,mc}}

\newglossaryentry{software}{name={Software},description={Programm zur Steuerung der Rechenabläufe in einem Computer. Durch die Software wird einem Computer die Fähigkeit gegeben, bestimmte Aufgaben zu lösen und Resultate in einer für Menschen lesbaren Form darzustellen. Die Software enthält Instruktionen, die in der CPU des Computers ausgeführt werden.}}

\newglossaryentry{hardware}{name={Hardware},description={Die Hardware ist das eigentliche Rechenwerk eines Computers, worauf die Software ausgeführt wird. Dazu gehören alle elektrischen, elektronischen und mechanischen Bauteile eines Computers. Die Central Processing Unit (CPU) eines Computers könnte mit dem Hirn verglichen werden, hier laufen fast sämtliche Informationen und Instruktionen zusammen.}}

\newglossaryentry{mc}{name={Mikrokontroller},plural={Mikrokontroller},description={Englisch Microcontroller (MC). Ein Prozessor mit weniger universellen Fähigkeiten als eine CPU, mit entsprechend geringerem Stromverbrauch. Ein Mikrokontroller verfügt meistens über spezielle Ein- und Ausgänge (Pins), über die zum Beispiel die anliegende Spannung gemessen oder eine bestimmte Spannung ausgegeben werden kann. Durch die kleinere Bauform und die geringere Leistungsaufnahme eignen sich diese Prozessoren besonders für den Einsatz in Embedded Systems, die oft längere Zeit unabhängig vom Stromnetz funktionieren müssen.},see={pin}}

\newglossaryentry{cli}{name={Kommandozeile},plural={Kommandozeilen},description={Eine Eingabeaufforderung auf dem Bildschirm, wo der Benutzer über eine Tastatur Befehle eingeben kann, die vom Computer interpretiert und ausgeführt werden.},see={compi}}

\newglossaryentry{pin}{name={Pin},plural={Pins},description={Ein Anschluss an einem Chip oder einer Leiterplatte. Über Pins werden elektronische Bauteile miteinander verbunden und Peripheriegeräte wie z.B. Sensoren angeschlossen.},see={sensor}}

\newglossaryentry{modus}{name={Betriebsmodus},plural={Betriebsmodi},description={Ein Modus bestimmt die Verhaltensweise eines Systems. Je nach gewähltem Modus können mit den gleichen Eingaben und Befehlen andere Aktionen ausgeführt und andere Resultate ausgegeben werden.}}

\newglossaryentry{fs}{name={Abtastrate},plural={Abtastraten},description={Definiert, in welchen zeitlichen Abständen ein Messwert erfasst werden soll. Üblicherweise wird dieser Wert in \ensuremath{Hz} oder \ensuremath{s^{-1}} angegeben.}}

\newglossaryentry{busmaster}{name={Busmaster},plural={Busmaster},description={Ein Gerät, das die Kontrolle über ein Bussystem hat. Der Busmaster kann den anderen Busteilnehmern (Slaves) eine Genehmigung erteilen, Daten über das Bussystem zu übertragen. Der Busmaster hat aber jederzeit die Möglichkeit, einen Slave in der Übertragung zu unterbrechen.},see={slave}}

\newglossaryentry{slave}{name={Slave},plural={Slaves},description={Ein Busteilnehmer, der nur Daten über den Bus übertragen darf, wenn ihm der Busmaster die Genehmigung dafür erteilt. Dies kann z.B. in Form eines sog. Tokens geschehen.},see={busmaster}}

\newglossaryentry{adwandler}{name={A/D-Wandler},plural={A/D-Wandler},description={Analog/Digital-Wandler, (Englisch: \gls{adc}). Ein A/D-Wandler misst die Spannung, die an einem Pin anliegt und gibt einen digitalen Wert aus, der die Höhe der Spannung angibt. Bei der Umwandlung in einen digitalen Wert erfolgt eine Quantisierung. Je grösser die Bit-Breite des A/D-Wandlers ist, umso kleiner wird die Schrittgrösse von einem digitalen Wert zum nächst höheren Wert.},see={quantis,bitbreite}}

\newglossaryentry{quantis}{name={Quantisierung},plural={Quantisierung},description={Einteilung eines Werts aus einer kontinuierlichen Skala in eine abgestufte Skala. Bei der Quantisierung wird der nächstgelegene Wert auf der abgestuften Skala ausgewählt. Je nach Anwendung wird die abgestufte Skala mit konstanter Stufengrösse (linear) gewählt, oder mit unterschiedlichen Stufengrössen je nach absolutem Wert (z.B. exponentiell). Der Quantisierungsfehler entspricht der Differenz zwischen dem analogen und dem quantisierten, digitalen Wert. Je grösser die Bit-Breite der Quantisierung ist, desto kleiner wird der Quantisierungsfehler.},see={bitbreite}}

\newglossaryentry{bitbreite}{name={Bit-Breite},plural={Bit-Breiten},description={Die Bit-Breite gibt an, wie viele Bit für die Darstellung eines Wertes verwendet werden. Je grösser die Bit-Breite, desto mehr unterschiedliche Werte können dargestellt werden. Mit einer Bit-Breite von \ensuremath{n} können \ensuremath{2^n} Werte dargestellt werden.}}

\newglossaryentry{blockg}{name={Blockgrösse},plural={Blockgrössen},description={Anzahl Werte, die in einer \gls{dft} oder \gls{idft} verrechnet wird.}}

\newglossaryentry{hilbert}{name={Hilbert-Transformation},plural={Hilbert-Transformationen},description={Mathematische Umrechnung einer Wertfolge, um die umhüllende Kurve zu erhalten.}}

\newglossaryentry{fir}{name={FIR-Filter},plural={FIR-Filter},description={Finite Impulse Response, Englisch für Filter mit endlicher Impuls-Antwort. Ein FIR-Filter hat immer eine endliche Impulsantwort, d.h. auf einen kurzen Impuls am Eingang des Filters folgt am Ausgang eine Antwort des Filters, die garantiert endet. Die Ordnung des FIR-Filters gibt an, wie lange die Antwort dauern wird.}}

\newglossaryentry{fsmgloss}{name={Finite State Machine},description={Englisch für Zustandsmaschine. Eine Finite State Machine definiert eine endliche Anzahl Zustände, die die Maschine einnehmen kann. Die \gls{fsm} reagiert auf Ereignisse, indem sie Aktionen auslöst und allenfalls in einen anderen Zustand wechselt. Für jeden Zustand ist definiert, welche möglichen Ereignisse welche Aktionen auslösen, und in welchen Folgezustand gewechselt werden soll.}}

\newglossaryentry{nullpegel}{name={Nullpegel},description={Signalpegel, wenn keine Beschleunigung gemessen wird.}}

\newglossaryentry{queue}{name={Queue},plural={Queues},description={Englisch für Warteschlange. In der Informationstechnologie wird eine Queue verwendet, um mehrere Werte zwischenzuspeichern, bis sie verarbeitet werden können. Eine Queue hat meistens eine vordefinierte Grösse und kann deshalb nur eine bestimmte Anzahl Werte aufnehmen. Ein Überlaufen der Queue hat einen Datenverlust zur Folge, wenn der Werte einfüllende Prozess nicht angehalten werden kann/darf, bis Platz in der Queue vorhanden ist.}}

\newglossaryentry{dspgloss}{name={DSP},plural={DSP},description={Digitaler Signal-Prozessor. Ein Mikroprozessor oder ein Bestandteil eines solchen, der dank spezieller Hardware fähig ist, Multiplikationen und Additionen extrem schnell auszuführen. Ausserdem verfügt ein \gls{dsp} über Schieberegister, um Rechenoperationen über eine Serie der aktuellsten Messwerte auszuführen. Da bei der digitalen Signalverarbeitung oft für jeden Messwert viele Multiplikations- und Additions-Schritte ausgeführt werden müssen, ist es wichtig, dass Multiplikations-Operationen in einem Taktzyklus ausgeführt werden können. In einem normalen Prozessor werden für eine Multiplikation mehrere Taktzyklen benötigt.},see={mc}}

\newglossaryentry{nvicgloss}{name={Nested Vectored Interrupt Controller},description={Wird benötigt, um möglichst rasch auf mehrere asynchron eintreffende Ereignisse reagieren zu können. Ein Prozessor mit NVIC kann auf verschiedene Ereignisse reagieren, indem Interrupts ausgelöst werden. Jedem Interrupt kann eine Priorität zugewiesen werden, um festzulegen, ob ein Interrupt einen anderen unterbrechen darf, der gerade vom Prozessor abgearbeitet wird.},see={interrupt}}

\newglossaryentry{macgloss}{name={MAC},description={Multiply-ACcumulate unit: Ein Rechenwerk in der CPU, wo Multiplikationen und Additionen ausgeführt werden.}}

\newglossaryentry{fpugloss}{name={FPU},description={Floating Point Unit, ein Rechenwerk in der CPU, die Berechnungen mit Dezimalbrüchen sehr schnell ausführen kann.}}

\newglossaryentry{sdram}{name={SDRAM},description={Synchronous Dynamic RAM. Flüchtiger Arbeitsspeicher. Der Inhalt des Arbeitsspeichers muss regelmässig aufgefrischt werden, sonst geht die Information verloren.},see={RAM}}

\newglossaryentry{flash}{name={Flash-Speicher},description={Nicht-flüchtiger Speicher. Der Inhalt des Speichers bleibt auch bestehen, wenn keine Versorgungsspannung anliegt.}}

\newglossaryentry{ram}{name={RAM},description={Random Access Memory. Arbeitsspeicher mit Zugriff auf beliebige Adressen. Benötigt keine Positionierung eines Lese-/Schreib-Kopfs oder eines Magnetbands vor einem Zugriff.}}

\newglossaryentry{polling}{name={Polling},description={Englisch für Abfragen. Bezeichnet das wiederholte Abfragen einer Funktion oder eines Wertes, bis ein bestimmtes Ereignis auftritt. Der Prozessor ist mit der Abfrage beschäftigt und verliert dadurch Rechenzeit.}}

\newglossaryentry{interrupt}{name={Interrupt},plural={Interrupts},description={Englisch für Unterbrechen. Durch ein spezielles Signal wird dem Prozessor mitgeteilt, dass ein Ereignis eingetreten ist. Der Prozessor unterbricht die laufende Funktion und ruft eine spezielle Routine, die \gls{isr} auf. Die \gls{isr} verarbeitet das Ereignis und gibt danach die Kontrolle an die vorher unterbrochene Funktion zurück. Mit einem Interrupt wird zwar die Abarbeitung einer Funktion kurzzeitig unterbrochen, dafür verliert der Prozessor keine Rechenzeit mit Polling.},see={polling}}

\newglossaryentry{cdetgloss}{name={Collision Detection},description={Kollisionserkennung. Als Kollision bezeichnet man den gleichzeitigen Versuch mehrerer Busteilnehmer, eine Nachricht zu übermitteln. Dies führt zu einer Überlagerung der Nachrichten und macht diese unlesbar. Gleichzeitig mit dem Schreiben liest der Transceiver die Signale vom Bus. Stimmen die gelesenen Signale nicht mit den Geschriebenen überein, bedeutet dies, dass ein anderer Teilnehmer ebenfalls auf den Bus schreibt und eine Kollision vorliegt. Das Verhalten bei einer Kollision ist vom Bussystem abhängig.},see={transceiver}}

\newglossaryentry{transceiver}{name={Transceiver},plural={Transceiver},description={Ein Gerät das gleichzeitig Transmitter und Receiver, also Sender und Empfänger ist. Der Prozessor sendet die zu übertragenden Daten über eine \gls{serial} an den Transceiver. Der Transceiver erzeugt dann die notwendigen elektrischen Signale auf dem Bus, um die Daten zu übertragen. Beim Empfangen einer Meldung liest der Transceiver die Signale vom Bus und übersetzt sie für die serielle Übertragung zum Prozessor. Ein Transceiver ist notwendig, wenn der Prozessor die vom Bus erwarteten elektrischen Signale nicht selbst erzeugen kann.},see={serial}}

\newglossaryentry{serial}{name={serielle Schnittstelle},plural={serielle Schnittstellen},description={Eine Schnittstelle, über die Daten bitweise übertragen werden, im Gegensatz zu paralleler Übertragung, wo auf mehreren Leitungen mehrere Bits gleichzeitig übertragen werden.}}

\newglossaryentry{daisy}{name={Daisy Chain},description={Englisch für Gänseblümchenkette. Gemeint ist das Aneinanderreihen mehrerer Geräte. Im Unterschied zu einem Bussystem müssen die Geräte alle Nachrichten, die für andere Empfänger bestimmt sind, aktiv weiterleiten. Dies braucht Rechenleistung in den Geräten. Wenn ein Gerät ausfällt, gehen alle Nachrichten in diesem Gerät verloren.}}

\newglossaryentry{crcgloss}{name={CRC},description={Cyclic Redundancy Check. Ein Codeverfahren, das eine Prüfsumme über eine Nachricht berechnet. Durch Nachrechnen der Prüfsumme im Empfänger kann die Nachricht auf Fehlerfreiheit überprüft werden.}}

\newglossaryentry{terminal}{name={Terminal},plural={Terminals},description={Eine Arbeitsstation, von der aus auf einen zentralen Rechner zugegriffen wurde, normalerweise mit einer Kommandozeile zur Befehlseingabe.}}

\newglossaryentry{terminalemu}{name={Terminal-Emulator},plural={Terminal-Emulatoren},description={Ein Programm, das ein Terminal emuliert. Der Benutzer sieht ein Fenster mit einer Kommandozeile zur Befehlseingabe, das aussieht wie auf einem Terminal.},see={terminal}}

\newglossaryentry{stichleitung}{name={Stichleitung},description={Abzweigungsleitung an einer Busleitung. Um das Signal vom Bus zum Transceiver zu führen, wird eine Stichleitung benötigt, die wie eine T-Kreuzung von der Busleitung wegführt. Je länger die Stichleitung ist, desto stärker leider die Signalqualität auf der Busleitung.},see={transceiver}}

\newglossaryentry{terminator}{name={Abschlusswiderstand},plural={Abschlusswiderstände},description={Bei elektrischen Signalübertragungen treten am Ende der Leitung Signalreflexionen auf. Die Reflexionen pflanzen sich in der Leitung in entgegengesetzter Richtung fort und stören so die Übertragung. Um die Reflexionen zu verhindern, können Widerstände eingesetzt werden, die die elektrische Energie in Wärme umwandeln. Dadurch treten keine Reflexionen mehr auf.}}

\newglossaryentry{sample}{name={Sample},plural={Samples},description={Englisch für Messwert, Messpunkt.}}

\newglossaryentry{mems}{name={MEMS},description={Microelectromechanical System, Englisch für Mikroelektromechanisches System. Bezeichnung für Elektromechanische Systeme, die sehr stark miniaturisiert worden sind. MEMS-Geräte verwenden Bauteile, die nur wenige Mikrometer bis einen Millimeter gross \cite{wiki_mems}.}}

\newglossaryentry{Hydrologie}{name={Hydrologie},description={Wissenschaft über das Wasser.}}

\newglossaryentry{Hydrophon}{name={Hydrophon},plural={Hydrophone},description={}}

\newglossaryentry{fifo}{name={FIFO-Queue},description={First In First Out Queue. Bezeichnung für eine Warteschlange, bei der der älteste Eintrag als nächstes herauskommt. Im Gegensatz zu LIFO (Last In First Out), wo jeweils der jüngste Eintrag zuerst ausgelesen wird.}}

\newglossaryentry{geschiebekorn}{name={Geschiebkorn},plural={Geschiebekörner},description={Ein Stein oder Kiesel, der vom Fluss transportiert wird. Die Grösse der Steine wird als Korngrösse bezeichnet.}}

\newglossaryentry{thread}{name={Thread},plural={Threads},description={Englisch für Faden. In der IT bezeichnet Thread einen Programmteil. Mehrere Threads werden vom Kernel verwaltet. Der Kernel erlaubt einem Thread die Nutzung der CPU für eine gewisse Zeit, danach wird der Thread unterbrochen und ein anderer Thread darf die CPU nutzen. Der Kernel sorgt so dafür, dass die Threads abwechslungsweise und in rascher Abfolge zur Ausführung kommen. Es entsteht der Eindruck, dass mehrere Threads parallel laufen.}}