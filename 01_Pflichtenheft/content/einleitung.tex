%%%%%%%%%%%%%%%%%%%%%%%%%%%%%%%%%%%%%%%%%%%%%%%%%%%%%%%%%%%%%%%%%
%  _____   ____  _____                                          %
% |_   _| /  __||  __ \    Institute of Computitional Physics   %
%   | |  |  /   | |__) |   Zuercher Hochschule Winterthur       %
%   | |  | (    |  ___/    (University of Applied Sciences)     %
%  _| |_ |  \__ | |        8401 Winterthur, Switzerland         %
% |_____| \____||_|                                             %
%%%%%%%%%%%%%%%%%%%%%%%%%%%%%%%%%%%%%%%%%%%%%%%%%%%%%%%%%%%%%%%%%
%
% Project     : LaTeX doc Vorlage für Windows ProTeXt mit TexMakerX
% Title       : 
% File        : einleitung.tex Rev. 00
% Date        : 23.4.12
% Author      : Remo Ritzmann
% Feedback bitte an Email: remo.ritzmann@pfunzle.ch
%
%%%%%%%%%%%%%%%%%%%%%%%%%%%%%%%%%%%%%%%%%%%%%%%%%%%%%%%%%%%%%%%%%

\chapter{Einleitung}\label{chap.einleitung}



\section{Ausgangslage}\label{ausgangslage}
Das WSL betreibt Messstationen zur Registrierung von Geschiebe-Bewegungen im Fluss mittels Geophonen, die unter Stahlplatten montiert sind. Diese Platten sind in einer Betonkonstruktion eingelassen, um sie im Flussbett zu fixieren. Die Geophone sind über Kabel mit einem Auswertungs-Rechner (Embedded PC) verbunden, der die Signale auswertet. Die baulichen Massnahmen für die Installation der Sensoren, der Auswertungsstation sowie der Stromversorgung sind sehr teuer. Zukünftig sollen die Geophone durch eindimensionale MEMS Beschleunigungssensoren ersetzt werden, da diese kleiner sind.

\begin{itemize}
\item Nennt bestehende Arbeiten/Literatur zum Thema -> Literaturrecherche
\item Stand der Technik: Bisherige Lösungen des Problems und deren Grenzen
\item (Nennt kurz den Industriepartner und/oder weitere Kooperationspartner und dessen/deren Interesse am Thema Fragestellung)
\end{itemize}



\section{Zielsetzung / Aufgabenstellung / Anforderungen}\label{zielsetzung}
Um die Kosten zu senken und die zu übertragende Datenmenge zu reduzie-ren, soll die Auswertung der Daten direkt am Sensor erfolgen. Somit könnten die Daten über ein Bussystem übertragen werden und der Auswertungsrech-ner bräuchte weniger Leistung.
Dank der Bustopologie ist das Messsystem weniger komplex und kann einfa-cher installiert werden. Denkbar wäre die Integration in einer Gummimatte anstelle der Stahl- und Betonkonstruktion, da viel weniger Leitungen nötig sind.
Ziel der Arbeit ist die Entwicklung der Auswertungshardware und des Bussys-tems. Die Auswertungsalgorithmen sind nicht Bestandteil der Arbeit und wer-den vom WSL zur Verfügung gestellt.
Denkbar wäre es, einen Prototyp für Vergleichsmessungen im Erlenbach (Alptal, SZ) an einer bestehenden Schwelle zu implementieren.
\begin{itemize}
\item Formuliert das Ziel der Arbeit
\item Verweist auf die offizielle Aufgabenstellung des/der Dozierenden im Anhang
\item (Pflichtenheft, Spezifikation)
\item (Spezifiziert die Anforderungen an das Resultat der Arbeit)
\item (Übersicht über die Arbeit: stellt die folgenden Teile der Arbeit kurz vor)
\item (Angaben zum Zielpublikum: nennt das für die Arbeit vorausgesetzte Wissen)
\item (Terminologie: Definiert die in der Arbeit verwendeten Begriffe)
\end{itemize}
