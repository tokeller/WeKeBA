%%%%%%%%%%%%%%%%%%%%%%%%%%%%%%%%%%%%%%%%%%%%%%%%%%%%%%%%%%%%%%%%%
%  _____   ____  _____                                          %
% |_   _| /  __||  __ \    Institute of Computitional Physics   %
%   | |  |  /   | |__) |   Zuercher Hochschule Winterthur       %
%   | |  | (    |  ___/    (University of Applied Sciences)     %
%  _| |_ |  \__ | |        8401 Winterthur, Switzerland         %
% |_____| \____||_|                                             %
%%%%%%%%%%%%%%%%%%%%%%%%%%%%%%%%%%%%%%%%%%%%%%%%%%%%%%%%%%%%%%%%%
%
% Project     : LaTeX doc Vorlage für Windows ProTeXt mit TexMakerX
% Title       : 
% File        : einleitung.tex Rev. 00
% Date        : 23.4.12
% Author      : Remo Ritzmann
% Feedback bitte an Email: remo.ritzmann@pfunzle.ch
%
%%%%%%%%%%%%%%%%%%%%%%%%%%%%%%%%%%%%%%%%%%%%%%%%%%%%%%%%%%%%%%%%%

\chapter{Einleitung}\label{chap.einleitung}



\section{Ausgangslage}\label{sec.ausgangslage}
Die Eidg. Forschungsanstalt für Wald, Schnee und Landschaft (WSL) betreibt Messstationen zur Registrierung von Geschiebe-Bewegungen im Fluss mittels Geophonen, die unter Stahlplatten montiert sind. Diese Platten sind in einer Betonkonstruktion eingelassen, um sie im Flussbett zu fixieren. Die Geophone sind über Kabel mit einem Auswertungs-Rechner (Embedded PC) verbunden, der die Signale auswertet. Die baulichen Massnahmen für die Installation der Sensoren, der Auswertungsstation sowie der Stromversorgung sind sehr teuer. 

\begin{itemize}
\item Nennt bestehende Arbeiten/Literatur zum Thema -> Literaturrecherche
\item Stand der Technik: Bisherige Lösungen des Problems und deren Grenzen
\item (Nennt kurz den Industriepartner und/oder weitere Kooperationspartner und dessen/deren Interesse am Thema Fragestellung)
\end{itemize}



\section{Zielsetzung}\label{sec.zielsetzung}
Im Rahmen dieser Bachelorarbeit soll eine Lösung erarbeitet werden, um zukünftige Installationen günstiger zu machen. Da solche Messanlagen an sehr vielen Orten auf der ganzen Welt aufgebaut werden, kann durch eine Vereinfachung der Installation viel Aufwand gespart werden. Die Projektidee stammt von Bruno Fritschi (WSL). Sein Vorschlag sieht vor, die aufgezeichneten Signale direkt am Sensor auszuwerten und nur die gewünschten Ereignisse zu speichern.

Die Miniaturisierung des Sensors ist eines der Hauptziele dieser Arbeit, da dadurch die baulichen Massnahmen einer Installation viel geringer ausfallen. Weiter soll der Stromverbrauch gesenkt werden, um die Anlagen durch erneuerbare Energien, die vor Ort erzeugt werden, versorgt werden können. Die Qualität der gemessenen Signale soll mindestens erhalten werden.

Die von der bisherigen Anlage gemachten Messdaten enthalten die Dauer und Intensität jedes Aufschlags (Ereignis) auf der Sensorplatte, sowie die Anzahl Ausschläge (Peaks) pro Aufschlag. Pro Minute wird ein Histogramm über die Intesitäten der Peaks gebildet und abgespeichert.

\subsection{Musskriterien}
\begin{itemize}
\item Die Anlage zeichnet den Geschiebetransport im Bachbett auf. Die bisherige Aufzeichnungsrate von 10'000 Messpunkten pro Sekunde soll nicht unterschritten werden.
\item Die Anlage liefert eine minütliche Zusammenfassung über die Ereignisse an jedem Sensor. Diese Zusammenfassung enthält die Anzahl, Dauer und Intensität der einzelnen Ereignisse sowie ein Histogramm über die Intensitätsverteilung.
\item Die Messstation ist fähig, mindestens zehn Sensoren zu betreiben und ihre Messignale aufzuzeichnen.
\item Es ist möglich, die kompletten Rohdaten von einem Sensor über eine Dauer von XXX (Bruno?) Minuten/Stunden aufzuzeichnen. Während einer solchen Messung dürfen die anderen Sensoren ihre Messung einstellen.
\item Die Sensoren können über bis zu XXX (Bruno?) Meter im Bachbett verteilt sein.
\item Die Leistungsaufnahme der Anlage beim Betrieb von 10 Sensoren ist kleiner als XXX (Bruno?) Watt.
\item
\end{itemize}
\subsection{Wunschkriterien}
\begin{itemize}
\item Die Anlage liefert für jedes Ereignis die Rohdaten in voller zeitlicher Auflösung.
\item Der Sensoraufbau ermöglicht es, die Sensoren in einer Elastomerplatte zu verpacken. Die Elastomerplatte kann ohne Betonkonstruktion im Bachbett verankert werden.
\item 
\end{itemize}
\subsection{Abgrenzungskriterien}
\begin{itemize}
\item Es würde den Rahmen dieser Arbeit sprengen, die Messeinheiten zur Produktreife zu bringen. Es wird lediglich aufgezeigt, wie solche Messeinheiten realisiert werden könnten.
\item Eine Testinstallation in einem Bach ist nicht möglich. Allenfalls kann in der Versuchsanstalt für Wasserbau, Hydrologie und Glaziologie der ETH Zürich ein kleiner Testlauf stattfinden.
\item
\end{itemize}

AB HIER NUR NOCH PROVISORIUM:
===============================

Zukünftig sollen die Geophone durch eindimensionale MEMS Beschleunigungssensoren ersetzt werden, da diese kleiner sind.
Die Kosten für zukünftige Anlagen sollen gesenkt werden. Dazu gibt es verschiedene Ansätze.
\begin{itemize}
\item Bauliche Massnahmen verringern
\item Einfachere Sensorkonstruktion wählen
\item Stromverbrauch senken
\end{itemize}

\subsection{Bauliche Massnahmen}
Die Geophone werden in der bestehenden Konstruktion unter Stahlplatten montiert, um sie im Bachbett zu verankern. Die Stahlplatten sind in einer Betonrinne fixiert, die zugleich als Kabelkanal für die Signalkabel dient. Da die Messstationen häufig im Gebirge oder schwer zugänglichem Gelände erstellt werden, ist schon der Bau der Betonrinne sehr teuer.

Mit einer Konstruktion aus Elastomerplatten, die im Bachbett verankert werden, könnte man diese Kosten senken. Der Schutz der Kabel darf natürlich nicht beeinträchtigt sein.

\subsection{Sensorkonstruktion}
Jedes Geophon ist über ein Kabel mit dem Auswertungsrechner verbunden. Der Rechner wertet die Signale aller angeschlossen Geophone kontinuierlich aus, um die Ereignisse zu detektieren. Bei mehreren Geophonen ist hier ein recht leistungsfähiger Rechner nötig, der eine entsprechend hohe Leistungsaufnahme hat. 

\subsection{Stromverbrauch}
XXX sensor und stromverbrauch hängen zusammen.... XXX

 und die zu übertragende Datenmenge zu reduzieren, soll die Auswertung der Daten direkt am Sensor erfolgen. Somit könnten die Daten über ein Bussystem übertragen werden und der Auswertungsrechner bräuchte weniger Leistung.
Dank der Bustopologie ist das Messsystem weniger komplex und kann einfa-cher installiert werden. Denkbar wäre die Integration in einer Gummimatte anstelle der Stahl- und Betonkonstruktion, da viel weniger Leitungen nötig sind.
Ziel der Arbeit ist die Entwicklung der Auswertungshardware und des Bussys-tems. Die Auswertungsalgorithmen sind nicht Bestandteil der Arbeit und wer-den vom WSL zur Verfügung gestellt.
Denkbar wäre es, einen Prototyp für Vergleichsmessungen im Erlenbach (Alptal, SZ) an einer bestehenden Schwelle zu implementieren.





================= bla von der vorlage =============
\begin{itemize}
\item Formuliert das Ziel der Arbeit
\item Verweist auf die offizielle Aufgabenstellung des/der Dozierenden im Anhang
\item (Pflichtenheft, Spezifikation)
\item (Spezifiziert die Anforderungen an das Resultat der Arbeit)
\item (Übersicht über die Arbeit: stellt die folgenden Teile der Arbeit kurz vor)
\item (Angaben zum Zielpublikum: nennt das für die Arbeit vorausgesetzte Wissen)
\item (Terminologie: Definiert die in der Arbeit verwendeten Begriffe)
\end{itemize}
